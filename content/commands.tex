 \newtheorem{theorem}{Теорема}[chapter]
 \newtheorem*{theorem*}{Теорема}
 \newtheorem*{lemma*}{Лемма}
 \newtheorem{property}{Свойство}
 \newtheorem{lemma}{Лемма}[chapter]
 \newtheorem{statement}{Утверждение}[chapter]
 \newtheorem{definition}{Определение}[chapter]
 \newtheorem{example}{Пример}[chapter]
 \newtheorem{corollary}{Следствие}[chapter]
 \newtheorem*{corollary*}{Следствие}
 \newtheorem{remark}{Замечание}[chapter]
 \newtheorem*{remark*}{Замечание}
 \newtheorem{hypothesis}{Гипотеза}[chapter]

 \newtheorem{cond}{Условие}
\newtheorem{theoremA}{Теорема}
 \newtheorem{lemmaA}{Лемма}[section]
 \newtheorem{state}{Предложение}
 \newtheorem{proposition}{Предложение}
\renewcommand{\thetheoremA}{\Alph{theoremA}}
 \renewcommand{\thelemmaA}{\thesection.\Alph{lemmaA}}
 \newcommand{\No}{\textnumero}

 \newcommand{\norm}[1]{\|#1\|_{p(\cdot),w}}
 \newcommand{\ip}[2]{\langle #1, #2 \rangle}

 \DeclareMathOperator*{\esssup}{ess\,sup}
 \DeclareMathOperator*{\essinf}{ess\,inf}

 \numberwithin{equation}{chapter} %
 \renewcommand{\theequation}{\thechapter.\arabic{equation}}

 %\newenvironment{description}{}{}

% \newcommand{\ifNotEmpty}[2] {
%     \ifx&#1& \empty \else #2 \fi
% }
%
% \newcommand{\row}[3][]{
%     \small{#2}&
%     \centering{\rule[-2mm]{5cm}{0.2mm}} \newline \centering \footnotesize{{подпись, дата}} &
%     #3 \newline \small{{\ifNotEmpty{#1}{(#1)}}} \\ & & \\
% }
