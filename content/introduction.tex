
На протяжении более трех десятилетий исследование систем полиномов, ортогональных относительно скалярного произведения Соболева вызывает огромный интерес. Отчасти это связано с тем, что соболевские скалярные произведения и соответствующие им ортогональные системы (и их дифференциальные аналоги) играют важную роль во многих проблемах теории функций, квантовой механики, математической физики, вычислительной математики и т.д. Ряды Фурье по ним обладают важными для приложений свойствами, которые отсутствуют у рядов Фурье по классическим ортогональным системам (см. \cite{Ram-Ba-Ra-Pe, Ram-Mar-Xu, mmg-SharapudinovUMN}).
Например, ряды Фурье по соболевским полиномам оказываются более естественным, чем ряды Фурье по классическим ортогональным полиномам, аппаратом для приближенного решения краевых задач, в которых требуется контроль поведения приближенного решения в одной или нескольких точках.
При решении таких задач важную роль играют вопросы сходимости и скорости сходимости рядов Фурье в различных функциональных пространствах. В отчетном году в рамках научно-исследовательской работы рассмотрены вопросы скорости сходимости частичных сумм ряда Фурье по полиномам Лагерра -- Соболева.


Продолжены исследования вопроса базисности системы полиномов Лежандра и Якоби в пространствах Лебега с переменным показателем. Интерес к таким пространствам возрос с 1990-х годов ввиду их использования в различных приложениях (математическое моделирование электрореологических жидкостей, обработка зашумленных изображений и др.), которые, в свою очередь, приводят к поиску систем, образующих базисы в указанных пространствах. В работах И.И. Шарапудинова и его учеников \cite{tad-SHII-Haar, tad-SHII-AnalisysMath, tad-SHII-Leg, tad-MMG-Haar, tad-SHII-Jacob, tad-SHII-Ult, tad-RAM-Jacob} была показана базисность различных систем в пространстве Лебега с переменным показателем $p(x)$, удовлетворяющим определенным условиям.
В частности, в \cite{tad-SHII-Leg} было показано, что если $p(x)$ принадлежит классу переменных показателей $\mathcal{P}(-1,1)$ (см. п.\ref{Intro-Tadg}),
то при $p(\pm1)\in (4/3, 4)$ система полиномов Лежандра образует базис в пространстве $L^{p(\cdot)}=L^{p(\cdot)}([-1,1])$. В отчетном году нам удалось усилить этот результат, а именно, избавиться от условия постоянства переменного показателя в окрестностях точек $\pm1$.

В случае полиномов Якоби $P_n^{\alpha,\beta}(x)$ удалось найти условия на переменный показатель $p(x)\in \mathcal{P}(-1,1)$, обеспечивающие базисность системы $\{P_n^{\alpha,\beta}(x)\}$ в весовом пространстве Лебега с переменным показателем $L^{p(\cdot)}_{w_{a,b}}$. При этом эти условия аналогичны условиям Мукенхаупта для базисности полиномов $P_n^{\alpha,\beta}(x)$ в пространстве Лебега $L^{p}_{w_{a,b}}$ с постоянным показателем.


В отчетном году были исследованы вопросы, связанные с приложением рациональных сплайнов к решению дифференциальных уравнений вида
\begin{equation}\label{eq:intro-0.1}
a_0(x) y^{\prime\prime}+a_1(x) y^\prime+q(x) y=f(x),\quad a< x<b,
\end{equation}
где $a_0(x)=(x-a)^2(b-x)^2$, $a_1(x)=(x-a)(b-x)p(x)$, функция $p(x)$ непрерывна и ограничена на $(a,b)$, функции $q(x)$ и $f(x)$ непрерывны на $(a,b)$.
Следует отметить, что выбор функций $p(x)$, $q(x)$ и $f(x)$ в \eqref{eq:intro-0.1} дает различные широко используемые в математической физике и во многих других областях дифференциальные уравнения, в частности, уравнение Римана \cite{bib:ark-1, bib:ark-2, bib:ark-3}. Такие уравнения глубоко
исследованы, для определенных видов их коэффициентов разработаны методы решения при помощи степенных рядов и их обобщений.
Известны также приближенные решения частных видов уравнения \eqref{eq:intro-0.1} с различными
определенными ограничениями на их коэффициенты, на функциональный класс допустимого
решения $y(x)$, а также на краевые условия (см., напр.,  \cite{bib:ark-4, bib:ark-5, bib:ark-6, bib:ark-7, bib:ark-8, bib:ark-9, bib:ark-10}).
В отчетном году разработан новый способ построения приближенного решения уравнения
\eqref{eq:intro-0.1} в виде гладких рациональных сплайн-функций, причем структура применяемых
рациональных сплайн-функций позволяет получить сравнительно более простые алгоритмы
построения решения.
В развитие исследований по эффективным приложениям сплайн-функций по рациональным
 интерполянтам к решению дифференциальных уравнений и их систем в работе \cite{bib:ark-19}
 построены непрерывно дифференцируемые рациональные сплайн-решения
 для начальной задачи в случае нормальной системы двух дифференциальных уравнений.
Структура применяемых рациональных сплайн-функций допускает построение сравнительно
простых алгоритмов поиска гладких решений дифференциальных уравнений.

Другим направлением исследований, проводившихся в отчетном году, были исследования вопросов усреднения уравнения Бельтрами на всей плоскости.
Для усреднения эллиптических уравнений достаточно иметь однозначно разрешимую задачу во всем пространстве и соответствующие априорные оценки. Это позволяет изучить различные  аспекты усреднения не только во всем пространстве, но и в ограниченных областях. Однозначно разрешимую задачу в $\mathbb{R}^n$ и соответствующие априорные оценки имеют дивергентные эллиптические уравнения. В случае недивергентных уравнений задачи во всем пространстве мало изучены.
В отчетном году рассмотрена задача на всей плоскости для уравнения Бельтрами. Получены априорные оценки, доказана гипоэллиптичность уравнения Бельтрами с постоянным коэффициентом. Изложены свойства сглаживания функций по Стеклову в $L_p$-пространствах. Полученные результаты в дальнейшем будут использованы нами в вопросах усреднения уравнения Бельтрами на всей плоскости .

Стохастические дифференциальные уравнения описывают многие реальные, практически важные процессы современной физики, биологии, иммунологии, экономики, кибернетики и т.д. Изучение таких процессов привело к необходимости исследований вопросов устойчивости решений стохастических дифференциальных
уравнений, т.е. к созданию соответствующего направления в теории устойчивости. 
Вопросам устойчивости решений систем со случайными параметрами посвящено большое количество работ как отечественных, так и
зарубежных математиков. Исследования устойчивости в этих и многих других
работах проводятся методом функционалов Ляпунова-Красовского-Разумихина. Применение этого метода для функционально-дифференциальных уравнений, частным случаем которых являются дифференциальные уравнения с отклоняющим аргументом, во многих случаях встречает серьёзные трудности. В теории устойчивости решений для детерминированных функционально--дифференциальных
уравнений широкое применение и высокую эффективность показал метод регуляризации, основанный на выборе вспомогательных или "модельных"
{\,} уравнений  --- "W--метод" {\,} Н.В. Азбелева.
Главной целью исследований за отчетный период является развитие метода вспомогательных уравнений на основе теории
неотрицательно обратимых матриц и покомпонентных оценок решений применительно к исследованию вопросов моментной устойчивости решений
по части переменных для систем линейных дифференциальных уравнений Ито с последействием относительно начальных данных. Этот подход
ранее показал свою эффективность при исследовании вопросов устойчивости решений для систем линейных дифференциальных уравнений
Ито с последействием и, как показано в настоящем отчете, позволяет получить новые, конструктивные результаты устойчивости решений по
части переменных относительно начальных данных для детерминированных и стохастических систем с запаздываниями и без него.

В 2023 г. продолжена работа над проблемами интервальной раскрашиваемости и вычисления димерных чисел.
Задача вычисления димерных чисел возникает в исследованиях свойств химических соединений, а также при исследовании адсорбции двухатомных молекул на поверхности.
Известны классы химических соединений, которые синтезируются только тогда, когда графы соединений в топологической модели молекулы имеют совершенное паросочетание; более того, стабильность компонентов этих семейств зависит от количества совершенных паросочетаний в их графах.
Еще более актуальной задача вычисления димерных чисел оказалась в физике; вычисление числа способов объединения атомов в двухатомные молекулы (димеры) с соблюдением некоторых условий и привела известных ученых-физиков Kateleyn P. W., Temperley H. N. V. и Fisher M. E. к знаменитой формуле двойного произведения. Но все известные формулы решения задачи использовали операции с плавающей запятой, что требует значительных компьютерных ресурсов.
Исключение составляет алгоритм, получивший название <<динамическое программирование>>. В отчетном году построен модифицированный алгоритм динамического программирования по профилю: уменьшение вычислительной сложности и оптимизация требуемого объема памяти.


В ОМИ продолжены исследования в области математического моделирования сложных систем. Исследованы задачи восстановления векторных и тензорных полей в трехмерном пространстве и на плоскости по заданным интегральным преобразованиям двух типов. Первое преобразование можно охарактеризовать как множество интегралов от тензорных полей, взятых по лучам, касающимся данной поверхности. Это одно из трех преобразований на трехмерных комплексах лучей, перечисленных в работе \cite{Medzhidov}.
Второе преобразование, называемое еще $V$-лучевым преобразованием, задано на семействах ломаных на плоскости. Идею $V$-образных преобразований Радона с фиксированным направлением оси симметрии выдвинули Трунг Т. и Нгуен М. К. (\cite{Truong}). Такие преобразования могут представлять теоретический интерес в интегральной геометрии. Они возникают в результате связанного томографического процесса пропускания-отражения. Интегральные преобразования по преломленным лучам изучаются и во многих других работах. Из недавних работ можно отметить статьи \cite{Sharafutdinov}, \cite{Ambartsoumian}  и др.
Мы рассматриваем обобщения указанных преобразований на векторные и тензорные поля на плоскости.

Проведены исследования двумерной векторной модели Поттса на треугольной решетке с $q = 5$. Исследование этой модели с антиферромагнитными обменными взаимодействиями на треугольной решетке в литературе практически не встречается. Антиферромагнитное обменного взаимодействие в данной модели может привести к фрустрации, вырождению основного состояния, появлению различных фаз и фазовых переходов, а также влиять на его термодинамические, магнитные и критические свойства. В связи с этим в отчетном году нами предпринята попытка провести исследование фазовых переходов и термодинамических свойств этой модели на треугольной решетке. 