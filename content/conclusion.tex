В 2023 году в Отделе математики и информатики Института физики Дагестанского федерального исследовательского центра РАН по теме НИР были получены следующие основные результаты.

Была исследована задача об оценке отклонения частичных сумм $S_{1,n+1}(f,x)$ ряда Фурье по полиномам Лагерра -- Соболева от функции $f$ из весового пространства Соболева, для которой \linebreak $|f'(x)|e^{-\frac x2}\le 1$.
В частности, получена оценка вида $\frac{x^{-\frac14}e^{-\frac x2}}{\sqrt{x+1}}|f(x)-S_{1,n+1}(f,x)|\le c
\frac{\ln(n+1)}{n^{\frac14}}$, где $c$ -- положительная константа.

Были получены условия базиcности полиномов Якоби ($-1<\alpha,\beta<-1/2$) в пространстве $L^{p(\cdot)}_{w_{a,b}}$. Эти условия аналогичны условиям, полученным Б. Мукенхауптом в \cite{tad-Muckenhoupt}. В случае $w_{a,b}=1$ и полиномов Лежандра удалось усилить результат, полученный в \cite{tad-SHII-Leg}, путем устранения условия постоянства переменного показателя у концов отрезка $[-1,1]$.

Для обобщенного дифференциального уравнения Римана представлен новый метод построения приближенного решения в виде дважды непрерывно
дифференцируемых интерполяционных рациональных сплайн-функций. Найдены условия на коэффициенты дифференциального уравнения, которые обеспечивают
единственность решения трехдиагональной системы линейных алгебраических уравнений для численного решения соответствующей краевой задачи.
Для начальной задачи в случае нормальной системы двух дифференциальных уравнений получено приближенное решение в виде непрерывно дифференцируемых рациональных
сплайн-функций.

Были изучены вопросы моментной устойчивости решений по части
переменных относительно начальных данных для систем линейных
дифференциальных уравнений Ито с последействием. Для упомянутых систем получены достаточные условия
устойчивости в терминах неотрицательной обратимости матриц,
построенных по параметрам этих систем. Также была исследована глобальная моментная устойчивость систем
дифференциальных уравнений Ито дробного порядка с последействием. Кроме того, было изучено  асимптотическое поведение моментов решений систем
нелинейных дифференциальных уравнений Ито с запаздываниями.

Была рассмотрена задача на всей плоскости для уравнения Бельтрами. Получены априорные оценки, доказана гипоэллиптичность уравнения Бельтрами с постоянным коэффициентом. Изложены свойства сглаживания функций по Стеклову в $L_p$-пространствах.

Решены задачи обращения лучевого и $V$-лучевого преобразований векторных и тензорных полей на плоскости и в пространстве. Отличие данных результатов от результатов других авторов состоит в том, что восстанавливается векторное поле целиком, при этом используются моменты первого порядка, а в случае тензорного поля определяется оператор Сен-Венана.
Методы решения приведенных задач могут быть применены при обращении лучевых преобразований и $V$-лучевых преобразований тензорных полей высокого ранга.

Построен модифицированный алгоритм вычисления числа совершенных паросочетаний $\tau(m,n)$ в решёточном графе $m\times n$, оптимизированный по быстродействию и требуемой оперативной памяти.

Было проведено исследование фазовых переходов и термодинамических свойств двумерной часовой модели на треугольной решетке с числом состояний спина $q = 5$ с использованием алгоритма Ванга-Ландау метода Монте-Карло. На основе гистограммного метода проведен анализ рода фазовых переходов данной модели.  