% Настройки реферата
% [Количество книг]
% \TotalBooks{0}
% [Количество рисунков]
% \TotalFigures{0}
% [Количество таблиц]
% \TotalTables{0}
% [Количество использованных источников]
% \TotalBibItems{0}
% [Количество приложения]
% \TotalAppendixes{0}

% Ключевые слова
\KeyWords{полиномы Лагерра -- Соболева, полиномы Якоби, ряд Фурье, пространство Лебега с переменным показателем, рациональные сплайны, уравнение Бельтрами, уравнение Ито, метод Монте-Карло.}

% Оптимальный объем (~850 печатных знаков)

Объектом исследования являются система полиномов Лагерра -- Соболева; интерполяционные рациональные сплайн-функции;
базисность полиномов Якоби в $L^{p(\cdot)}_{w_{a,b}}$;
системы дифференциальных уравнений Ито с последействием;
задача на всей плоскости для уравнения Бельтрами;
интегральные преобразования векторных и тензорных полей;
модификация алгоритма динамического программирования по профилю;
часовая модель с числом состояний спина $q=5$ на треугольной решетке.

В ходе выполнения НИР изучена  скорость сходимости частичных сумм ряда Фурье по полиномам Лагерра -- Соболева.
Показано, что при выполнении условий Мукенхаупта система полиномов Якоби образует базис в $L^{p(\cdot)}_{w_{a,b}}$.
Также был усилен результат, касающийся базисности полиномов Лежандра в $L^{p(\cdot)}$, а именно убрано условие постоянства переменного показателя у концов отрезка $[-1,1]$.
Для обобщенного дифференциального уравнения Римана получены достаточные условия аппроксимируемости посредством дважды непрерывно
 дифференцируемых интерполяционных рациональных сплайн-функций.
Исследована глобальная моментная устойчивость систем дифференциальных уравнений Ито дробного порядка с последействием. Изучено асимптотическое поведение моментов решений систем нелинейных дифференциальных уравнений Ито с запаздываниями.
Доказана гипоэллиптичность уравнения Бельтрами с постоянным коэффициентом. Изложены свойства сглаживания функций по Стеклову в $L_p$-пространствах.
Получены новые формулы обращения интегральных преобразований векторных и тензорных полей, определенных на некоторых семействах прямых в пространстве и ломаных на плоскости.
Выявлены тесные связи вопроса совместимости профилей и чисел Фибоначчи и создано программное обеспечение для вычислительного сопровождения результатов.
Методом Монте-Карло исследована часовая модель с числом состояний спина $q=5$ на треугольной решетке. Определены структуры основного состояния и построена фазовая диаграмма.

Полученные результаты могут найти применение в задачах математической физики, цифровой обработки сигналов, теории управления и при приближенном решении дифференциальных уравнений. 