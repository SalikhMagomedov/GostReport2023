\begin{thebibliography}{111}

  % Ниже указаны примеры форматирования литературы
  
\bibitem{tad-Pollard-1} Pollard H. The mean convergence of orthogonal series // Trans. Amer. Math. Soc.
--- 1947. --- Vol. 62. --- Issue 2. --- Pp. 387---403.
\bibitem{tad-Pollard-2} Pollard H. The mean convergence of orthogonal series. II // Trans. Amer. Math. Soc.
--- 1948. --- Vol. 63. --- Issue 2. --- Pp. 355---367.
\bibitem{tad-Pollard-3} Pollard H. The mean convergence of orthogonal series. III // Duke Math. J. --- 1949.
--- Vol. 16. --- Issue 1. --- Pp. 189---191.
\bibitem{tad-Newman-Rudin} Newman J., Rudin W. Mean convergence of orthogonal series // Proc. Amer. Math.
Soc. --- 1952. --- Vol. 3. --- Issue 2. --- Pp. 219---222.
\bibitem{tad-Muckenhoupt}Muckenhoupt B. Mean convergence of Jacobi series // Proc. Amer. Math. Soc. --- 1969.
--- Vol. 23. --- Issue 2. --- Pp. 306---310.

\bibitem{tad-SHII-Leg} О базисности системы полиномов Лежандра в пространстве Лебега $L^{p(x)}(-1,1)$ с переменным показателем $p(x)$ // Мат. сборник. --- 2009. --- Том 200, № 1. --- С. 137---160.
\bibitem{tad-SHII-Jacob}Шарапудинов И.И., Шах-Эмиров Т.Н. Сходимость рядов Фурье по полиномам Якоби в весовом пространстве Лебега с переменным показателем // Дагестанские электронные математические известия. --- 2017. --- Вып. 8. --- С. 27---47.
%\bibitem{tad-MMG-Haar} Магомед-Касумов М.Г. Базисность системы Хаара в весовых пространствах Лебега с переменным показателем // Владикавк. матем. журн. --- 2014. Том 16, № 3. --- С. 38---46.
\bibitem{tad-SHII-Ult}Шарапудинов И.И. О базисности ультрасферических полиномов Якоби в весовом пространстве Лебега с переменным показателем // Мат. заметки. --- 2019. --- Том 106, № 4. --- С. 616---638.
\bibitem{tad-RAM-Jacob} Shakh-Emirov T.N., Gadzhimirzaev R.M.  The Convergence of the Fourier–Jacobi Series in Weighted Variable Exponent Lebesgue Spaces // Operator Theory and Differential Equations. Trends in Mathematics. Birkhäuser, Cham.  In: Kusraev, A.G., Totieva, Z.D. (eds). --- 2021. --- Pp. 205---227.
\bibitem{tad-lpxtopology} Шарапудинов И.И. О топологии пространства $L^{p(t)}([0,1])$ // Мат. заметки. --- 1979. --- Том 26, № 4. --- С. 613---632.

%\bibitem{tad-ShII-MonogLpx} Шарапудинов И.И. Некоторые вопросы теории приближения в пространствах Лебега с переменным показателем.  --- Владикавказ: ЮМИ ВНЦ РАН и РСО-А, 2012. --- 267 с.
 
% bab

\bibitem{bib:bab-1}
Паташинский А.З., Покровский В.А. Флуктуационная теория фазовых переходов. -- М.: Наука, 1982. 380 с.

\bibitem{bib:bab-2}
Вильсон К., Когут Д. Ренормализационная группа и $\varepsilon$--разложение. Пер. с англ. В.А. Загребного; Под ред. Alves д. В.К. Федянина. -- М.: Мир, 1975. 256 с.

\bibitem{bib:bab-3}
Паташинский А.З., Покровский В.А. УФН. - 1977. - Т. 121. - С. 55.

\bibitem{bib:bab-4}
Ма Ш. Современная теория критических явлений / Пер. с англ. А.Н. Ермилова, А.М. Курбатова; Под ред. Н.Н. Боголюбова (мл.), В.К. Федянина. -- М.: Мир, 1980. - 298 с.

\bibitem{bib:bab-5}
Kadanoff L.P. Scaling laws for Ising models near Tc // Physica. - 1966. - V. 2. - P. 263.

\bibitem{bib:bab-6}
Стенли Г. Фазовые переходы и критические явления / Пер. с англ. А.И.  Мицека, Т.С. Шубиной; Под ред. С.В. Вонсовского. -- М.: Мир, 1973. - 419 с.

\bibitem{bib:bab-7}
Фишер М. Физика критического состояния / Пер.с англ. М.Ш. Гитермана. -- М.: Мир, 1968. - 221 с.

\bibitem{bib:bab-8}
Бэкстер Р. Точно решаемые модели в статистической механике / Пер. с англ. Е.П. Вольского, Л.И. Дайхина; Под ред. А.М. Бродского. -- М.: Мир, 1985. - 486 с.

\bibitem{bib:bab-9}
Wu F.Y. Exactly Solved Models: A Journey in Statistical Mechanics. World Scientific, London, 2009. -- 661 p.

\bibitem{bib:bab-10}
Wolff U. Collective Monte Carlo Updating for spin systems // Phys. Lett. -- 1989. -- V.62. - № 4. -- P.361-364.

\bibitem{bib:bab-11}
Peczac P., Ferrenberg A.M., Landau D.P. High-accuracy Monte Carlo study of the three-dimensional classical Heisenberg ferromagnet // Phys.Rev. B -- 1991. --V.43. -- P. 6087-6093.

\bibitem{bib:bab-12}
Eichhorn K., Binder K. Monte Carlo investigation of the three-dimensional random-field three-state Potts model // J. Phys.: Condens. Matter.- 1996. - V 8. - C. 5209.

\bibitem{bib:bab-13}
Loison D., Schotte K.D. First and second order transition in frustrated XY systems // Eur. Phys. J. B. - 1998. - V. 5. - P. 735.

\bibitem{bib:bab-14}
Babaev A.B., Murtazaev A.K. The Tricritical Point of the Site-Diluted Three-Dimensional 5-State Potts Model // Journal of magnetism and magnetic materials -- 2022. -- V. 563. -- P. 169864\_1-169864\_5.

\bibitem{bib:bab-15}
Fisher M.E., Barber M.N. Scaling theory for finite-size effects in the critical region // Phys. Rev. Lett. - 1972. - V.28. -- P. 1516

\bibitem{bib:bab-16}
Loison D. Monte Carlo cluster algorithm for ferromagnetic Hamiltonians $H = J\sum(S_i, S_j)^3$ // Phys. Lett. A. - 1999. - V. 257. - P. 83.

\bibitem{bib:bab-17}
Kim J.-K., Landau D.P. Corrections to finite-size-scaling in two dimensional Potts models // Physica A. - 1998. - V. 250. - P. 362.

\bibitem{bib:bab-18}
Муртазаев А.К., Бабаев А.Б., Атаева Г.Я., Магомедов М.А. Фазовые переходы и критические явления в двумерной примесной модели Поттса с числом состояний спина q=4 на квадратной решетке // ЖЭТФ. 2022. Т.161. С. 847.


  % \bibitem{mmg-MarcellanXu2015}
  % Marcellán F., Xu Y. 
  % On Sobolev orthogonal polynomials
  % //
  % Expositiones Math.
  % --- 2015.
  % --- Vol 33.
  % --- P. 308---352.

  % \bibitem{mmg-mmg-walsh-Shii-UMN}
  % Шарапудинов И.И. 
  % Ортогональные по Соболеву системы функций и некоторые их приложения
  % //
  % УМН.
  % --- 2019.
  % % --- Т. 74, \No 4(448).
  % --- С. 87---164.

  % \bibitem{ark-bib-2}
  % Стечкин С.Б., Субботин Ю.Н.
  % Сплайны в вычислительной математике.
  % --- М.: Наука, 1976.
  % --- 248~с.

  % \bibitem{ark-bib-3}
  % Завьялов Ю.С., Квасов Б.И., Мирошниченко В.Л.
  % Методы сплайн-функций.
  % --- М.: Наука, 1980.
  % --- 352 c.

\end{thebibliography} 