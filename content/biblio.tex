\begin{thebibliography}{11}
\bibitem{Ram-Ba-Ra-Pe}
{Barry P. Rajkovi\'c P.M., Petkovi\'c M.D.}
An application of Sobolev orthogonal polynomials to the computation of a special Hankel determinant
//
In book: Approximation and Computation (Chapter 4).
--- 2011.
--- Vol. 42.
--- P. 53---60.





\bibitem{Ram-Mar-Xu}
{Marcell\'an F., Xu Y.}
On Sobolev orthogonal polynomials
//
Expo Math.
--- 2015.
--- Vol. 33.
--- P. 308---352.





\bibitem{mmg-SharapudinovUMN}
Шарапудинов И.И.
Ортогональные по Соболеву системы функций и некоторые их приложения
//
УМН
--- 74:4(448)
--- 2019.
--- С. 87---164.





\bibitem{tad-SHII-Haar}
Шарапудинов И.И.
О базисности системы Хаара в пространстве $L^{p(t)}([0,1])$ и принципе локализации в среднем
//
Мат. сборник.
--- 1986.
--- Т. 130(172), № 2(6).
--- С. 275---283.




\bibitem{tad-SHII-AnalisysMath}
Sharapudinov I.I. Some aspects of approximation theory in the spaces $L^{p(x)}(E)$
//
Analysis Mathematica.
--- 2007.
--- Vol 33.
--- P. 135---153.





\bibitem{tad-SHII-Leg}
О базисности системы полиномов Лежандра в пространстве Лебега $L^{p(x)}(-1,1)$ с переменным показателем $p(x)$ // Мат. сборник. --- 2009. --- Том 200, № 1. --- С. 137---160.





\bibitem{tad-MMG-Haar}
Магомед-Касумов М.Г.
Базисность системы Хаара в весовых пространствах Лебега с переменным показателем
//
Владикавк. матем. журн.
--- 2014.
--- Т. 16, № 3.
--- С. 38---46.





\bibitem{tad-SHII-Jacob}
Шарапудинов И.И., Шах-Эмиров Т.Н. Сходимость рядов Фурье по полиномам Якоби в весовом пространстве Лебега с переменным показателем // Дагестанские электронные математические известия. --- 2017. --- Вып. 8. --- С. 27---47.

%\bibitem{tad-MMG-Haar} Магомед-Касумов М.Г. Базисность системы Хаара в весовых пространствах Лебега с переменным показателем // Владикавк. матем. журн. --- 2014. Том 16, № 3. --- С. 38---46.





\bibitem{tad-SHII-Ult}
Шарапудинов И.И. О базисности ультрасферических полиномов Якоби в весовом пространстве Лебега с переменным показателем // Мат. заметки. --- 2019. --- Том 106, № 4. --- С. 616---638.





\bibitem{tad-RAM-Jacob}
Shakh-Emirov T.N., Gadzhimirzaev R.M.  The Convergence of the Fourier-Jacobi Series in Weighted Variable Exponent Lebesgue Spaces // Operator Theory and Differential Equations. Trends in Mathematics. Birkhäuser, Cham.  In: Kusraev, A.G., Totieva, Z.D. (eds). --- 2021. --- P. 205---227.
    




\bibitem{bib:ark-1}
Смирнов~В.И. Курс высшей математики. Т.~2.
--- М.: Наука, 1967. 





\bibitem{bib:ark-2}
Кратцер~А., Франц~Ф. Трансцендентные функции.
 --- М.: Мир, 1963.





\bibitem{bib:ark-3}
Никифоров~А.Ф., Уваров~В.Б. Специальные функции математической физики.
 --- М.: Наука, 1984.





\bibitem{bib:ark-4}
Бахвалов~Н.С. Численные методы.
--- М.: Наука, 1973. 





\bibitem{bib:ark-5}
Варга~Р. Функциональный анализ и теория аппроксимации
в численном анализе. --- М.: Мир, 1974. 





\bibitem{bib:ark-6}
Parter~S.V. Numerical methods for generalized axially symmetric potentials
 // SIAM Journal. Series B2.  --- 1965.  --- P.~500---516.





\bibitem{bib:ark-7}
Jamet~P. On the convergence of finite-difference 
approximations to one-dimensional singular boundary-value problems
 // Numer. Math.  --- 1970. --- Vol.~14. --- P.~355---378.





\bibitem{bib:ark-8}
Natterer~F. A generalized spline method for singular boundary value problems in ordinary differential equations
 //Linear Algebra Appl. --- 1973. --- Vol.~7. --- P.~189---216.





\bibitem{bib:ark-9}
Brabston~D.C., Keller~H.B. A numerical method for singular
 two point  boundary value problems
 // SIAM Journal Numer. Anal. --- 1977. --- Vol.~14. --- P.~779---791.





\bibitem{bib:ark-10}
Weinmuller~E. A difference  method for a singular
 boundary value problem of second order
 // Math. Comput. --- 1984. --- Vol.~42. --- P.~441---464. 
 




\bibitem{bib:ark-19}
Рамазанов~А.-Р.К., Магомедова~В.Г. О решении начальной задачи для 
нормальной системы с помощью рациональных сплайн-функций // Вестник Дагестанского 
государственного университета. Серия~1. Естественные науки. --- 2023. --- Том~38. Вып.~1. 
--- С.~33--39. 

% MZG





\bibitem{Medzhidov}
Меджидов З. Г. Формулы обращения тензорной томографии по неполным данным // ДЭМИ. --- 2014. Вып. 2. --- С. 75---86.





\bibitem{Truong}
Truong T. T., Nguen M. K. On V-line Radon transform in $R^2$   and thear inversion // J. Phis. A: Math. Theor. --- 2011. --- Vol. 44. No. 075206. --- P. 13.





\bibitem{Sharafutdinov}
\foreignlanguage{english}{%  
Sharafutdinov V., Venkateswaran P. Krishnan, Ramesh Manna, Suman Kumar ahoo. Momentum ray transforms // Inverse Problems and Imaging. --- 2019. --- Vol. 13. \No 3. DOI:10.3934/ipi.2019031.
}%

\bibitem{Ambartsoumian}
Gaik Ambartsoumian, Latifi-Jebelli M.J., Mishra R.K.
Generalized $V$-line transforms in 2D vector tomography
//
Inverse Problems.
--- 2020.
--- Vol. 36. \No 10.
--- 104002.
  




\bibitem{Gadzhimirzaev:DEMR2016}
{Шарапудинов~И.И., Гаджиева~З.Д., Гаджимирзаев~Р.М.} 
Системы функций, ортогональных относительно скалярных произведений типа Соболева с дискретными массами, порожденных классическими ортогональными системами 
// 
Дагестанские электронные математические известия. 
--- 2016. 
--- Вып. 6.
--- С.~31---60.





\bibitem{Gadzhimirzaev:ShII-MMG}
{Шарапудинов~И.И., Магомед-Касумов~М.Г.} 
О представлении решения задачи Коши рядом Фурье по полиномам, ортогональным по Соболеву, порождённым многочленами Лагерра 
// 
Дифференц. уравнения. 
--- 2018. 
--- Т. 54, вып. 1. 
--- С.~51---68.





\bibitem{Gadzhimirzaev:RamIzv2020}
{Гаджимирзаев~Р.М.} 
О равномерной сходимости ряда Фурье по системе полиномов, порождённой системой полиномов Лагерра 
// 
Изв. Сарат. ун-та. Нов. сер. Сер. Математика. Механика. Информатика. 
--- 2020. 
--- Т. 20, вып. 4. 
--- С.~416---423.






\bibitem{Approx-Xu}
{Xu~Y.} 
Approximation by polynomials in Sobolev spaces with Jacobi weight 
// 
J. Fourier Anal. Appl. 
--- 2018. 
--- Vol. 24. 
--- P.~1438---1459.



\bibitem{Approx-XuWang}
{Xu~Y., Wang~Z., Li~H.} 
Jacobi-Sobolev Orthogonal Polynomials and Spectral Methods for Elliptic Boundary Value Problems 
//
Commun. Appl. Math. Comput.
--- 2019.
--- Vol. 1.
---~P.~283---308.

\bibitem{Approx-Juan}
\foreignlanguage{english}{%
{Garc\'ia-Ardila~J.C., Marriaga~M.E.}
Approximation by polynomials in Sobolev spaces associated with classical moment functionals 
//
Numer. Algor.
--- 2023.
}%


\bibitem{Approx-Leonardo}
{Leonardo~E. Figueroa} 
Weighted Sobolev orthogonal polynomials and approximation in the ball 
// 
arXiv:2308.05469 [math.CA].
--- 2023.





\bibitem{Gadzhimirzaev:Szego}
			
{Сеге~Г.} 
Ортогональные многочлены. 
--- М. Физматгиз, 1962.





\bibitem{Gadzhimirzaev:AskeyWain}
{Askey~R., Wainger~S.} 
Mean convergence of expansions in Laguerre and Hermite series 
// 
Amer. J. Math.
--- 1965.
--- Vol. 87.
--- P.~698---708.





\bibitem{Gadzhimirzaev:Mocken}
{Muckenhoupt~B.} 
Mean convergence of Hermite and Laguerre series. II 
// 
Trans. Amer. Math. Soc. 
--- 1970.
--- Vol. 147, № 2.
--- P.~433---460.





\bibitem{Gadzhimirzaev:mathnot2021}
	
{Гаджимирзаев~Р.М., Шах-Эмиров~Т.Н.} 
Аппроксимативные свойства средних Валле Пуссена частичных сумм специального ряда по полиномам Лагерра 
//
Матем. заметки.
--- 2021.
--- Vol. 110, № 4.
--- C. 483---497.





\bibitem{tad-lpxtopology}
Шарапудинов И.И. О топологии пространства $L^{p(t)}([0,1])$ // Мат. заметки. --- 1979. --- Том 26, № 4. --- С. 613---632.

%\bibitem{tad-ShII-MonogLpx} Шарапудинов И.И. Некоторые вопросы теории приближения в пространствах Лебега с переменным показателем.  --- Владикавказ: ЮМИ ВНЦ РАН и РСО-А, 2012. --- 267 с.


%ARK


 




\bibitem{tad-Pollard-1}
Pollard H. The mean convergence of orthogonal series // Trans. Amer. Math. Soc.
--- 1947. --- Vol. 62. --- Issue 2. --- P. 387---403.





\bibitem{tad-Pollard-2}
Pollard H. The mean convergence of orthogonal series. II // Trans. Amer. Math. Soc.
--- 1948. --- Vol. 63. --- Issue 2. --- P. 355---367.





\bibitem{tad-Pollard-3}
Pollard H. The mean convergence of orthogonal series. III // Duke Math. J. --- 1949.
--- Vol. 16. --- Issue 1. --- P. 189---191.





\bibitem{tad-Newman-Rudin}
Newman J., Rudin W. Mean convergence of orthogonal series // Proc. Amer. Math.
Soc. --- 1952. --- Vol. 3. --- Issue 2. --- P. 219---222.





\bibitem{tad-Muckenhoupt}
Muckenhoupt B. Mean convergence of Jacobi series // Proc. Amer. Math. Soc. --- 1969.
--- Vol. 23. --- Issue 2. --- P. 306---310.





\bibitem{bib:ark-11}
Рамазанов~А.-Р.К., Магомедова~В.Г. Сплайны по трехточечным 
рациональным интерполянтам с автономными полюсами // Дагестанские электронные
 математические известия. --- 2017. --- Вып.~7. --- C.~16---28.





\bibitem{bib:ark-12}
Рамазанов~А.-Р.К., Магомедова~В.Г. Безусловно сходящиеся интерполяционные рациональные сплайны // Мат. заметки. --- 2018. Т.~103. --- Вып.~4. --- С.~592---603.





\bibitem{bib:ark-14}
Алберг~Дж., Нильсон~Э., Уолш~Дж. Теория сплайнов и ее приложения. --- М.: Мир, 1972. --- 319~c.





\bibitem{bib:ark-15}
Стечкин~С.Б., Субботин~Ю.Н. Сплайны в вычислительной математике.
 --- М.: Наука, 1976. --- 248~с.





\bibitem{bib:ark-16}
Завьялов~Ю.С., Квасов~Б.И., Мирошниченко~В.Л. Методы сплайн-функций.
 --- М.: Наука, 1980. --- 352~c.





\bibitem{bib:ark-17}
Schaback~R.  Spezielle rationale Splinefunktionen // J. Approx. Theory. --- 1973. --- Vol.~7. No.~2. --- P.~281---292.  





\bibitem{bib:ark-18}
Edeoa~A., Gofeb~G., Tefera~T. Shape preserving rational cubic spline interpolation // American Scientific Research Journal for Engineering, Technology and Sciences. --- 2015. --- Vol.~12. --- No.~1. --- P.~110---122.  
 

 
% bab





\bibitem{bib:ark-13}
Рамазанов~А.-Р.К., Магомедова~В.Г. О приближенном решении дифференциальных 
уравнений с помощью рациональных сплайн-функций // Журнал вычислительной математики и математической физики. --- 2019. --- Т.~59. \No~4. --- С.~579---586.





\bibitem{bib:mma-1}
Коршунов~С.Е. 
Фазовые переходы в двумерных системах с непрерывным вырождением 
// 
УФН. 
--- 2006.
--- Вып. 176. 
--- С.~233.





\bibitem{bib:mma-2}
Свистов~Л.Е., Прозорова~Л.А., Бюттген~Н. и др. 
Исследование магнитной структуры квазидвумерного антиферромагнетика RbFe(MoO4)2 на треугольной решетке методом ЯМР(87Rb) 
// 
Письма в ЖЭТФ 
--- 2005.
--- Вып. 81.
--- С. 133.





\bibitem{bib:mma-3}
Prewitt C.T., Shannon R.D., Rogers D.B.
Chemistry of noble metal oxides. II. Crystal structures of PtCoO2, PdCoO2, CuFeO2 and AgFeO2
//
Inorg. Chem.
--- 1971.
--- V. 10, №4.
--- P. 719---723.





\bibitem{bib:mma-4}
\foreignlanguage{english}{%
Hirakawa K., Kadowaki H., Ubukoch K.
Experimental studies of triangular lattice antiferromagnets with S = ½: NaTiO2 and LiNiO2
//
J. Phys. Soc. Japan.
--- 1985.
--- V. 54, №9.
--- P. 3526---3536.
}%

\bibitem{bib:mma-5}
Townsend~M.G., Longworth~G. and Roudaut~E.
Triangular-spin, kagome plane in jarosites 
// 
Physical Review В.
--- 1986.
--- V. 33.
--- P. 4919---4926.





\bibitem{bib:mma-6}
\foreignlanguage{english}{%
Li~J., Sleight~A.W. 
Structure of β-AgAlO2 and structural systematics of tetrahedral MM'X2
compounds 
// 
J. Solid State Chem. 
--- 2004. 
--- V. 177, №3.
--- P. 889---894.
}%

\bibitem{bib:mma-7}
Sachdev~S. 
Kagome- and triangular-lattice Heisenberg antiferromagnets: ordering from quantum fluctuations and quantum-disordered ground states with unconfined bosonic spinons 
//
Phys. Rev. B.
--- 1992.
--- V. 45.
--- P. 12377---12396.





\bibitem{bib:mma-8}
Balents~L. 
Spin liquids in frustrated magnets
// 
Nature
--- 2010.
--- V. 464.
--- P. 199---208.





\bibitem{bib:mma-9}
Landau D.P., Wang F., Tsai S.-H.
Critical endpoint behavior: A Wang-Landau study
//
Comp. Phys. Comm.
--- 2008.
--- V. 179.
--- P. 8.





\bibitem{bib:mma-10}
Chiaki Y., Yutaka O.
Three-dimensional antiferromagnetic q -state Potts models: application of the Wang-Landau algorithm
//
Journal of Physics A: Mathematical and General.
--- 2001.
--- V. 34.
--- 8781.





\bibitem{bib:mma-11}
Zhou~C., Bhatt~R.N. 
Understanding and improving the Wang-Landau algorithm 
//
Physical Review E.
--- 2005.
--- V. 72(2).
--- P. 025701.




\bibitem{bib:bab-8}
Бэкстер~Р. 
Точно решаемые модели в статистической механике / Пер. с англ. Е.П.~Вольского, Л.И.~Дайхина; Под ред. А.М.~Бродского. 
--- М.: Мир, 1985. 
--- 486~с.





\bibitem{bib:bab-9}
Wu~F.Y. 
Exactly Solved Models: A Journey in Statistical Mechanics. 
--- London: World Scientific, 2009. 
--- 661~p.





\bibitem{bib:bab-11}
\foreignlanguage{english}{%
Peczac~P., Ferrenberg~A.M., Landau~D.P. 
High\-/accuracy Monte Carlo study of the three\-/dimensional classical Heisenberg ferromagnet // Phys. Rev. B 
--- 1991. 
--- V. 43. 
---~P.~6087---6093.
}%

\bibitem{bib:bab-12}
Eichhorn~K., Binder~K. 
Monte Carlo investigation of the three-dimensional random-field three-state Potts model 
// 
J. Phys.: Condens. Matter.
--- 1996. 
--- V 8. 
--- P.~5209.





\bibitem{bib:bab-13}
Loison~D., Schotte~K.D. 
First and second order transition in frustrated XY systems 
// 
Eur. Phys. J. B. 
--- 1998. 
--- V. 5. 
--- P.~735.





\bibitem{bib:bab-14}
Babaev~A.B., Murtazaev~A.K. 
The Tricritical Point of the Site-Diluted Three-Dimensional 5-State Potts Model // Journal of magnetism and magnetic materials 
--- 2022. 
--- V. 563. 
--- P.~169864\_1---169864\_5.





\bibitem{bib:bab-15}
Fisher~M.E., Barber~M.N. 
Scaling theory for finite-size effects in the critical region 
// 
Phys. Rev. Lett. 
--- 1972. 
--- V. 28. 
--- P.~1516.





\bibitem{bib:bab-16}
Loison~D. 
Monte Carlo cluster algorithm for ferromagnetic Hamiltonians $H = J\sum(S_i, S_j)^3$ 
// 
Phys. Lett. A. 
--- 1999. 
--- V. 257. 
--- P.~83.





\bibitem{bib:bab-17}
Kim~J.-K., Landau~D.P. 
Corrections to finite-size-scaling in two dimensional Potts models 
// 
Physica A. 
--- 1998. 
--- V. 250. 
--- P.~362.


\end{thebibliography} 