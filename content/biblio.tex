\begin{thebibliography}{111}

  % Ниже указаны примеры форматирования литературы
  
\bibitem{tad-Pollard-1} Pollard H. The mean convergence of orthogonal series // Trans. Amer. Math. Soc.
--- 1947. --- Vol. 62. --- Issue 2. --- Pp. 387---403.
\bibitem{tad-Pollard-2} Pollard H. The mean convergence of orthogonal series. II // Trans. Amer. Math. Soc.
--- 1948. --- Vol. 63. --- Issue 2. --- Pp. 355---367.
\bibitem{tad-Pollard-3} Pollard H. The mean convergence of orthogonal series. III // Duke Math. J. --- 1949.
--- Vol. 16. --- Issue 1. --- Pp. 189---191.
\bibitem{tad-Newman-Rudin} Newman J., Rudin W. Mean convergence of orthogonal series // Proc. Amer. Math.
Soc. --- 1952. --- Vol. 3. --- Issue 2. --- Pp. 219---222.
\bibitem{tad-Muckenhoupt}Muckenhoupt B. Mean convergence of Jacobi series // Proc. Amer. Math. Soc. --- 1969.
--- Vol. 23. --- Issue 2. --- Pp. 306---310.

\bibitem{tad-SHII-Leg} О базисности системы полиномов Лежандра в пространстве Лебега $L^{p(x)}(-1,1)$ с переменным показателем $p(x)$ // Мат. сборник. --- 2009. --- Том 200, № 1. --- С. 137---160.
\bibitem{tad-SHII-Jacob}Шарапудинов И.И., Шах-Эмиров Т.Н. Сходимость рядов Фурье по полиномам Якоби в весовом пространстве Лебега с переменным показателем // Дагестанские электронные математические известия. --- 2017. --- Вып. 8. --- С. 27---47.
%\bibitem{tad-MMG-Haar} Магомед-Касумов М.Г. Базисность системы Хаара в весовых пространствах Лебега с переменным показателем // Владикавк. матем. журн. --- 2014. Том 16, № 3. --- С. 38---46.
\bibitem{tad-SHII-Ult}Шарапудинов И.И. О базисности ультрасферических полиномов Якоби в весовом пространстве Лебега с переменным показателем // Мат. заметки. --- 2019. --- Том 106, № 4. --- С. 616---638.
\bibitem{tad-RAM-Jacob} Shakh-Emirov T.N., Gadzhimirzaev R.M.  The Convergence of the Fourier–Jacobi Series in Weighted Variable Exponent Lebesgue Spaces // Operator Theory and Differential Equations. Trends in Mathematics. Birkhäuser, Cham.  In: Kusraev, A.G., Totieva, Z.D. (eds). --- 2021. --- Pp. 205---227.
\bibitem{tad-lpxtopology} Шарапудинов И.И. О топологии пространства $L^{p(t)}([0,1])$ // Мат. заметки. --- 1979. --- Том 26, № 4. --- С. 613---632.

%\bibitem{tad-ShII-MonogLpx} Шарапудинов И.И. Некоторые вопросы теории приближения в пространствах Лебега с переменным показателем.  --- Владикавказ: ЮМИ ВНЦ РАН и РСО-А, 2012. --- 267 с.
 

  \bibitem{mmg-MarcellanXu2015}
  Marcellán F., Xu Y. 
  On Sobolev orthogonal polynomials
  //
  Expositiones Math.
  --- 2015.
  --- Vol 33.
  --- P. 308---352.

  \bibitem{mmg-mmg-walsh-Shii-UMN}
  Шарапудинов И.И. 
  Ортогональные по Соболеву системы функций и некоторые их приложения
  //
  УМН.
  --- 2019.
  % --- Т. 74, \No 4(448).
  --- С. 87---164.

  \bibitem{ark-bib-2}
  Стечкин С.Б., Субботин Ю.Н.
  Сплайны в вычислительной математике.
  --- М.: Наука, 1976.
  --- 248~с.

  \bibitem{ark-bib-3}
  Завьялов Ю.С., Квасов Б.И., Мирошниченко В.Л.
  Методы сплайн-функций.
  --- М.: Наука, 1980.
  --- 352 c.

\end{thebibliography} 