\begin{thebibliography}{111}

  % Ниже указаны примеры форматирования литературы
\bibitem{Ram-Ba-Ra-Pe}
{Barry P. Rajkovi\'c P.M., Petkovi\'c M.D.}
An application of Sobolev orthogonal polynomials to the computation of a special Hankel determinant
//
In book: Approximation and Computation (Chapter 4).
--- 2011.
--- Vol. 42.
--- P. 53---60.

\bibitem{Ram-Mar-Xu}
{Marcell\'an F., Xu Y.}
On Sobolev orthogonal polynomials
//
Expo Math.
--- 2015.
--- Vol. 33.
--- P. 308---352.

\bibitem{mmg-SharapudinovUMN}
Шарапудинов И.И.
Ортогональные по Соболеву системы функций и некоторые их приложения
//
УМН
--- 74:4(448)
--- 2019.
--- С. 87---164.

\bibitem{tad-SHII-Haar}
Шарапудинов И.И.
О базисности системы Хаара в пространстве $L^{p(t)}([0,1])$ и принципе локализации в среднем
//
Мат. сборник.
--- 1986.
--- Т. 130(172), № 2(6).
--- С. 275---283.
\bibitem{tad-SHII-AnalisysMath}
Sharapudinov I.I. Some aspects of approximation theory in the spaces $L^{p(x)}(E)$
//
Analysis Mathematica.
--- 2007.
--- Vol 33.
--- P. 135---153.
\bibitem{tad-SHII-Leg} О базисности системы полиномов Лежандра в пространстве Лебега $L^{p(x)}(-1,1)$ с переменным показателем $p(x)$ // Мат. сборник. --- 2009. --- Том 200, № 1. --- С. 137---160.
\bibitem{tad-MMG-Haar}
Магомед-Касумов М.Г.
Базисность системы Хаара в весовых пространствах Лебега с переменным показателем
//
Владикавк. матем. журн.
--- 2014.
--- Т. 16, № 3.
--- С. 38---46.
\bibitem{tad-SHII-Jacob}Шарапудинов И.И., Шах-Эмиров Т.Н. Сходимость рядов Фурье по полиномам Якоби в весовом пространстве Лебега с переменным показателем // Дагестанские электронные математические известия. --- 2017. --- Вып. 8. --- С. 27---47.
%\bibitem{tad-MMG-Haar} Магомед-Касумов М.Г. Базисность системы Хаара в весовых пространствах Лебега с переменным показателем // Владикавк. матем. журн. --- 2014. Том 16, № 3. --- С. 38---46.
\bibitem{tad-SHII-Ult}Шарапудинов И.И. О базисности ультрасферических полиномов Якоби в весовом пространстве Лебега с переменным показателем // Мат. заметки. --- 2019. --- Том 106, № 4. --- С. 616---638.
\bibitem{tad-RAM-Jacob} Shakh-Emirov T.N., Gadzhimirzaev R.M.  The Convergence of the Fourier–Jacobi Series in Weighted Variable Exponent Lebesgue Spaces // Operator Theory and Differential Equations. Trends in Mathematics. Birkhäuser, Cham.  In: Kusraev, A.G., Totieva, Z.D. (eds). --- 2021. --- Pp. 205---227.
    
\bibitem{bib:ark-1} Смирнов~В.И. Курс высшей математики. Т.~2.
--- М.: Наука, 1967. 

\bibitem{bib:ark-2} Кратцер~А., Франц~Ф. Трансцендентные функции.
 --- М.: Мир, 1963.

\bibitem{bib:ark-3} Никифоров~А.Ф., Уваров~В.Б. Специальные функции математической физики.
 --- М.: Наука, 1984.

\bibitem{bib:ark-4}  Бахвалов~Н.С. Численные методы.
--- М.: Наука, 1973. 

\bibitem{bib:ark-5}  Варга~Р. Функциональный анализ и теория аппроксимации
в численном анализе. --- М.: Мир, 1974. 

\bibitem{bib:ark-6} Parter~S.V. Numerical methods for generalized axially symmetric potentials
 // SIAM Journal. Series B2.  --- 1965.  --- P.~500---516.

\bibitem{bib:ark-7}  Jamet~P. On the convergence of finite-difference 
approximations to one-dimensional singular boundary-value problems
 // Numer. Math.  --- 1970. --- Vol.~14. --- P.~355---378.

\bibitem{bib:ark-8}  Natterer~F. A generalized spline method for singular boundary value problems in ordinary differential equations
 //Linear Algebra Appl. --- 1973. --- Vol.~7. --- P.~189---216.

\bibitem{bib:ark-9} Brabston~D.C., Keller~H.B. A numerical method for singular
 two point  boundary value problems
 // SIAM Journal Numer. Anal. --- 1977. --- Vol.~14. --- P.~779---791.

\bibitem{bib:ark-10}  Weinmuller~E. A difference  method for a singular
 boundary value problem of second order
 // Math. Comput. --- 1984. --- Vol.~42. --- P.~441---464. 
 
\bibitem{bib:ark-19} Рамазанов~А.-Р.К., Магомедова~В.Г. О решении начальной задачи для 
нормальной системы с помощью рациональных сплайн-функций // Вестник Дагестанского 
государственного университета. Серия~1. Естественные науки. --- 2023. --- Том~38. Вып.~1. 
--- С.~33-–39. 

% MZG

\bibitem{Medzhidov}  Меджидов З. Г. Формулы обращения тензорной томографии по неполным данным // ДЭМИ. --- 2014. Вып. 2. --- С. 75---86.
\bibitem{Truong} Truong T. T., Nguen M. K. On V-line Radon transform in $R^2$   and thear inversion // J. Phis. A: Math. Theor. --- 2011. --- Vol. 44. No. 075206. --- P. 13.
\bibitem{Sharafutdinov} Sharafutdinov V., Venkateswaran P. Krishnan, Ramesh Manna, Suman Kumar ahoo. Momentum ray transforms. 2019. Inverse Problems and Imaging 13(3) DOI:10.3934/ipi.2019031
\bibitem{Ambartsoumian} Gaik Ambartsoumian, M. J. Latifi-Jebelli, and R. K. Mishra. Generalized V-line transforms in 2D vector tomography, Inverse Problems 36 (2020), no. 10, 104002.
  
\bibitem{tad-Pollard-1} Pollard H. The mean convergence of orthogonal series // Trans. Amer. Math. Soc.
--- 1947. --- Vol. 62. --- Issue 2. --- P. 387---403.
\bibitem{tad-Pollard-2} Pollard H. The mean convergence of orthogonal series. II // Trans. Amer. Math. Soc.
--- 1948. --- Vol. 63. --- Issue 2. --- P. 355---367.
\bibitem{tad-Pollard-3} Pollard H. The mean convergence of orthogonal series. III // Duke Math. J. --- 1949.
--- Vol. 16. --- Issue 1. --- P. 189---191.
\bibitem{tad-Newman-Rudin} Newman J., Rudin W. Mean convergence of orthogonal series // Proc. Amer. Math.
Soc. --- 1952. --- Vol. 3. --- Issue 2. --- P. 219---222.
\bibitem{tad-Muckenhoupt}Muckenhoupt B. Mean convergence of Jacobi series // Proc. Amer. Math. Soc. --- 1969.
--- Vol. 23. --- Issue 2. --- P. 306---310.

\bibitem{tad-lpxtopology} Шарапудинов И.И. О топологии пространства $L^{p(t)}([0,1])$ // Мат. заметки. --- 1979. --- Том 26, № 4. --- С. 613---632.

%\bibitem{tad-ShII-MonogLpx} Шарапудинов И.И. Некоторые вопросы теории приближения в пространствах Лебега с переменным показателем.  --- Владикавказ: ЮМИ ВНЦ РАН и РСО-А, 2012. --- 267 с.


%ARK


 
\bibitem{bib:ark-11} Рамазанов~А.-Р.К., Магомедова~В.Г. Сплайны по трехточечным 
рациональным интерполянтам с автономными полюсами // Дагестанские электронные
 математические известия. --- 2017. --- Вып.~7. --- C.~16---28.

\bibitem{bib:ark-12} Рамазанов~А.-Р.К., Магомедова~В.Г. Безусловно сходящиеся интерполяционные рациональные сплайны // Мат. заметки. --- 2018. Т.~103. --- Вып.~4. --- С.~592---603.

\bibitem{bib:ark-13} Рамазанов~А.-Р.К., Магомедова~В.Г. О приближенном решении дифференциальных 
уравнений с помощью рациональных сплайн-функций // Журнал вычислительной математики и математической физики. --- 2019. --- Т.~59. \No~4. --- С.~579---586.

\bibitem{bib:ark-14}  Алберг~Дж., Нильсон~Э., Уолш~Дж. Теория сплайнов и ее приложения. --- М.: Мир, 1972. --- 319~c.

\bibitem{bib:ark-15}  Стечкин~С.Б., Субботин~Ю.Н. Сплайны в вычислительной математике.
 --- М.: Наука, 1976. --- 248~с.

\bibitem{bib:ark-16}  Завьялов~Ю.С., Квасов~Б.И., Мирошниченко~В.Л. Методы сплайн-функций.
 --- М.: Наука, 1980. --- 352~c.

\bibitem{bib:ark-17}  Schaback~R.  Spezielle rationale Splinefunktionen // J. Approx. Theory. --- 1973. --- Vol.~7. No.~2. --- P.~281---292.  

\bibitem{bib:ark-18} Edeoa~A., Gofeb~G., Tefera~T. Shape preserving rational cubic spline interpolation // American Scientific Research Journal for Engineering, Technology and Sciences. --- 2015. --- Vol.~12. --- No.~1. --- P.~110---122.  
 

 
% bab

\bibitem{bib:bab-1}
Паташинский А.З., Покровский В.А. Флуктуационная теория фазовых переходов. -- М.: Наука, 1982. 380 с.

\bibitem{bib:bab-2}
Вильсон К., Когут Д. Ренормализационная группа и $\varepsilon$--разложение. Пер. с англ. В.А. Загребного; Под ред. Alves д. В.К. Федянина. --- М.: Мир, 1975. 256 с.

\bibitem{bib:bab-3}
Паташинский А.З., Покровский В.А. УФН. --- 1977. --- Т. 121. --- С. 55.

\bibitem{bib:bab-4}
Ма Ш. Современная теория критических явлений / Пер. с англ. А.Н. Ермилова, А.М. Курбатова; Под ред. Н.Н. Боголюбова (мл.), В.К. Федянина. -- М.: Мир, 1980. - 298 с.

\bibitem{bib:bab-5}
Kadanoff L.P. Scaling laws for Ising models near Tc // Physica. - 1966. - V. 2. - P. 263.

\bibitem{bib:bab-6}
Стенли Г. Фазовые переходы и критические явления / Пер. с англ. А.И.  Мицека, Т.С. Шубиной; Под ред. С.В. Вонсовского. -- М.: Мир, 1973. - 419 с.

\bibitem{bib:bab-7}
Фишер М. Физика критического состояния / Пер.с англ. М.Ш. Гитермана. -- М.: Мир, 1968. - 221 с.

\bibitem{bib:bab-8}
Бэкстер Р. Точно решаемые модели в статистической механике / Пер. с англ. Е.П. Вольского, Л.И. Дайхина; Под ред. А.М. Бродского. -- М.: Мир, 1985. - 486 с.

\bibitem{bib:bab-9}
Wu F.Y. Exactly Solved Models: A Journey in Statistical Mechanics. World Scientific, London, 2009. -- 661 p.

\bibitem{bib:bab-10}
Wolff U. Collective Monte Carlo Updating for spin systems // Phys. Lett. -- 1989. -- V.62. - № 4. -- P.361-364.

\bibitem{bib:bab-11}
Peczac P., Ferrenberg A.M., Landau D.P. High-accuracy Monte Carlo study of the three-dimensional classical Heisenberg ferromagnet // Phys.Rev. B -- 1991. --V.43. -- P. 6087-6093.

\bibitem{bib:bab-12}
Eichhorn K., Binder K. Monte Carlo investigation of the three-dimensional random-field three-state Potts model // J. Phys.: Condens. Matter.- 1996. - V 8. - C. 5209.

\bibitem{bib:bab-13}
Loison D., Schotte K.D. First and second order transition in frustrated XY systems // Eur. Phys. J. B. - 1998. - V. 5. - P. 735.

\bibitem{bib:bab-14}
Babaev A.B., Murtazaev A.K. The Tricritical Point of the Site-Diluted Three-Dimensional 5-State Potts Model // Journal of magnetism and magnetic materials -- 2022. -- V. 563. -- P. 169864\_1-169864\_5.

\bibitem{bib:bab-15}
Fisher M.E., Barber M.N. Scaling theory for finite-size effects in the critical region // Phys. Rev. Lett. - 1972. - V.28. -- P. 1516

\bibitem{bib:bab-16}
Loison D. Monte Carlo cluster algorithm for ferromagnetic Hamiltonians $H = J\sum(S_i, S_j)^3$ // Phys. Lett. A. - 1999. - V. 257. - P. 83.

\bibitem{bib:bab-17}
Kim J.-K., Landau D.P. Corrections to finite-size-scaling in two dimensional Potts models // Physica A. - 1998. - V. 250. - P. 362.

\bibitem{bib:bab-18}
Муртазаев А.К., Бабаев А.Б., Атаева Г.Я., Магомедов М.А. Фазовые переходы и критические явления в двумерной примесной модели Поттса с числом состояний спина q=4 на квадратной решетке // ЖЭТФ. 2022. Т.161. С. 847.

% grm

\bibitem{Gadzhimirzaev:DEMR2016}
{Шарапудинов И.И., Гаджиева З.Д., Гаджимирзаев Р.М.} Системы функций, ортогональных относительно скалярных произведений типа Соболева с дискретными массами, порожденных классическими ортогональными системами // Дагестанские электронные математические известия. 2016. Вып. 6. С. 31--60.

\bibitem{Gadzhimirzaev:ShII-MMG}
{Шарапудинов И.И., Магомед-Касумов М.Г.} О представлении решения задачи Коши рядом Фурье по полиномам, ортогональным по Соболеву, порождённым многочленами Лагерра // Дифференц. уравнения. 2018. Т. 54. Вып. 1. С. 51--68.

\bibitem{Gadzhimirzaev:RamIzv2020}
{Гаджимирзаев Р.М.} О равномерной сходимости ряда Фурье по системе полиномов, порождённой системой полиномов Лагерра // Изв. Сарат. ун-та. Нов. сер. Сер. Математика. Механика. Информатика. 2020. Т. 20. Вып. 4. С. 416--423.


\bibitem{Approx-Xu}
{Xu Y.} Approximation by polynomials in Sobolev spaces with Jacobi weight // J. Fourier Anal. Appl. 2018. Vol. 24. Pp. 1438–-1459.


\bibitem{Approx-XuWang}
{Xu Y., Wang Z., Li H.} Jacobi–Sobolev Orthogonal Polynomials and Spectral Methods for Elliptic Boundary Value Problems //
Commun. Appl. Math. Comput. 2019. Vol. 1. Pp. 283–-308.

\bibitem{Approx-Juan}
{Garc\'ia-Ardila, J.C., Marriaga, M.E.} Approximation by polynomials in Sobolev spaces associated with classical moment functionals //
Numer. Algor. 2023.

\bibitem{Approx-Leonardo}
{Leonardo E. Figueroa} Weighted Sobolev orthogonal polynomials and approximation in the ball // arXiv:2308.05469 [math.CA]. 2023.

\bibitem{Gadzhimirzaev:Szego}			
{Сеге Г.} Ортогональные многочлены. М. Физматгиз. 1962

\bibitem{Gadzhimirzaev:AskeyWain}
{Askey R., Wainger S.} Mean convergence of expansions in Laguerre and Hermite series // Amer. J. Math. 1965. Vol. 87. Pp. 698--708.

\bibitem{Gadzhimirzaev:Mocken}
{Muckenhoupt B.} Mean convergence of Hermite and Laguerre series. II // Trans. Amer. Math. Soc. 1970. Vol. 147. № 2. Pp. 433--460.

\bibitem{Gadzhimirzaev:mathnot2021}	
{Гаджимирзаев Р.М., Шах-Эмиров Т.Н.} Аппроксимативные свойства средних Валле Пуссена частичных сумм специального ряда по полиномам Лагерра //
Матем. заметки. 2021. Vol. 110. № 4. C. 483--497.												

\bibitem{Sal-Ram1}
{Магомедов С.Р., Гаджимирзаев Р.М.} Свидетельство № 2023682765 государственной регистрации программы для ЭВМ «Программа вычисления полиномов Лагерра -- Соболева с помощью рекуррентных формул». Заявка № 2023668290, дата поступления 05 сентября 2023 г., дата государственной регистрации в Реестре программ для ЭВМ 31 октября 2023 г.
\bibitem{Sal-Ram2}
{Гаджимирзаев Р.М., Магомедов С.Р.} Свидетельство № 2023681818 государственной регистрации программы для ЭВМ «Программа рекуррентного вычисления на сетке значений функций Лагерра -– Соболева». Заявка № 2023680782, дата поступления 11 октября 2023 г., дата государственной регистрации в Реестре программ для ЭВМ 18 октября 2023 г.

% rmk

\bibitem{bib:rmk-1}
Л.Е. Свистов, Л.А. Прозорова, Н. Бюттген и др., Письма в ЖЭТФ 81, 133 (2005).

\bibitem{bib:rmk-2}
С.Е. Коршунов. УФН 176, 233 (2006).

\bibitem{bib:rmk-3}
M. Tisser, B. Delamotte, D. Mouhanna. Phys. Rev. Lett. 84, 5208 (2000).

\bibitem{bib:rmk-4}
M. Nauenberg, D.J. Scalapino, Phys. Rev. Lett. 44, 837 (1980).

\bibitem{bib:rmk-5}
A.K. Murtazaev, M.K. Badiev, M.K. Ramazanov et al., Phase transitions 94, 394 (2021).

\bibitem{bib:rmk-6}
F.Y. Wu, Rev. Mod. Phys. 54, 235,(1982); errata, ibid. 55, 315 (1983).



% MMA

\bibitem{bib:mma-1}
Коршунов С.Е. Фазовые переходы в двумерных системах с непрерывным вырождением // УФН 176, 233 (2006).

\bibitem{bib:mma-2}
Свистов Л.Е., Прозорова Л.А., Бюттген Н. и др. Исследование магнитной структуры квазидвумерного антиферромагнетика RbFe(MoO4)2 на треугольной решетке методом ЯМР(87Rb) // Письма в ЖЭТФ 81, 133 (2005).

\bibitem{bib:mma-3}
Prewitt C.T., Shannon R.D., Rogers D.B. Chemistry of noble metal oxides. II. Crystal structures of PtCoO2, PdCoO2, CuFeO2 and AgFeO2. // Inorg. Chem. 1971. V. 10, №4. P. 719--723.

\bibitem{bib:mma-4}
Hirakawa K., Kadowaki H., Ubukoch K. Experimental studies of triangular lattice antiferromagnets with S = ½: NaTiO2 and LiNiO2. // J. Phys. Soc. Japan. 1985. V. 54, №9. P. 3526-3536.

\bibitem{bib:mma-5}
Townsend M.G., Longworth G. and Roudaut E. Triangular-spin, kagome plane in jarosites // Physical Review В, 1986. V. 33: p. 4919-4926.

\bibitem{bib:mma-6}
Li J., Sleight A.W. Structure of β-AgAlO2and structural systematics of tetrahedral MM’X2compounds. // J. Solid State Chem. 2004.  V. 177, №3. P. 889--894.

\bibitem{bib:mma-7}
Sachdev, S. Kagome- and triangular-lattice Heisenberg antiferromagnets: ordering from quantum fluctuations and quantum-disordered ground states with unconfined bosonic spinons. Phys. Rev. B 45, 12377--12396 (1992).

\bibitem{bib:mma-8}
Balents L. Spin liquids in frustrated magnets. Nature 464, 199--208 (2010)

\bibitem{bib:mma-9}
Landau D.P., Wang F., Tsai S.-H. Critical endpoint behavior: A Wang-Landau study // Comp. Phys. Comm. 2008. V. 179. P. 8.

\bibitem{bib:mma-10}
Chiaki Y., Yutaka O. Three-dimensional antiferromagnetic q -state Potts models: application of the Wang-Landau algorithm // Journal of Physics A: Mathematical and General. 2001. V. 34. 8781.

\bibitem{bib:mma-11}
Zhou C.Bhatt R.N. Understanding and improving the Wang-Landau algorithm // Physical Review E. 2005. V. 72(2): p. 025701.

  % \bibitem{mmg-MarcellanXu2015}
  % Marcellán F., Xu Y. 
  % On Sobolev orthogonal polynomials
  % //
  % Expositiones Math.
  % --- 2015.
  % --- Vol 33.
  % --- P. 308---352.

  % \bibitem{mmg-mmg-walsh-Shii-UMN}
  % Шарапудинов И.И. 
  % Ортогональные по Соболеву системы функций и некоторые их приложения
  % //
  % УМН.
  % --- 2019.
  % % --- Т. 74, \No 4(448).
  % --- С. 87---164.

  % \bibitem{ark-bib-2}
  % Стечкин С.Б., Субботин Ю.Н.
  % Сплайны в вычислительной математике.
  % --- М.: Наука, 1976.
  % --- 248~с.

  % \bibitem{ark-bib-3}
  % Завьялов Ю.С., Квасов Б.И., Мирошниченко В.Л.
  % Методы сплайн-функций.
  % --- М.: Наука, 1980.
  % --- 352 c.

\end{thebibliography} 