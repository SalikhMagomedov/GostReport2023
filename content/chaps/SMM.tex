\chapter{О некоторых задачах на всей плоскости для уравнения Бельтрами}


\section{Введение}
%\begin{abstract}
Для усреднения эллиптических уравнений достаточно иметь однозначно разрешимую задачу во всем пространстве и соответствующие априорные оценки. Это позволяет изучить различные  аспекты усреднения не только во всем пространстве, но и в ограниченных областях. Однозначно разрешимую задачу в $\mathbb{R}^n$ и соответствующие априорные оценки имеют дивергентные эллиптические уравнения. В случае недивергентных уравнений задачи во всем пространстве мало изучены. 

В данной работе рассматривается задача на всей плоскости для уравнения Бельтрами. Получены априорные оценки, доказана гипоэллиптичность уравнения Бельтрами с постоянным коэффициентом. Изложены свойства сглаживания функций по Стеклову в $L_p$\-/пространствах. Полученные результаты в дальнейшем будут использованы нами в вопросах усреднения уравнения Бельтрами на всей плоскости.
%\end{abstract} 



\section{Задача на всей плоскости для уравнения Бельтрами}
	
Рассмотрим на плоскости $\mathbb{R}^2$ задачу
\begin{equation}\label{eq:smm-1}
  Au\equiv\partial_{\bar{z}}u+\mu\partial_z u=f\in L_q (\mathbb{R}^2),\quad u\in B_q^1 (\mathbb{R}^2),
\end{equation}
                                
	где 
	$$
	\partial_{\bar{z}}=2^{-1}\left(\frac{\partial}{\partial x_1}+i\frac{\partial}{\partial x_2}\right),\quad \partial_{z}=2^{-1}\left(\frac{\partial}{\partial x_1}-i\frac{\partial}{\partial x_2}\right),
	$$
	 $\mu$ - измеримая комплекснозначная ограниченная на плоскости $\mathbb{R}^2$ функция, удовлетворяющая условию
\begin{equation}\label{eq:smm-2}
\mathop{\rm vrai\ sup}_{x\in \mathbb{R}^2} |\mu(x)|\leqslant k_0<1,
\end{equation}      
	$k_0>0$ --- постоянная эллиптичности.
	$B_q^1 (\mathbb{R}^2)$, --- банахово пространство комплекснозначных функций, имеющих обобщенные производные первого порядка из $L_q (\mathbb{R}^2)$, $q>2$, и удовлетворяющих условию Гельдера с показателем $\alpha=1-2/q$  на всей плоскости $\mathbb{R}^2$. Кроме того функции из $B_q^1 (\mathbb{R}^2)$ равны нулю в произвольной фиксированной точке $z_0\in \mathbb{R}^2$. Норму в пространстве $B_q^1 (\mathbb{R}^2)$ задает равенство
	$$\|u\|_{B_q^1 (\mathbb{R}^2)}=H_{\mathbb{R}^2}^\alpha [u]+\|\partial_{\bar{z} }u\|_{L_q (\mathbb{R}^2)}+\|\partial_z u\|_{L_q (\mathbb{R}^2)},$$
	где
	$$H_{{\mathbb{R}^2}}^\alpha [u]=\mathop{\rm sup}\limits_{\substack {z^\prime,\, z^{\prime\prime}\in\mathbb{R}^2\\z^\prime\ne z^{\prime\prime}}}\frac{|u(z^\prime
		)-u(z^{\prime\prime})|}{|z^\prime-z^{\prime\prime}|}.$$
	
	\begin{theorem}\label{th:smm-1}
		Найдется показатель повышенной суммируемости $q>2$ такой, что задача \eqref{eq:smm-1} имеет единственное решение для любой правой части $f\in L_q(\mathbb{R}^2)$, причем имеет место априорная оценка 
	\begin{equation}\label{eq:smm-3}
	H_{\mathbb{R}^2}^\alpha [u]+\|\partial_{\bar{z} }u\|_{L_q (\mathbb{R}^2)}+\|\partial_z u\|_{L_q (\mathbb{R}^2)}\leqslant c\|Au\|_{L_q(\mathbb{R}^2)}, \quad u\in B_q^1({\mathbb R}^2),
	\end{equation}
		где $c >0$ --- постоянная, зависящая только от $k_0$ и $q$. 
	\end{theorem}
	
	\textit{Отметим некоторые следствия теоремы~\ref{th:smm-1}.} Для любых  $z^\prime$, $z^{\prime\prime}\in {\mathbb R}^2$, ввиду \eqref{eq:smm-3}, имеем
	$$|u(z^\prime)-u(z^{\prime\prime})| \leqslant c\|Au\|_{L_q ({\mathbb R}^2) }\leqslant |z^\prime-z^{\prime\prime}|^{1-2/q}.$$
	Отсюда, так как $u(z_0 )=0$, вытекает
	\begin{equation}\label{eq:smm-4}
|u(z)|\leqslant c\|Au\|_{L_q ({\mathbb R}^2 )} |z-z_0 |^{1-2/q},\quad z\in {\mathbb R}^2.
\end{equation}                           
	Пусть $K$ --- ограниченная подобласть плоскости с кусочно гладкой границей. Тогда проинтегрировав \eqref{eq:smm-4} по области K, получим
\begin{equation*}%\label{eq:smm-5}
\|u\|_{L_q (K) }\leqslant c_1 \|Au\|_{L_q ({\mathbb R}^2 )}|z-z_0|^{1-2/q}, 
\end{equation*}                                      
	где $c_1>0$ --- постоянная, зависящая только от $k_0$, $q$ и $K$.

\section{Гипоэллиптичность уравнения Бельтрами \texorpdfstring{\eqref{eq:smm-1}}{} с постоянным коэффициентом}
Рассмотрим следующую задачу с постоянным коэффициентом  $\mu$, $|\mu|\leqslant k_0<1$,
\begin{equation}\label{eq:smm-6}
\begin{split}
	&Au\equiv\partial_{\bar{z}}u+\mu\partial_z u=f\in W^1_q (\mathbb{R}^2),\\
	&u\in B_q^2(\mathbb{R}^2) =\{u,\,\partial_{\bar{z}}u,\,\partial_z u\in 
	B_q^1(\mathbb{R}^2)\}
\end{split} 
\end{equation}

$B_q^2(\mathbb{R}^2)$ --- банахово пространство с нормой
$$
\|u\|_{B_q^2(\mathbb{R}^2)}=\|u\|_{B_q^1(\mathbb{R}^2)}+\|\partial_{\bar{z}}u\|_{B_q^1(\mathbb{R}^2)}+\|\partial_{z}u\|_{B_q^1(\mathbb{R}^2)}
$$

\begin{theorem}\label{th:smm-2}
	Задача \eqref{eq:smm-6} однозначно разрешима для любой правой части $f$ из пространства $W_q^1 (\mathbb{R}^2)$, причем имеют место соотношения
	\begin{gather*}
		\|\partial_{\bar{z}}u\|_{L_\infty(\mathbb{R}^2)}+\|\partial_{z}u\|_{L_\infty(\mathbb{R}^2)}
		\leqslant c\|f\|_{W^1_q (\mathbb{R}^2)}, %\eqno{(7)}
		\\
		\|\partial^2_{\bar{z}\bar{z}}u\|_{L_q(\mathbb{R}^2)}+\|\partial^2_{z\bar{z}}u\|_{L_q(\mathbb{R}^2)}+\|\partial^2_{zz}u\|_{L_q(\mathbb{R}^2)}
		\leqslant c\|f\|_{W^1_q (\mathbb{R}^2)}, %\eqno{(8)}
	\end{gather*}
	где $c >0$ --- постоянная, зависящая только от $k_0$ и $q$.
\end{theorem}

Следствием теоремы~\ref{th:smm-2} является гипоэллиптичность уравнения Бельтрами.

\section{Периодическая задача для уравнения Бельтрами}
Рассмотрим периодическую задачу
\begin{equation}\label{eq:smm-9}
Au\equiv\partial_{\bar{\xi}}N(y)+\mu\partial_z N(y)=g(y)\in L_r(\square) ,\quad  \langle g\rangle=0, \quad N\in W_r^1(\square), \quad \langle N\rangle=0,
\end{equation} 
	где $\square=[-1/2, 1/2)^2$ --- ячейка периодов, $2\leqslant r\leqslant q$, $q>2$ --- показатель повышенной суммируемости, $\mu(y),y=(y_1,y_2 )$ --- периодическая функция (периода 1 по каждой переменной), удовлетворяющая условию эллиптичности \eqref{eq:smm-2},
$$
\partial_{\bar{\xi}}=2^{-1}\left(\frac{\partial}{\partial y_1}+i\frac{\partial}{\partial y_2}\right),\quad \partial_{\xi}=2^{-1}\left(\frac{\partial}{\partial y_1}-i\frac{\partial}{\partial y_2}\right),
$$
\begin{theorem}\label{th:smm-3}
	Оператор $A:W_r^1(\square)\to L_r(\square) $ периодической  задачи  \eqref{eq:smm-9}  фредгольмовый, размерности ядра  $\text{\rm Ker}\,A$  и ядра сопряженного оператора $A^\ast$ равны единице. Кроме того $\text{\rm Ker}\,A^\ast$ имеет базисный вектор $P\in L_2(\square)$ со средним значением, равным единице, $\langle P\rangle=1$, и удовлетворяющий уравнению
	$$
	\partial_\xi\overline{P}+\partial_{\bar\xi}(\mu\overline{P})=0,\quad P\in L_2(\square),                                
	$$
	где производные понимаются в смысле распределений, $\overline{P}$ --- комплексносопряженная $P$ \linebreak функция.
\end{theorem}

Отметим также, что для задачи \eqref{eq:smm-9} имеет место априорная оценка 
$$
\|\partial_{\bar\xi} N\|_{L_r(\square) }+\|\partial_{\xi} N\|_{L_r(\square) }  \leqslant
c\|AN\|_{L_r(\square)},
$$   
из которой следуют оценки
$$
\|N\|_{L_\infty(\square)}\leqslant c, \quad \|N\|_{W_r^1(\square)},
$$         
где $c>0$ --- постоянная, зависящая только от $k_0$ и $r$.

% %\begin{abstract}\Large
% \begin{center} Аннотация
% \end{center}
% Сглаживание функций по Стеклову применяется во многих вопросах анализа и дифференциальных уравнений. При этом в вопросах усреднения дифференциальных уравнений используются $L_2$-свойства сглаживания. В заметке рассмотрены $L_p$-свойства сглаживания в удобном (при усреднении дифференциальных уравнений) виде.

	
	%\end{abstract}
\section{Свойства сглаживания функций по Стеклову}
Пусть $\phi(x)$ функция определенная во всем пространстве  $R^n$, $n\geqslant1$. Сглаживанием по Стеклову функции $\phi(x)$, $x\in R^n$, называется функция определенная равенством
\begin{equation}\label{eq:smm-10}
\phi^\varepsilon (x)=\int_\square \phi(x-\varepsilon \omega)d\omega, \quad   x\in R^n, 
\end{equation}                
где  $\varepsilon$ --- малый параметр,  $0<\varepsilon<1$, $\omega$ --- точка единичного куба $\square=[-1/2,1/2)^n$ с центром в нуле.
В таком виде \eqref{eq:smm-10} сглаживание по Стеклову применяется в вопросах усреднения дифференциальных уравнений. При этом используются $L_2$-свойства сглаживания \eqref{eq:smm-10}. В заметке мы получим аналоги этих свойств в  $L_p$ пространствах.

\begin{property}\label{prop:smm-1}
	Пусть  $\phi\in L_p (R^n)$, $1\leqslant p<+\infty$, тогда 
	$\phi^\varepsilon\in L_p (R^n)$ и имеет место оценка
	$$
	\|\phi^\varepsilon\|_{L_p (R^n )} \leqslant\|\phi\|_{L_p (R^n)}.    
	$$
\end{property}

\begin{property}\label{prop:smm-2}
	Пусть $\phi$ принадлежит пространству Соболева $W_p^1 (R^n)$, тогда имеет место оценка
	$$
	\int_{R^n}|\phi(x-\varepsilon\omega)-\phi(x)|^p  dx \leqslant \varepsilon^p |\omega|^p  \int_{R^n}|\nabla \phi|^p dx,   \quad \omega\in\square,        
	$$
\end{property}


Из предыдущего свойства следует

\begin{property}\label{prop:smm-3}
	Пусть $phi$ принадлежит пространству Соболева $W_p^1 (R^n)$, тогда имеет место оценка
	$$
	\int_{R^n}\left|\int_\square\phi(x-\varepsilon\omega)d\omega-\phi(x)\right|^p  dx \leqslant \varepsilon^p \left(\frac{\sqrt{n}}{2}\right)^p  \int_{R^n}|\nabla \phi|^p dx,         
	$$
\end{property}



