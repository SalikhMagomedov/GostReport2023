
\chapter{Приближенное решение дифференциальных уравнений с помощью рациональных сплайн-функций}

\section*{Введение}

В этом разделе рассмотрены вопросы приближенного решения в виде рациональных сплайн-функций
дифференциальных уравнений вида
\begin{equation}\label{eq:ark-0.1}
a_0(x) y^{\prime\prime}+a_1(x) y^\prime+q(x) y=f(x),\quad a< x<b,
\end{equation}
где $a_0(x)=(x-a)^2(b-x)^2$, $a_1(x)=(x-a)(b-x)p(x)$, функция $p(x)$
непрерывна и ограничена на $(a,b)$, функции $q(x)$ и $f(x)$ непрерывны на $(a,b)$.
Следует отметить, что выбор функций $p(x)$, $q(x)$ и $f(x)$ в \eqref{eq:ark-0.1} дает различные
широко используемые в математической физике и во многих других областях дифференциальные
уравнения, в частности, уравнение Римана \cite{bib:ark-1,bib:ark-2,bib:ark-3}. Такие уравнения глубоко
исследованы, для определенных видов их коэффициентов разработаны методы
решения при помощи степенных рядов и их обобщений.

Известны также приближенные решения частных видов уравнения \eqref{eq:ark-0.1} с различными
определенными ограничениями на их коэффициенты, на функциональный класс допустимого
решения $y(x)$, а также на краевые условия (см., напр., \cite{bib:ark-4,bib:ark-5,bib:ark-6,bib:ark-7,bib:ark-8,bib:ark-9,bib:ark-10}).

В данной разделе разработан новый способ построения приближенного решения \linebreak уравнения
\eqref{eq:ark-0.1} в виде гладких рациональных сплайн-функций, причем структура применяемых
рациональных сплайн-функций позволяет получить сравнительно более простые алгоритмы
построения решения.

В развитие исследований по эффективным приложениям сплайн-функций по рациональным
 интерполянтам к решению дифференциальных уравнений и их систем в работе \cite{bib:ark-19}
 построены непрерывно дифференцируемые рациональные сплайн-решения
 для начальной задачи в случае нормальной системы двух дифференциальных уравнений.

Структура применяемых рациональных сплайн-функций допускает построение сравнительно
простых алгоритмов поиска гладких решений дифференциальных уравнений.

\section{Рациональные сплайн-решения обобщенного дифференциального уравнения Римана}

\subsection{Введение}

Возросший определенный интерес к исследованию обобщенного дифференциального уравнения
Римана, в частности, связан с поиском более эффективных методов решения задач
математической физики, приводящих к обыкновенным дифференциальным уравнениям
с особенностями в коэффициентах производных.

В данном пункте рассмотрен вопрос приближенного решения дифференциального уравнения
достаточно общего вида при минимальных условиях на класс искомого решения, а именно,
допускаем, что уравнение \eqref{eq:ark-0.1} имеет единственное решение
$y=y(x)$ из класса $C^2(a,b)$, имеющее конечные пределы
\begin{equation}\label{eq:ark-0.2}
y(a+0)=A,\quad y(b-0)=B.
\end{equation}

Приближенное решение задачи \eqref{eq:ark-0.1}, \eqref{eq:ark-0.2} строится с помощью дважды
 непрерывно дифференцируемой рациональной сплайн--функции специального вида,
 которая однозначно определяется сеткой попарно различных узлов на отрезке $[a,b]$ и
соответствующими этим узлам конечными дискретными значениями.

Для упрощения конструкций возьмем равномерную сетку узлов
$\Delta_h: x_i=a+ih$, $i=0,1,\dots,N$; $h=\frac{b-a}N$, $N\geqslant 3$.
По тройкам узлов $x_{i-1}, x_i, x_{i+1}$ $(i=1,2,\dots,N-1)$
построим \cite{bib:ark-11} рациональные интерполянты
\begin{equation}\label{eq:ark-0.3}
R_i(x,y)=\alpha_i+\beta_i(x-x_i)+\gamma_i\frac 1{x-g_i}
\end{equation}
с условиями $R_i(x_j,y)=y(x_j)$ при $j=i-1,i,i+1$, а полюсы
$g=\{g_1,g_2,\dots,g_{N-1}\}$ для произвольного $\lambda>0$ определяются
равенством $g_i=x_{i+1}+\lambda h$, $i=1,2,\dots,N-1.$ Положим также
$R_0(x,y)\equiv R_1(x,y)$, $R_N(x,y)\equiv R_{N-1}(x,y)$.

Тогда на отрезке $[a,b]$ можем определить \cite{bib:ark-12} дважды
непрерывно дифференцируемую рациональную сплайн-функцию
$R_{N,2}(x,y)=R_{N,2}(x,y,\Delta_h,g)$ такую, что при $x\in[x_{i-1}, x_i]$,
$i=1,2,\dots,N$, выполняется равенство
\begin{equation}\label{eq:ark-0.4}
R_{N,2}(x,y)=R_i(x,y)A_i(x)+R_{i-1}(x,y)B_i(x),
\end{equation}
где $A_i(x)=(x-x_{i-1})^2/((x-x_{i-1})^2+(x-x_i)^2)$, $B_i(x)=1-A_i(x)$.

\subsection{Основные результаты}

Чтобы выяснить возможность приближенного решения дифференциальной задачи
\eqref{eq:ark-0.1}--\eqref{eq:ark-0.2} с помощью рациональных сплайн-функций вида \eqref{eq:ark-0.4},
важно знать поведение при $h\to 0$ следующей определенной на интервале $(a,b)$
функции-невязки:
\begin{equation}\label{eq:ark-0.5}
G(x, \Delta_h)=a_0(x)R_{N,2}^{\prime\prime}(x,y)+a_1(x)R_{N,2}^\prime(x,y)+
q(x)R_{N,2}(x,y)-f(x).
\end{equation}

Как показано в \cite{bib:ark-12}, при $r=0,1,2$ для сплайн-функций \eqref{eq:ark-0.4} и рациональных
функций \eqref{eq:ark-0.3} во внутренних узлах сетки $\Delta_h$ выполняются равенства
\begin{equation}\label{eq:ark-0.6}
R_{N,2}^{(r)}(x_i,y)=R_i^{(r)}(x_i,y),\quad i=1,2,\dots,N-1.
\end{equation}
Тогда с учетом интерполяционных условий $R_i(x_i,y)=y(x_i)$ из равенств \eqref{eq:ark-0.5},
\eqref{eq:ark-0.6} и уравнения \eqref{eq:ark-0.1} получим при каждом $i=1,2,\dots,N-1$ равенство
\begin{equation*}\label{eq:ark-0.7}
G(x_i,\Delta_h)=a_0(x_i)[R_i^{\prime\prime}(x_i,y)-y^{\prime\prime}(x_i)]+
a_1(x_i)[R_i^\prime(x_i,y)-y^\prime(x_i)].
\end{equation*}
Будем считать решение $y(x)$ продолженным по непрерывности на весь отрезок
$[a,b]$. На рост производных допустимого
решения $y=y(x)$ задачи \eqref{eq:ark-0.1}--\eqref{eq:ark-0.2} в окрестности граничных точек
накладываются ограничения в виде следующих равенств:
\begin{equation}\label{eq:ark-0.10}
\lim_{x\to a+0}(x-a)^m y^{(m)}(x)=0,\quad \lim_{x\to b-0}(b-x)^my^{(m)}(x)=0,
\,\, m=1,2.
\end{equation}
Это позволяет, в частности, ввести для значений $m=1,2$  вспомогательные функции
\begin{equation*}\label{eq:ark-0.11}
F_m(x)=(x-a)^m(b-x)^m y^{(m)}(x),
\end{equation*}
продолженные непрерывно на отрезок $[a,b]$ со значениями $F_m(a)=F_m(b)=0$.

Для оценки невязок $G(x_i, \Delta_h)$ используются модули непрерывности решения $y(x)$
и функций $F_m(x)$, $m=1,2$.
Модули непрерывности первого и второго порядков непрерывной на данном отрезке
$[\alpha, \beta]$ функции $F(x)$ определим, как обычно, через конечные разности
соответственно первого и второго порядков, а именно положим
$$
\omega(\delta, F)=\omega(\delta, F,[\alpha,\beta])=\sup\{|\Delta_t^1(F, x)|:
 0\leqslant t\leqslant \delta;\, x, x+t\in[\alpha, \beta]\},
$$
$$
\omega_2(\delta, F)=\omega_2(\delta, F,[\alpha,\beta])=\sup\{|\Delta_t^2(F, x-t)|:
 0\leqslant t\leqslant \delta; \,x, x\pm t\in[\alpha, \beta]\}.
$$

Сами оценки невязок $G(x_i, \Delta_h)$ даются следующей теоремой.

\begin{theorem}\label{th:ark-theo1}
Пусть задача \eqref{eq:ark-0.1}, \eqref{eq:ark-0.2} имеет решение $y=y(x)$ класса $C^2(a,b)$, удовлетворяющее
условию \eqref{eq:ark-0.10}, $\Delta_h=\{x_i=a+ih |i=0,1,\dots,N\}$, $h=\frac{b-a}N$,
$N\geqslant 4$, функция-невязка $G(x, \Delta_h)$ определена равенством \eqref{eq:ark-0.5}.

Тогда для значений $i=2,3,\dots,N-2$ выполняется неравенство
\begin{equation*}\label{eq:ark-0.12}
|G(x_i,\Delta_h)|\leqslant
\left(10+\frac{6|p(x_i)|}{b-a}\right)\omega(h, F_2, [a,b]),
\end{equation*}
а для значений $i=1,N-1$  -- неравенство
\begin{multline*}\label{eq:ark-0.13}
|G(x_i,\Delta_h)|\leqslant [(b-a)^2+(b-a)|p(x_i)|]\omega_2(h, y, [x_{i-1}, x_{i+1}])+
\\
+3|p(x_i)|\omega(h, F_1,[x_{i-1}, x_{i+1}])+
\omega(h, F_2, [x_{i-1}, x_{i+1}]).
\end{multline*}
\end{theorem}

Так как функция $p(x)$ ограничена на $(a,b)$, из теоремы \ref{th:ark-theo1} при $h\to 0$
следует, что
$\max\{|G(x_i, \Delta_h)|: i=1,2,\dots,N-1\}\to 0$.
Для построения приближенного решения возьмем произвольное натуральное $N\geqslant 3$,
равномерную сетку
$\Delta_h: a=x_0<x_1<\dots<x_N=b$ с $h=x_i-x_{i-1}$ и множество полюсов
$g=\{g_1,g_2,\dots,g_{N-1}\}$ с $g_i=x_{i+1}+\lambda h$, где параметр $\lambda=N-1$.

Через $R_{N,2}(x)$ обозначим рациональную сплайн-функцию, составленную
аналогично $R_{N,2}(x,y)$ из \eqref{eq:ark-0.4} по сетке узлов $\Delta_h$, соответствующим этим узлам
неизвестным
значениям $y_0,y_1,\dots,y_N$, множеству полюсов $g$ и рациональным интерполянтам
$R_i(x)$, $1\leqslant i\leqslant N-1$, вида \eqref{eq:ark-0.3} таким, что $R_i(x_j)=y_j$ при
$j=i-1,i,i+1$, и $R_0(x)\equiv R_1(x)$, $R_N(x)\equiv R_{N-1}(x)$.
Для нахождения неизвестных $y_0,y_1,\dots,y_N$ составим следующую систему уравнений:
\begin{equation}\label{eq:ark-2.1}
\begin{split}
&R_{N,2}(a)=A, \quad R_{N,2}(b)=B,\\
&a_0(x_i)R^{\prime\prime}_{N,2}(x_i)+a_1(x_i)R^\prime_{N,2}(x_i)+q(x_i)R_{N,2}(x_i)=f(x_i),
\quad 1\leqslant i\leqslant N-1.
\end{split}
\end{equation}

Следующее утверждение дает достаточные условия на коэффициенты дифференциального
уравнения \eqref{eq:ark-0.1},
при выполнении которых система уравнений однозначно разрешима относительно
$y_0,y_1,\dots,y_N$.

\begin{theorem}\label{th:ark-theo2}
Пусть $0<\rho<2(b-a)$ и для коэффициентов уравнения \eqref{eq:ark-0.1} выполнены условия
\begin{equation}\label{eq:ark-2.2}
|p(x)|\leqslant \rho,\quad q(x)<0 \text{ при } x\in (a,b).
\end{equation}
Тогда система \eqref{eq:ark-2.1} имеет единственное решение $(y_0,y_1,\dots,y_N)$ для
всех $N$, начиная с некоторого номера $N(\rho)$; $N(\rho)=3$ при $\rho=\frac 89 (b-a)$.
\end{theorem}

Следующее утверждение дает достаточные условия сходимости приближенного решения
в узлах $x_1,x_2, \dots,x_{N-1}$ сеток к точному решению задачи \eqref{eq:ark-0.1}--\eqref{eq:ark-0.2}
в терминах величины
$$
G_{h,q}=\max\left\{\left|\frac{G(x_i,\Delta_h)}{q(x_i)}\right|: i=1,2,\dots,N-1\right\},
$$
где $G(x,\Delta_h)$ -- функция-невязка, определяемая равенством \eqref{eq:ark-0.5}, $q(x)$ --
коэффициент уравнения \eqref{eq:ark-0.1}.

\begin{theorem} \label{th:ark-theo3}
Если для коэффициентов уравнения \eqref{eq:ark-0.1} выполнены условия \eqref{eq:ark-2.2}, то для
точного решения $y(x)$ задачи \eqref{eq:ark-0.1}, \eqref{eq:ark-0.2} и решения $(y_1,y_2,\dots,y_{N-1})$
системы \eqref{eq:ark-2.1} при $i=1,2,\dots,N-1$ выполняется неравенство
\begin{equation}\label{eq:ark-3.1}
|y(x_i)-y_i|\leqslant G_{h,q}.
\end{equation}
\end{theorem}

Следовательно, если величина $G_{h,q}\to 0$ при $h\to 0$, то из \eqref{eq:ark-3.1} вытекает сходимость
приближенного решения $(y_1,y_2,\dots,y_{N-1})$ в узлах сетки к точному решению
$y(x)$ задачи \eqref{eq:ark-0.1}, \eqref{eq:ark-0.2}.


\section{Гладкие рациональные сплайн-решения начальной задачи для нормальной системы}

\subsection{Введение}

Полиномиальные сплайны и их важные приложения в самых разных областях науки и
техники исследованы многими авторами (см., напр., \cite{bib:ark-14,bib:ark-15,bib:ark-16} и цитированную в них
литературу).
Рассматривались также приложения рациональных сплайн-функций различных видов,
в основном, для решения задач о формосохраняющих интерполяциях дискретных данных
(см., напр., \cite{bib:ark-16,bib:ark-17,bib:ark-18}).
Много научных работ посвящено численным решениям дифференциальных уравнений и
их систем с использованием полиномиальных сплайнов (см., напр., \cite{bib:ark-14,bib:ark-15,bib:ark-16}).

Определенный интерес представляют вопросы численного решения систем нормальных
дифференциальных уравнений с помощью гладких рациональных сплайн-функций специального
вида. Дело в том, что структура этих сплайн-функций позволяет построить сравнительно
более простые алгоритмы поиска гладких решений систем дифференциальных уравнений.
Гладкостные свойства таких рациональных сплайн-функций, вопросы их сходимости,
приложения к решению некоторых
дифференциальных уравнений рассмотрены в \cite{bib:ark-11,bib:ark-12,bib:ark-13}.

В этом пункте речь идет о начальной задаче вида
\begin{gather}\label{eq:ark-4.1}
\frac{dy}{dx}=F_1(x,y,z),
\\\label{eq:ark-4.2}
\frac{dz}{dx}=F_2(x,y,z),
\\\label{eq:ark-4.3}
y(a)=A_1,\quad z(a)=A_2.
\end{gather}

Будем считать, что задача \eqref{eq:ark-4.1}--\eqref{eq:ark-4.3} имеет единственное решение
$(y(x), z(x))$ класса $C^1$ на рассматриваемом промежутке $[a,b]$ или
$[a,b)$, а функции $F_1(x,y,z)$ и $F_2(x,y,z)$ имеют ограниченные частные
 производные первого порядка по переменным $y$ и $z$ в некоторой трехмерной
области $T$, определяемой промежутком изменения аргумента $x$ и области
изменения функций $y(x)$ и $z(x)$.

\subsection{Основные результаты}

В случае, когда функции  $y(x), z(x)\in C^1[a,b]$, как известно \cite{bib:ark-11},
 для любой сетки узлов $\Delta: a=x_0<x_1<\dots<x_N=b$  $(N>1)$ и
набора полюсов $g=\{g_1, g_2,\dots, g_{N-1}\}$ таких, что для  $\lambda>0$ имеем
$$
g_i=\begin{cases}
x_{i+1}+\lambda h_{i+1} \text{при } h_{i+1}\leqslant h_i,\\
x_{i-1}-\lambda h_i \text{при } h_{i+1}> h_i \quad (i=1,2,\dots,N-1),
\end{cases}
$$
существуют рациональные сплайн-функции $R_{N,1}(x,y)$ и $R_{N,1}(x,z)$,
 для которых при $u(x)=y(x)$ и, отдельно, при $u(x)=z(x)$ выполняются неравенства
\begin{gather*}\label{eq:ark-4.4}
|R_{N,1}(x,u)-u(x)|\leqslant
\left(4+\frac 2\lambda \right)\Delta_i \omega(\Delta_i,u^\prime),
\\\label{eq:ark-4.5}
|R^\prime_{N,1}(x,u)-u^\prime(x)|\leqslant
\left(12+\frac 2\lambda \right)\omega(\Delta_i,u^\prime),
\end{gather*}
где $\Delta_i=\max\{h_{i-1}, h_i, h_{i+1}\}$, $\lambda >0$, $x\in [a,b]$, $\omega(t, u^\prime)$
-- модуль непрерывности производной $u^\prime (x)$ на отрезке $[a,b]$.

Сплайн-функции $R_{N,1}(x,u)=R_{N,1}(x,u,\Delta,g)$ при $x\in [x_{i-1}, x_i]$
$(i=1,2,\dots,N)$ удовлетворяют равенствам
$$
R_{N,1}(x,u)=R_i(x,u) \frac{x-x_{i-1}}{x_i-x_{i-1}}+
R_{i-1}(x,u)\frac{x_i-x}{x_i-x_{i-1}},
$$
где для $i=1,2,\dots,N-1$ рациональные интерполянты
$$
R_i(x,u)=\alpha_{i,u}+\beta_{i,u} (x-x_i)+\gamma_{i,u} \frac 1{x-g_i}
$$
определяются равенствами $R_i(x_j,u)=u(x_j)$  при $j=i-1, i, i+1$, а также полагаем
$$
R_0(x,u)\equiv R_1(x,u),\quad R_N(x,u)\equiv R_{N-1}(x,u).
$$
Тогда из условий интерполяции для $i=1,2,\dots, N-1$ можно найти
$\alpha_{i,u}, \beta_{i,u}, \gamma_{i,u}$.
Для $x\in[a,b]$ обозначим также
\begin{gather*}
  G_1(x)=R_{N,1}^\prime(x,y)-F_1(x, R_{N,1}(x,y),R_{N,1}(x,z)),
  \\
  G_2(x)=R_{N,1}^\prime(x,z)-F_2(x, R_{N,1}(x,y),R_{N,1}(x,z)).
\end{gather*}

В силу условий на частные производные при $k=1,2$ имеем
$\max_{a\leqslant x \leqslant b} |G_k(x)|\to 0$ при $\|\Delta\| \to 0$.
Отсюда вытекает сходимость приближенного решения в виде рациональных
сплайн-функций к точному решению задачи \eqref{eq:ark-4.1} -- \eqref{eq:ark-4.3}.

Введем неизвестные $Y=\{y_0,y_1, \dots,y_N\}$  и $Z=\{z_0,z_1, \dots, z_N\}$,
 в выражениях интерполянтов $R_i(x,y)$ значения $y(x_j)$ заменим на $y_j$
  (с соответствующими индексами), а в выражениях $R_i(x,z)$ значения $z(x_j)$
  заменим на $z_j$.
В результате получим новые рациональные функции $R_i(x,Y)$ и $R_i(x, Z)$ и,
соответственно, новые сплайн-функции $R_{N,1}(x,Y)=R_{N,1}(x,Y, \Delta,g)$
 и $R_{N,1} (x,Z)=R_{N,1}(x,Z,\Delta, g)$.

Для нахождения неизвестных значений $Y=\{y_0,y_1, \dots,y_N\}$ и
$Z=\{z_0,z_1, \dots, z_N\}$, а значит, и искомых сплайн-функций $R_{N,1}(x,Y)$
 и $R_{N,1}(x,Z)$  вместо начальной задачи \eqref{eq:ark-4.1} -- \eqref{eq:ark-4.3}
 рассмотрим следующую систему алгебраических уравнений относительно
 $y_0,y_1, \dots,y_N$ и $z_0,z_1, \dots, z_N$:
\begin{gather*}
  R_{N,1}^\prime(x_i,Y)=F_1(x_i, R_{N,1}(x_i,Y),R_{N,1}(x_i,Z)),\quad i=1,2,\dots,N,
  \\
  R_{N,1}^\prime(x_i,Z)=F_2(x_i, R_{N,1}(x_i,Y),R_{N,1}(x_i,Z)),\quad i=1,2,\dots,N,
  \\
  R_{N,1}(a,Y)=A_1,\quad R_{N,1}(a,Z)=A_2.
\end{gather*}

Приближенные решения подобных систем строим двояко в зависимости от того, решение
принадлежит классу $C^1[a,b]$  или классу $C^1[a,b)$.






















