
% \documentclass[11pt,twoside]{article}
% \usepackage[cp1251]{inputenc}
% \usepackage[russian]{babel}
% \usepackage[cp1251]{inputenc}
% \usepackage{amsmath}
% \usepackage{amssymb}
% \usepackage{amsfonts}
% \usepackage{amssymb,amsmath,amscd}
% \usepackage{fancyheadings}
% \usepackage[russian]{babel}
% \usepackage{bm}
% \textwidth=165mm
% \oddsidemargin=-4.4mm
% \evensidemargin=-4.4mm
% \textheight=234mm
% \topmargin=-16.6mm
% \headheight=5.0mm
% \headsep=6.6mm
% \footskip=8.1mm
% \renewcommand{\baselinestretch}{0.9}
% \usepackage{graphicx}
% \makeatother
% \usepackage{wrapfig}
% \newcommand{\GOD}{2022}
% \newcommand{\UDK}[1]{\noindent{\footnotesize\sl УДК #1}}
% \newcommand{\Nazva}[1]{\begin{center}\baselineskip=6.0mm{\Large\textbf{#1}}\end{center}\vspace*{0.5mm}}
% \newcommand{\Avtor}[1]{\centerline{\large\textbf{\copyright~\GOD~г. \ #1}}\vspace*{-4.0mm}}
% \newcommand{\AVTOR}{~}
% \newcommand{\NAZVA}{~}
% \newcommand{\lit}[3]{\vspace*{0.7mm}\par\noindent\makebox[5.2mm][r]{#1.~}\parbox[t]{159.8mm}{{\textit{#2}}~{#3}}\hspace*{-1.6mm}}
% \mathsurround=2pt
% \renewcommand{\thefootnote}{\fnsymbol{footnote}}

% \begin{document}

\chapter{Краткая аннотация важнейших и основных научных результатов,
полученных в 2022 г.}

1. Изучены вопросы моментной устойчивости решений по части
переменных относительно начальных данных для систем линейных
дифференциальных уравнений Ито с последействием. Исследование
проводится модифицированным методом регуляризации, основанным на
выборе вспомогательного уравнения и применении теории неотрицательно
обратимых матриц. Для упомянутых систем получены достаточные условия
устойчивости в терминах неотрицательной обратимости матриц,
построенных по параметрам этих систем. Проверяется выполнимость этих
условий для конкретных классов систем линейных уравнений Ито с
последействием. Результаты исследований опубликованы в работах \cite{bib:kad-1},
\cite{bib:kad-2}.
\smallskip

2. Исследована глобальная моментная устойчивость систем
дифференциальных уравнений Ито дробного порядка с последествием. Для
этого применяется метод регуляризации в качестве альтернативы
алгоритмам, основанных на функционалах Ляпунова. Центральная идея
этого метода основана на параллелизме между устойчивостью по
Ляпунову и стохастической версией устойчивости по входным данным,
которая хорошо известна в теории управления детерминированными
обыкновенными дифференциальными уравнениями. Первый шаг алгоритма
состоит в преобразовании заданного уравнения с последествием в
функционально-дифференциальное уравнение Ито дробного порядка со
стохастическим управлением и тем самым заменяя устойчивость по
Ляпунову первого уравнения устойчивостью  второго по входному
состоянию. Для оценки норм решений функционально-дифференциального
уравнения Ито дробного порядка мы используем метод регуляризации,
основанный на концепции положительно обратимых матриц. Эффективность
этого метода демонстрируется с помощью некоторых примеров, где метод
функционалов Ляпунова может быть труден в применении. Результаты
исследований  опубликованы в работе \cite{bib:kad-3}.
\smallskip

3. Изучено  асимптотическое поведение моментов решений систем
нелинейных дифференциальных уравнений Ито с запаздываниями. Эти
вопросы тесно связаны с вопросами глобальной моментная устойчивости
этих же систем. Исследования проведены модифицированным методом
регуляризации (W-метод), основанный на выборе вспомогательного
уравнения и применении теории положительно обратимых матриц.
Получены достаточные условия, обеспечивающие наличие определенного
асимптотического поведения для моментов решений как для достаточно
общих, так и конкретных классов уравнений Ито в терминах параметров
этих классов. Установлена связь между асимптотическим поведением
моментов решений для упомянутых систем и запаздываниями. Результаты
исследований  изложены в работе \cite{bib:kad-4}, которая принята к печати в
журнале "Известия вузов. Математика".
\smallskip


\textbf{Введение.} Стохастические дифференциальные уравнения
описывают многие реальные, практически важные процессы современной
физики, биологии, иммунологии, экономики, кибернетики и т.д.
Изучение таких процессов привело к необходимости исследований
вопросов устойчивости решений стохастических дифференциальных
уравнений, т.е. к созданию соответствующего направления в теории
устойчивости.

Математические модели многих реальных процессов, учитывающие
состояния процессов не только в текущий, но и в предшествующие
моменты времени описывают развитие процессов более точнее. В этом
случае поведение процесса, подверженного случайным воздействиям,
описывается стохастическими дифференциальными уравнениями с
последействием. Примеры показывают, что поведение решений уравнений
без учета запаздывания, даже при малой его величине, может
существенно отличаться от поведения решений тех же уравнений с
запаздывающим аргументом. Это обстоятельство подчеркивает
необходимость и принципиальную важность изучения уравнений с учетом
запаздывания.

Вопросам устойчивости решений систем со случайными параметрами
посвящено большое количество работ как отечественных, так и
зарубежных математиков. Достаточно полный их список приведен в
монографиях. Исследования устойчивости в этих и многих других
работах проводятся методом функционалов
Ляпунова-Красовского-Разумихина. Этот метод предполагает
существование подходящей функции Ляпунова (функционала
Ляпунова-Красовского), которая обеспечивает желаемое свойство
устойчивости (асимптотического поведения) решений исследуемых
уравнений. Однако применение этого метода для
функционально-дифференциальных уравнений, частным случаем которых
являются дифференциальные уравнения с отклоняющим аргументом, во
многих случаях встречает серьёзные трудности. В теории устойчивости
решений для детерминированных функционально--дифференциальных
уравнений широкое применение и высокую эффективность показал метод
регуляризации, основанный на выборе вспомогательных или "модельных"
{\,} уравнений  --- "W--метод" {\,} Н.В. Азбелева.

Исследованиям проблем устойчивости  по части переменных  решений для
детерминированных динамических систем и их приложениям посвящены
много работ. Главной целью исследований за отчетный период является
развитие метода вспомогательных уравнений на основе теории
неотрицательно обратимых матриц и покомпонентных оценок решений
применительно к исследованию вопросов моментной устойчивости решений
по части переменных для систем линейных дифференциальных уравнений
Ито с последействием относительно начальных данных. Этот подход
ранее показал свою эффективность при исследовании вопросов
устойчивости решений для систем линейных дифференциальных уравнений
Ито с последействием и, как показано в настоящем отчете, позволяет
получить новые, конструктивные результаты устойчивости решений по
части переменных относительно начальных данных для детерминированных
и стохастических систем с запаздываниями и без него.
\smallskip

\textbf{1.Постановка задачи.} В отчете используются
следующие обозначения:\\
 - $(\Omega , {\mathcal F}, ({\mathcal
F})_{t\ge0},P)$ --- стохастический базис, здесь $ \Omega $ ---
множество
 элементарных событий, ${\mathcal F}$ --- $\sigma$--алгебра событий на
 $\Omega$,  $({\mathcal F})_{t\ge 0}$ --- непрерывный справа неубывающий поток
 $\sigma$--подалгебр алгебры ${\mathcal F}$, $P$ --- вероятностная
 мера на ${\cal F}$, и  все $\sigma$--алгебры являются полными относительно этой
 меры;\\
-  $k^n$ --- линейное пространство $n$--мерных ${\mathcal F}_0$ ---
измеримых случайных величин;\\
- $\mathcal B_i,i=2,...,m$ --- скалярные независимые стандартные
винеровские процессы;\\
 -  $E$ --- символ математического
ожидания;\\
- $\bar E$ -- единичная $k \times k$--матрица;\\
-  $|.|$ --- норма в $R^n$;\\
- $||.||$
--- норма $k\times n$--матрицы, согласованная с нормой в $R^n$;\\
- $\mu$ --- мера Лебега на $[0,\infty)$.


В рамках отчета следующие константы также остаются фиксированными:\\
- $n \in N$ --- размерность фазового пространства уравнения, т.е. размер вектора решения уравнения;\\
- $p$ --- фиксированная вещественная константа, $1 \le  p < \infty $;\\
- $q$ --- фиксированная вещественная константа, $1 \le  q < \infty $;\\
- $l$ --- фиксированное натуральное число такое, что $1 \le  l < n$.

За отчетный период были исследованы вопросы моментной устойчивости
решений по части переменным для системы линейных дифференциальных
уравнений Ито с запаздываниями вида
\begin{equation}
\label{eq:kri-1}
\begin{array}{crl}
dx(t) = - \sum \limits_{j=1}^{m_1}A_{1j}(t)x(h_{1j}(t))dt + \sum
\limits_{i=2}^m\sum
\limits_{j=1}^{m_i}A_{ij}(t)x(h_{ij}(t))d\mathcal B_i(t) \, (t \ge
0)
\end{array}
\end{equation}
относительно начальных данных
\begin{equation}
\label{eq:kri-1a}
x(t) = \varphi(t) {\,\,} (t <0), {\,\,}   
\end{equation}
\begin{equation}
\label{eq:kri-1b}
x(0) = \upsilon,  
\end{equation}
где:

1. $x = col(x^1,...,x^n)$ --- $n$--мерный неизвестный случайный
процесс;

2.  $A_{ij} = (a^{ij}_{sl})^n_{s,l=1}$ --- $n \times n$--матрица при
$i = 1,...,m$, $j = 1,...,m_i$, элементами матриц $A_{1j}$, $j =
1,...,m_1$ являются прогрессивно измеримые скалярные случайные
процессы, траектории которых почти наверно (п.н.) локально
суммируемы, а элементами матриц $A_{ij}$, $i = 2,...,m$, $j =
1,...,m_i$ являются прогрессивно измеримые скалярные случайные
процессы, траектории которых п.н. локально суммируемы с квадратом;

3.  $ h_{ij}$ --- измеримая по Борелю функция, заданная на $[0,
\infty)$ такая, что $ h_{ij}(t)\leq t {\,} {\,} (t \in [0, \infty))$
$\mu $--почти всюду при $i = 1,...,m$, $j = 1,...,m_i$;

4.  $\varphi = col (\varphi _1,..., \varphi _n)$ --- ${\mathcal
F}_0$--измеримый $n$--мерный случайный процесс, заданный на
$(-\infty , 0)$;

5.  $\upsilon = col (\upsilon_1,.., \upsilon_n)$ --- ${\mathcal
F}_0$--измеримая $n$--мерная случайная величина, т.е. $\upsilon \in
k^n$.

\textbf {Определение 1.}  Под решением системы \eqref{eq:kri-1}, удовлетворяющим
начальным условиям \eqref{eq:kri-1a} и \eqref{eq:kri-1b}, понимается случайный процесс $x(t) =
col (x^1(t), ..., x^n(t))$ $ (t \in (-\infty , \infty))$,
прогрессивно измеримый при  $t \ge 0$, удовлетворяющий условиям
$x(\varsigma)=\varphi (\varsigma) {\,} (\varsigma < 0)$, $x(0) =
\upsilon$ и системе
$$
\begin {array}{crl}
 x(t) =   \upsilon - \sum \limits_{j=1}^{m_1}\int \limits _0^t A_{1j}(\varsigma)x(h_{1j}(\varsigma))d\varsigma
 + \sum \limits_{i=2}^m\sum
\limits_{j=1}^{m_i}\int \limits
 _0^t A_{ij}(\varsigma)x(h_{ij}(\varsigma))d\mathcal B_i(\varsigma)
 \,\, (t \ge 0)
\end {array}
$$
$P$--почти всюду, где первый интеграл --- интеграл Лебега, а второй
--- интеграл Ито.

Систему \eqref{eq:kri-1} с условиями \eqref{eq:kri-1a} и \eqref{eq:kri-1b} называют начальной задачей, а
условия \eqref{eq:kri-1a}, \eqref{eq:kri-1b} --- начальными условиями. Если $h_{ij}(t) = t$
$(t \geq 0)$  $\mu $--почти всюду при $i = 1,...,m$, $j =
1,...,m_i$, то условие \eqref{eq:kri-1a} лишнее. В этом случае \eqref{eq:kri-1}, \eqref{eq:kri-1b} называют
задачей Коши для системы линейных дифференциальных уравнений Ито.

Пусть в дальнейшем:

- $D^n$ --- линейное пространство $n$--мерных прогрессивно измеримых
случайных процессов на $[0, \infty )$, траектории которых п.н.
непрерывны;

- $L^n$ --- линейное пространство $n$--мерных случайных процессов на
$(-\infty , 0)$, которые не зависят от винеровских процессов
$\mathcal B_i, i = 2, ..., m$ и имеют п.н. ограниченные в
существенном траектории;

- $\gamma :[0, \infty) \rightarrow R^1 $
--- положительная непрерывная функция.

Отметим, что при сделанных предположениях задача \eqref{eq:kri-1}, \eqref{eq:kri-1a}, \eqref{eq:kri-1b}
имеет единственное решение. Обозначим через $x(t, \upsilon,
\varphi)$ $(t \in (-\infty , \infty ))$ решение системы \eqref{eq:kri-1},
удовлетворяющее условиям \eqref{eq:kri-1a} и \eqref{eq:kri-1b}, т.е. $x(t, \upsilon, \varphi )
= \varphi (t)$ при $t < 0$ и $x(0, \upsilon, \varphi ) = \upsilon$.
Очевидно, что при $t \ge 0$ имеем $x(., \upsilon, \varphi) \in D^n$.
Заметим также, что при нулевых начальных условиях \eqref{eq:kri-1a}, \eqref{eq:kri-1b} задача
\eqref{eq:kri-1}, \eqref{eq:kri-1a}, \eqref{eq:kri-1b} имеет только тривиальное решение.

Для любого $x = col (x^1, ..., x^n)$ введем обозначения $y =  col
(x^1, ..., x^l)$ и $\xi = col (x^{l+1                   }, ...,
x^n)$.

Введем также обозначения для следующих линейных нормированных
подпространств пространств $D^l$, $k^n$, $L^n$:

$ M_q^{\gamma } = \left \{x: x \in D^l, ||x||_{M_q^\gamma }
 \mathrel
 {\mathop {=} \limits ^{def}} \mathrel {\mathop {\sup}
 \limits _{t
 \ge 0}} (E|\gamma (t)x(t)|^q)^{1/q} < \infty \right \}  (
 M_q^1 = M_q)$;

$ k_q^n = \left \{\alpha: \alpha \in k^n, ||\alpha ||_{k_q^n}
 \mathrel
 {\mathop {=} \limits ^{def}} (E|\alpha |^q)^{1/q} < \infty
 \right  \}$;

 $L_q^n = \left \{\varphi: \varphi \in L^n,
 ||\varphi||_{L_q^n}
 \mathrel {\mathop {=} \limits ^{def}} \mathrel {\mathop
 {vrai \sup}
 \limits _{\varsigma < 0}}(E|\varphi (\varsigma ) |^q)^{1/q} < \infty
 \right \}$.


\textbf {Определение 2.}  Тривиальное решение $x(t,0, 0)\equiv 0$
системы \eqref{eq:kri-1} (или систему \eqref{eq:kri-1}) назовем:

---  {\it $q$--устойчивым} относительно первых $l$ компонент, если
для любого $\epsilon > 0$ найдется такое $\delta (\epsilon) > 0$,
что при любых $\upsilon \in k^n_q$, $\varphi \in L^n_q$ и
$\|\upsilon\|_{k^n_q} + \|\varphi \|_{L^n_q} < \delta (\epsilon)$
будет выполнено неравенство $(E|y(t,\upsilon, \varphi)|^q)^{1/q} \le
\epsilon $ для любого $t \ge 0$;

---  {\it асимптотически $q$--устойчивым }относительно первых $l$ компонент, если
оно $q$--устойчи\- во, и, кроме того, для любых $\upsilon \in
k^n_q$, $\varphi \in L^n_q$ и $\|\upsilon\|_{k^n_q} + \|\varphi
\|_{L^n_q} < \delta (\epsilon)$ будет $\lim \limits_{t  \rightarrow
+\infty }(E|y(t,\upsilon, \varphi)|^q)^{1/q}$ $ = 0$;

---  {\it экспоненциально $q$--устойчивым }относительно первых $l$ компонент, если
существуют некоторые положительные числа $\bar c, \beta$ такие, что
для любых $\upsilon \in k^n_q$, $\varphi \in L^n_q$ справедливо
неравенство $(E|y(t,\upsilon, \varphi)|^q)^{1/q} \le \bar
c\left(\|\upsilon\|_{k^n_q} + \|\varphi \|_{L^n_q}\right )\exp
\{-\beta t\}$.

Следующее определение объединяет все виды устойчивости из
Определения 2.

\textbf {Определение 3.}  Систему \eqref{eq:kri-1} назовем  $M_q^\gamma y
$--устойчивым, если для любых $\upsilon \in k^n_q$, $\varphi \in
L^n_q$ для решения задачи \eqref{eq:kri-1}, \eqref{eq:kri-1a}, \eqref{eq:kri-1b}   $x(t,\upsilon,
\varphi)(t \in (-\infty, +\infty))$ выполняются  соотношение $y(.,
\upsilon, \varphi)|_{[0,\infty)} \in M_q^\gamma$ и неравенство
\begin{equation}
    \label{eq:kri-2}
    \|y(., \upsilon, \varphi)|_{[0,\infty)}\|_{M_q^\gamma} \le \bar c\left(\|\upsilon\|_{k^n_q} +
    \|\varphi \|_{L^n_q}\right)
\end{equation}
для некоторого положительного числа $\bar c$.

Очевидно, что:

--- из  $M_qy$--устойчивости системы \eqref{eq:kri-1}
следует $q$--устойчивость этой же системы относительно  первых $l$
компонент;

--- из $M_q^\gamma y$--устойчивости системы \eqref{eq:kri-1}
(где $\gamma (t) \ge \delta > 0$ $(t \ge 0)$ и $\lim \limits _{t
\rightarrow +\infty } \gamma (t) = +\infty )$ следует
асимптотическая $q$--устойчивость этой же системы относительно
первых $l$ компонент;

--- из $M_q^\gamma y$--устойчивости системы \eqref{eq:kri-1}
(где $\gamma (t) = \exp \{\beta t\}$, $\beta$ --- некоторое
положительное число) следует экспоненциальная $q$--устойчивость этой
же системы относительно первых $l$ компонент.


Для установления $M_q^\gamma y$--устойчивости системы \eqref{eq:kri-1} необходимо
проверить принадлежность вектора $y(., \upsilon,\varphi
)|_{[0,\infty)}$ составленный из первых $l$ компонент решения задачи
\eqref{eq:kri-1}, \eqref{eq:kri-1a}, \eqref{eq:kri-1b} $x(t,\upsilon, \varphi)(t \in (-\infty, +\infty))$
пространству $M_q^\gamma$ при любых $\upsilon \in k^n_q$, $\varphi
\in L^n_q$ и выполнимость для него неравенства \eqref{eq:kri-2}. Для этого будет
использована модификация метода вспомогательных  или модельных
уравнений, известная также как $W$--метод \cite{bib:kad-5}, основанная на теории
положительно обратимых матриц и покомпонентных оценках решений.
\smallskip

\textbf {2. Метод исследования.} Как было отмечено ранее,
устойчивость системы \eqref{eq:kri-1} относительно первых $l$ компонент будем
проверять преобразованием этой системы с помощью вспомогательной
(модельной) системы в другую, более простую систему, для которой
условия, обеспечивающие устойчивость системы \eqref{eq:kri-1} относительно первых
$l$ компонент можно проверить непосредственно.

В дальнейшем будем пользоваться следующим представлением для решения
задачи \eqref{eq:kri-1}, \eqref{eq:kri-1a}, \eqref{eq:kri-1b}: $x(t) = \bar x(t) + \bar \varphi (t)$, где
$\bar x(t)$ --- неизвестный $n$--мерный случайный процесс на
$(-\infty, \infty)$ такой, что $\bar x(t) = 0$ при $t < 0$ и $\bar
x(t) = x(t)$ при $t \geq 0$,  а $\bar  \varphi (t)$ --- известный
$n$--мерный случайный процесс на $(-\infty, \infty)$ такой, что
$\bar \varphi(t) = \varphi (t)$ при $t < 0$ и $\bar \varphi(t) = 0$
при $t \geq 0$.  Тогда  задача \eqref{eq:kri-1}, \eqref{eq:kri-1a}, \eqref{eq:kri-1b} эквивалентна
следующей задаче

\begin{equation}
    \label{eq:kri-3}
    d\bar x(t) =  ((V\bar x)(t) +f(t))dZ(t) {\,} (t \ge 0),
\end{equation}
\begin{equation}
\label{eq:kri-3b}
\bar x(0) = \upsilon,  
\end{equation}
где

$(V\bar x)(t)= \left(- \sum \limits_{j=1}^{m_1}A_{1j}(t)\bar
x(h_{1j}(t)), \sum \limits_{j=1}^{m_2}A_{2j}(t)\bar x(h_{2j}(t)),
..., \sum \limits_{j=1}^{m_m}A_{mj}(t)\bar x(h_{mj}(t))\right ),$

$f(t)= \left(- \sum \limits_{j=1}^{m_1}A_{1j}(t)\bar
\varphi(h_{1j}(t)), \sum \limits_{j=1}^{m_2}A_{2j}(t)\bar \varphi
(h_{2j}(t)), ..., \sum \limits_{j=1}^{m_m}A_{mj}(t)\bar \varphi
(h_{mj}(t))\right),$

$Z(t)= col (t, \mathcal  B_2(t), ...,\mathcal B_m(t)).$

{\bf Замечание 1.} В детерминированном случае аналог уравнения \eqref{eq:kri-3}
является частным случаем функционально-дифференциального уравнения,
и оно записано, используя линейные операторы внутренней
суперпозиции.

Решение задачи \eqref{eq:kri-3}, \eqref{eq:kri-3b} обозначим через  $\bar x(t, \upsilon,
\varphi )$. Очевидно, что при $t   \geq 0$ имеем $x(t, \upsilon,
\varphi ) = \bar x(t, b, \varphi )$.

Пусть:

---  $I^n(Z)$ --- линейное пространство $n\times m$--матриц на $[0,
+\infty )$, строки которых $m$--мерные прогрессивно измеримые
случайные процессы, локально интегрируемые по $Z$;

---  $\bar D^n$ --- линейное пространство $n$--мерных случайных процессов на $(-\infty ,
+\infty )$ равных нулю при $t < 0$, прогрессивно измеримых при $t
\geq 0$ и имеющих п.н. непрерывные траектории на  $[0, \infty )$;

--- $\bar M_q^{\gamma } = \left \{x: x \in \bar D^l, ||x||_{\bar M_q^\gamma }
 \mathrel
 {\mathop {=} \limits ^{def}} \mathrel {\mathop {\sup}
 \limits _{t \in (-\infty, \infty )}} (E|\gamma (t)x(t)|^q)^{1/q} < \infty \right \}  (
 \bar M_q^1 = \bar M_q)$.

Очевидно, что  для уравнении \eqref{eq:kri-3} $V$  --- линейный оператор,
действующий из пространства $\bar D^n$ в пространство $I^n(Z)$ и $f
\in  I^n(Z)$.

Справедлива следующая  лемма.

\textbf {Лемма.}  При любых  $\upsilon \in k^n$, $\varphi \in L^n$
для решения задачи \eqref{eq:kri-3}, \eqref{eq:kri-3b}  $\bar x(t, \upsilon, \varphi)$ имеет
место представление
\begin{equation}
\label{eq:kri-4}
\bar x(t,\upsilon , \varphi) = X(t)\upsilon +(\hat Cf)(t){\,\,} (t
\ge 0), 
\end{equation}
где $X(t)(X(0) = \bar E$) --- $n \times n$--матрица, столбцами
которой являются решения однородного уравнения \eqref{eq:kri-3} (т.е. в случае $f
\equiv 0$) ({\it фундаментальная матрица}), а $\hat C:I^n(Z)
\rightarrow D^n$
--- линейный оператор ({\it оператор Коши}) такой, что
$\hat Cf$ --- решение уравнения \eqref{eq:kri-3}, удовлетворяющее условию $(\hat
Cf)(0) = 0$.

{\bf Замечание 2.} Формула \eqref{eq:kri-4} имеет место для решения задачи \eqref{eq:kri-3},
\eqref{eq:kri-3b} и в случае любого  $f \in I^n(Z)$.  Кроме того, из Леммы также
следует, что при любых  $\upsilon \in k^n$, $\varphi \in L^n$  для
решения задачи \eqref{eq:kri-1}, \eqref{eq:kri-1a}, \eqref{eq:kri-1b}  $ x(t,\upsilon, \varphi)$ при $t\geq
0$ имеет место представление \eqref{eq:kri-4}, где $f$ определена в уравнении
\eqref{eq:kri-3}.

Представление \eqref{eq:kri-4} является центральным результатом в теории
устойчивости решений уравнения \eqref{eq:kri-1}.  В силу этого представления,
асимптотические свойства решений системы \eqref{eq:kri-1}, в том числе и
компонент решений системы \eqref{eq:kri-1}, определяются фундаментальной матрицей
и оператором Коши для уравнения \eqref{eq:kri-3}.

В дальнейшем, если $B$ --- $n\times n$--матрица, то $B^1$ ---
$l\times l$--матрица, полученная из матрицы $B$ зачеркиванием $n - l
$ последних столбцов и строк, $B^2$ --- $l\times (n - l)$--матрица,
полученная из матрицы $B$ зачеркиванием $l$ первых столбцов и $n - l
$ последних строк, $B^3$ --- $(n - l)\times l$--матрица, полученная
из матрицы $B$ зачеркиванием первых $l$ строк и $n - l $ последних
столбцов, $B^4$ --- $(n - l)\times (n - l)$--матрица, полученная из
матрицы $B$ зачеркиванием $l $ первых строк и столбцов, $B^5$ ---
$l\times n$--матрица, полученная из матрицы $B$ зачеркиванием $n - l
$ последних строк, $B^6$ --- $(n - l)\times n$--матрица, полученная
из матрицы $B$ зачеркиванием $l $ первых строк. Тогда, с учетом этих
обозначений и так как $\bar x = col(\bar y, \bar \xi)$, в силу
введенных ранее обозначений, уравнение \eqref{eq:kri-3} можно записать в виде
следующей системы
\begin{equation}
\label{eq:kri-5}
\left \{
\begin{array}{crl}
d\bar y(t) = [(V_1\bar y)(t) + (V_2\bar \xi)(t) + f^y(t)]dZ(t) {\,\,} (t \ge 0),\\
d\bar \xi(t) = [(V_3\bar y)(t) + (V_4\bar \xi)(t) + f^\xi(t)]dZ(t)
{\,\,} (t \ge 0),
\end{array}
\right. 
\end{equation}
где \\
$ (V_i\bar y)(t)= \left(- \sum \limits_{j=1}^{m_1}(A_{1j}(t))^i\bar
y(h_{1j}(t)), \sum \limits_{j=1}^{m_2}(A_{2j}(t))^i\bar
y(h_{2j}(t)), ..., \sum \limits_{j=1}^{m_m}(A_{mj}(t))^i\bar
y(h_{mj}(t))\right ),i=1,3, $\\
$ (V_i\bar\xi)(t)= \left(- \sum \limits_{j=1}^{m_1}(A_{1j}(t))^i\bar
\xi(h_{1j}(t)), \sum \limits_{j=1}^{m_2}(A_{2j}(t)^i)\bar
\xi(h_{2j}(t)), ..., \sum \limits_{j=1}^{m_m}(A_{mj}(t))^i\bar
\xi(h_{mj}(t))\right ),i=2,4, $\\
$ f^y(t)(t)= \left(- \sum \limits_{j=1}^{m_1}(A_{1j}(t))^5\bar
\varphi(h_{1j}(t)), \sum \limits_{j=1}^{m_2}(A_{2j}(t))^5\bar
\varphi(h_{2j}(t)), ..., \sum \limits_{j=1}^{m_m}(A_{mj}(t))^5\bar
\varphi(h_{mj}(t))\right ), $\\
$ f^\xi(t)(t)= \left(- \sum \limits_{j=1}^{m_1}(A_{1j}(t))^6\bar
\varphi(h_{1j}(t)), \sum \limits_{j=1}^{m_2}(A_{2j}(t))^6\bar
\varphi(h_{2j}(t)), ..., \sum \limits_{j=1}^{m_m}(A_{mj}(t))^6\bar
\varphi(h_{mj}(t))\right ). $

В дальнейшем, если $\zeta (t)$  --- случайный процесс на
$[0,\infty)$, то $\bar \zeta (t)$ также является  случайным
процессом на  $(- \infty , \infty )$, значения которого  совпадают с
значениями процесса $\zeta (t)$ на $[0,\infty)$ и с нулем на
$(-\infty , 0)$.

В силу того, что через любое $\bar x(0) \in k^n$ проходит
единственное решение уравнения \eqref{eq:kri-3}, каждое из уравнений системы \eqref{eq:kri-5}
в отдельности будет иметь единственное решение при любых
фиксированных $\bar y(0), \bar \xi $ и $\bar \xi(0), \bar y$
соответственно. Тогда, в силу Леммы, второе уравнение системы \eqref{eq:kri-5}
эквивалентно уравнению
$$
\bar \xi(t) = H(t)\bar \xi(0) + (C_1(f^\xi + V_3\bar y))(t) {\,\,}
(t \ge 0),
$$
где $H$ --- фундаментальная матрица, а $C_1$ --- оператор Коши, для
второго уравнения системы \eqref{eq:kri-5}.

Следовательно, из первого уравнения системы \eqref{eq:kri-5} получим
\begin{equation}
    \label{eq:kri-6}
    \begin{array}{crl}
    d\bar y(t) = [(V_5\bar y)(t) + (V_2(\bar H\bar \xi(0)))(t)  +
    (V_2\bar C_1f^\xi)(t) + f^y(t)]dZ(t){\,\,} (t \ge 0),
    \end{array}
\end{equation}
где $(V_5\bar y)(t) = (V_1\bar y)(t)+ (V_2\bar C_1V_3\bar y)(t)$.


Отсюда следует, что система \eqref{eq:kri-1} $M_q^\gamma y$--устойчива тогда и
только тогда, когда при любых $\bar x(0) = \upsilon \in k^n_q$,
$\varphi \in L^n_q$ для решения уравнения \eqref{eq:kri-6} $\bar y(t, \upsilon,
\varphi)$ выполняются соотношение $\bar y(., \upsilon, \varphi) \in
\bar M_q^\gamma$ и неравенство
\begin{equation}
\label{eq:kri-7}
\|\bar y(., \upsilon, \varphi)\|_{\bar M_q^\gamma} \le \bar c\left(\|\upsilon\|_{k^n_q} +
 \|\varphi \|_{L^n_q}\right)   
\end{equation}
для некоторого положительного числа $\bar c$. Для установления
отмеченных фактов воспользуемся $W$--преобразованием, {\,\,} т.е.
эквивалентным преобразованием уравнения \eqref{eq:kri-6}. Для описания
$W$--преобразо\-вания уравнения \eqref{eq:kri-6} рассмотрим модельное уравнение,
асимптотические свойства решений которого известны. Пусть модельное
уравнение имеет вид

\begin{equation}
\label{eq:kri-8}
d\bar y(t) = [(Q\bar y)(t) + g(t)]dZ(t) {\,\,} (t \ge 0), 
\end{equation}
где $Q:\bar D^l \rightarrow I^l(Z)$ --- линейный оператор, $Z$
определен ранее, $g \in I^l(Z)$. Предполагается, что через любое
$\bar y(0) \in k^l$ проходит единственное (с точностью до
P--эквивалентности) решение $\bar y$ уравнения \eqref{eq:kri-8}. Тогда, в силу
Леммы, для решения $\bar y$ этого уравнения имеет место
представление $\bar y(t) = U(t)\bar y(0) + (Wg)(t){\,} (t \ge 0)$,
где $U$ --- фундаментальная матрица, $W$ --- оператор Коши для
уравнения \eqref{eq:kri-8}.

Уравнение \eqref{eq:kri-6} при помощью модельного уравнения \eqref{eq:kri-8} перепишем в виде
$$
d\bar y(t) = [(Q\bar y)(t) + ((V_5 - Q)\bar y)(t) + (V_2(\bar H\bar
\xi(0)))(t) + (V_2\bar C_1f^\xi)(t) + f^y(t)]dZ(t){\,\,} (t \ge 0)
$$
или
$$
\bar y(t) = U(t)\bar y(0) +(W(V_5 - Q)\bar y)(t) + (WV_2(\bar H\bar
\xi(0)))(t) + (W(V_2\bar C_1f^\xi + f^y))(t) {\,\,} (t \ge 0).
$$

Обозначив $ W(V_5 - Q) = \Theta $, получим
$$
((I - \Theta )\bar y)(t) = U(t)\bar y(0) + (WV_2(\bar H\bar
\xi(0)))(t) + (W(V_2\bar C_1f^\xi + f^y))(t) {\,\,} (t \ge 0).
$$

\textbf {Теорема 1.} {\it Пусть  $\bar U\bar y(0) + \bar WV_2(\bar
H\bar \xi(0))+ \bar W(V_2\bar C_1f^\xi + f^y) \in \bar M_q^\gamma $
для любых $\bar x(0) \in k^n_q$, $\varphi \in L^n_q$ и $||\bar U\bar
y(0) + \bar WV_2(\bar H \bar \xi(0))+ \bar W(V_2\bar C_1f^\xi +
f^y)||_{\bar M_q^\gamma}\le \bar c(||\bar x(0)||_{k^n_q} + ||\varphi
||_{L^n_q})$ для некоторого положительного числа $\bar c$, а
оператор $\Theta$ действует в пространстве $\bar M_q^\gamma $.
Тогда, если оператор $(I -\Theta ): \bar M_q^\gamma \rightarrow \bar
M_q^\gamma$ непрерывно обратим, то система \eqref{eq:kri-1} $M_q^\gamma
y$--устойчива.}

{\bf Доказательство.} Ввиду непрерывной обратимости оператора ${(I
-\Theta ): \bar M_q^\gamma \rightarrow \bar M_q^\gamma }$ уравнение
$(I - \Theta )\bar y = g$, где $g \in \bar M_q^\gamma $ имеет
единственное решение из $\bar M_q^\gamma $, т.е. $\bar y = (I -
\Theta)^{-1}g \in\bar M_q^\gamma $ и $\|\bar y \|_{\bar M_q^\gamma}
\le \|(I - \Theta )^{-1}\|_{\bar M_q^\gamma}\|g\|_{\bar
M_q^\gamma}$. Отсюда и из условий теоремы получим, что $(I -
\Theta)^{-1}(\bar U\bar y(0) + \bar WV_2(\bar H \bar \xi(0)) + \bar
W(V_2\bar C_1f^\xi + f^y)) \in \bar M_q^\gamma $ для любых $\bar
x(0) \in k^n_q$, $\varphi \in L^n_q$ и выполнено неравенство $\|(I -
\Theta)^{-1}(\bar U\bar y(0) + \bar WV_2(\bar H \bar \xi(0))+ \bar
W(V_2\bar C_1f^\xi + f^y))\|_{\bar M_q^\gamma}\le \bar c(\|\bar
x(0)\|_{k^n_q} + \|\varphi \|_{L^n_q})$ для некоторого
положительного числа $\bar c$. Но с другой стороны $\bar y(., \bar
x(0), \varphi) = (I - \Theta)^{-1}(\bar U \bar y(0) + \bar WV_2(\bar
H \bar \xi(0))) + \bar W(V_2\bar C_1f^\xi + f^y)$. Следовательно,
$\bar y(., \bar x(0), \varphi) \in \bar M_q^\gamma $ для любых $\bar
x(0) \in k^n_q$, $\varphi \in L^n_q$ и для него выполнено
неравенство \eqref{eq:kri-7}, а это и означает $M_q^\gamma y$--устойчивость
системы \eqref{eq:kri-1}.

Теорема доказана.

Теорему 1 можно использовать для получения достаточных условий
устойчивости системы \eqref{eq:kri-1} по части переменных в терминах параметров
этой системы, как это делается в классической версии $W$-метода.
Однако такие условия получаются более точными, если использовать
покомпонентные оценки решений. Ниже предлагается, поэтому,
улучшенный метод регуляризации для исследования вопросов
устойчивости по части переменных для системы \eqref{eq:kri-1}.

\textbf {Определение 4}. Обратимая матрица $\Phi =
(\phi_{ij})^m_{i,j=1}$ называется неотрицательно обратимой, если все
элементы матрицы $\Phi^{-1}$ неотрицательны.

Скак известно, матрица $\Phi$ неотрицательно обратима, если
$\phi_{ij} \leq 0$ при $i, j = 1,...,m$, $i\neq j$ и выполнено одно
из следующих условий:

- все диагональные миноры матрицы $\Phi$ положительны;

- существуют $\varsigma _i > 0$, $i = 1,...,m$ такие, что
$\varsigma_i \phi_{ii} > \sum \limits _{j=1}^m \varsigma_j
|\phi_{ij}|$, $i = 1,...,m$;

- существуют $\varsigma _i>0$, $i = 1,...,m$ такие, что $\varsigma_j
\phi_{jj} > \sum \limits _{i=1}^m\varsigma_i |\phi_{ij}|$, $j =
1,...,m$.

В частности, если положить $\varsigma _i = 1$, $i = 1,...,m$, то мы
получим класс матриц со строгим диагональным преобладанием и
неположительными внедиагональными элементами.

Для случайного процесса $\bar y(t) = col(\bar y^1(t),..., \bar
y^l(t))$ введем обозначение $ \bar y^\gamma (q) = col (\bar
y^{1\gamma }(q),...,$ $\bar y^{l\gamma }(q))$, где $\bar y^{i\gamma
}(q) = \sup \limits _{t \geq 0}\left (E|\gamma (t) \bar
y^i(t)|^{q}\right )^{1/q}$ при $i = 1,...,l$.

Пусть при некоторых $1\le q < \infty $ и положительной непрерывной
функции $\gamma :[0, \infty) \rightarrow R^1 $ для решения $\bar
y(t,\upsilon,\varphi) =  \bar y(t)$ системы \eqref{eq:kri-6} нам удалось
получить матричное неравенство следующего вида:
\begin{equation}
    \label{eq:kri-9}
    \bar E\bar y^\gamma (q) \leq C\bar y^\gamma (q) + \bar
    c\|\upsilon\|_{k^n_{q}} + \hat c \|\varphi \|_{L^n_q} ,
\end{equation}
где $C$ -- некоторая неотрицательная матрица размерности $l\times
l$, а $\bar c, \ \hat c$ -- некоторые $l$--мерные вектора--столбцы,
элементы которых неотрицательные числа. Тогда справедлива следующая
теорема.

\textbf {Теорема 2.} {\it Пусть в матричном неравенстве \eqref{eq:kri-9} матрица
$\bar E - C$ является неотрицательно обратимой. Тогда система \eqref{eq:kri-1}
$M_q^\gamma y$--устойчива.}

{\bf Доказательство.} Пользуясь положительной обратимостью матрицы
$\bar E - C$, перепишем неравенство \eqref{eq:kri-9} в следующем виде:
$$
\bar E\bar y^\gamma (q) \leq (\bar E - C)^{-1}\left(\bar
c\|\upsilon\|_{k^n_{q}} + \hat c \|\varphi \|_{L^n_q}  \right),
$$
откуда получим
\begin{equation}
    \label{eq:kri-10}
    |\bar y^\gamma (q)| \leq c\left (\|\upsilon\|_{k^n_{q}}+ \|\varphi
    \|_{L^n_q}\right ),
\end{equation}
где $c =\|(\bar E - C)^{-1}\|\max \{|\bar c|, |\hat c|\}$. Поскольку
$\bar y(t,\upsilon,\varphi) = \bar y(t)$  при $t \geq 0$ и $\|\bar
y(.,\upsilon,\varphi)\|_{\bar M_q^\gamma} \leq |\bar y^\gamma (q)|$,
то из неравенства \eqref{eq:kri-10} следует, что при любых $\upsilon \in k^n_q$,
$\varphi \in L^n_q$ для решений $\bar y(t, \upsilon, \varphi)$ $(t
\in (-\infty, \infty))$ задачи \eqref{eq:kri-3}, \eqref{eq:kri-3b} выполняются соотношение
$\bar y(., \upsilon, \varphi) \in \bar M_q^\gamma$ и неравенство
$$
\|\bar y(., \upsilon, \varphi)\|_{\bar M_q^\gamma} \le
c(\|b\|_{k^n_q} + \|\varphi \|_{L^n_q}),
$$
где $c$ -- некоторое положительное число. Следовательно,  система
\eqref{eq:kri-1} $M_q^\gamma y$--устойчива.

Теорема доказана.

На основе Теоремы 2 в следующем пункте будут получены достаточные
условия экспоненциальной моментной устойчивости системы \eqref{eq:kri-1} по части
компонент относительно начальных данных в терминах параметров этой
системы.
\smallskip

\textbf {3. Экспоненциальная устойчивость.} Как было отмечено в
пункте 2, из $M_q^\gamma y$--устойчи- вости системы \eqref{eq:kri-1} (где $\gamma
(t) = \exp \{\beta t\}$, $\beta$ --- некоторое положительное число)
следует экспоненциальная $q$--устойчивость этой же системы
относительно первых $l$ компонент. В пункте 2 также было указано,
что система \eqref{eq:kri-1} $M_q^\gamma y$--устойчива тогда и только тогда,
когда при любых $\bar x(0) = \upsilon \in k^n_q$, $\varphi \in
L^n_q$ решение уравнения \eqref{eq:kri-6} $\bar y(., \upsilon, \varphi)$
принадлежит пространству $\bar M_q^\gamma $ и для него выполнено
неравенство \eqref{eq:kri-7}.

В дальнейшем будем считать $\gamma (t) = \exp \{\beta t\}$, где
$\beta$ --- некоторое  нефиксированное положительное число, $q =
2p$, для  $f \in I^l(Z)$,
 $f^{si}, s = 1,...,l, i = 1,...,m$ --- элементы $f$.

Примеры показывают, что экспоненциальная устойчивость решений систем
линейных дифференциальных уравнений с последействием по начальным
данным наблюдается, как правило, только в случае ограниченного
последействия. Поэтому $M_q^\gamma y$--устойчивость системы \eqref{eq:kri-1}
будет изучена при допольнительных ограничениях  на параметры этой
системы.

Предположим, для системы \eqref{eq:kri-1} также выполнено:

- существуют неотрицательные числа $\tau_{ij}$, $i = 1,...,m$, $j =
1,...,m_i$ такие, что $0 \leq
 t- h_{ij}(t) \leq \tau _{ij}$, $i = 1,...,m$, $j = 1,...,m_i$ ($t
 \geq 0$) $\mu $--почти всюду;

 - существуют подмножества индексов $I_s \subset \{1,..., m_1\}$ ($s
= 1,..., l$), положительные числа  $ \lambda _s$, $s = 1, ..., l$ и
неотрицательные числа $\bar a_{si}^j$, $j = 1, ..., m_1$, $s,i = 1,
 ..., l$ такие, что $\sum \limits_{j\in I_s}a^{1j}_{ss}(t) \geq
 \lambda _s$, $s = 1,...,l$, $|a^{1j}_{si}(t)|\leq \bar a^j_{si}$, $j =
 1,...,m_1$, $s,i = 1, ..., l$, $t \geq 0$, $P\times\mu$--почти
всюду;

- существуют неотрицательные непрерывные функции $b_{si}^j(\beta)$,
$d_{s\nu}^j(\beta)$, $s, j = 1, ..., l$, $i = 1, ..., m$, $\nu = 2,
..., m$ такие, что для любых $u = col (u^1, ..., u^l) \in \bar D^l$
и $0<\beta < \min \{\lambda _s, s = 1, ..., l \}$ имеют  место
неравенства
 \begin{center}
 $ \mathrel {\mathop
 {vrai \sup} \limits _{\varsigma \geq 0}} \left(E\left |\gamma
(\varsigma)((V_2\bar C_1V_3 u)(\varsigma ))^{si}\right |^{2p}\right
)^{1/(2p)} \leq \sum \limits_{j=1}^lb_{si}^j(\beta)\sup \limits
_{\varsigma \geq 0}\left (E\left |\gamma
(\varsigma)u^j(\varsigma)\right |^{2p}\right )^{1/(2p)},$\\
$\mathrel {\mathop
 {vrai \sup}\limits _{\varsigma \geq 0}} \left(E\left |\gamma
(\varsigma)((V_1 u)(\varsigma ))^{s\nu}\right |^{2p}\right
)^{1/(2p)} \leq \sum \limits_{j=1}^ld_{s\nu}^j(\beta)\sup \limits
_{\varsigma \geq 0}\left (E\left |\gamma
(\varsigma)u^j(\varsigma)\right |^{2p}\right )^{1/(2p)}$
\end{center} при $s = 1,...,l, i = 1, ..., m , \nu = 2, ..., m$;

- существуют неотрицательные непрерывные функции $b_{si}(\beta)$,
$\bar b_{si}(\beta)$, $\hat b_{si}(\beta)$, $s= 1, ..., l$, $i= 1,
..., m$ такие,  что для любых $\varphi \in L^n_{2p}$, $\bar x(0) \in
k_{2p}^n$ и  $0<\beta < \min \{\lambda _s, s = 1, ..., l \}$
выполняются неравенства
 \begin{center} $\mathrel {\mathop
 {vrai \sup} \limits _{\varsigma \geq 0}} \left(E\left |\gamma (\varsigma )
((V_2 \bar C_1 f^h)(\varsigma))^{si}\right |^{2p}\right )^{1/(2p)}
\leq b_{si}(\beta) \|\varphi \|_{L^n_{2p}}$, \\
$\mathrel {\mathop
 {vrai \sup} \limits _{\varsigma \geq 0} }\left(E\left|\gamma (\varsigma )
(f^y(\varsigma ))^{si}\right |^{2p}\right )^{1/(2p)}\leq \bar
b_{si}(\beta) \|\varphi \|_{L^n_{2p}}$,\\
$\sup \limits _{\varsigma \geq 0}\left (E\left |\gamma (\varsigma)
((V_2(\bar H \bar h(0)))(\varsigma))^{si}\right |^{2p}\right
)^{1/(2p)} \leq \hat b_{si}(\beta)\|\bar x(0)\|_{k^n_{2p}}$
\end{center} при $s = 1,...,l, i = 1, ..., m$.

В дальнейшем нам также понадобятся следующие неравенства:\\
 \begin{equation}
    \label{eq:kri-11}
    \left (E\left |\int \limits _0^tf(\varsigma )d\mathcal B(\varsigma
    )\right |^{2p}\right )^{1/(2p)} \leq c_p \left (E\left (\int \limits
    _0^t|f(\varsigma )|^2d\varsigma\right )^p\right )^{1/(2p)},
 \end{equation}
\begin{equation}
    \label{eq:kri-12}
     \sup \limits _{t \geq 0}\left(E\left|\int \limits
     _0^tg(\varsigma)\hat f(\varsigma)d\varsigma\right|^{2p}\right)^{1/(2p)}
     \leq \sup \limits _{t \geq 0}\left (\int \limits
     _0^t|g(\varsigma)|d\varsigma\right )
     \sup \limits _{\varsigma \geq
     0}\left (E\left |\hat f(\varsigma)\right |^{2p}\right )^{1/(2p)},
\end{equation}
 \begin{equation}
\label{eq:kri-13}
\sup \limits _{t \geq 0}\left(E|\int \limits
 _0^tg(\varsigma)^2\hat f(\varsigma)^2d\varsigma|^{p}\right)^{1/(2p)} \leq
 \sup \limits _{t \geq 0}\left (\int \limits
 _0^tg(\varsigma)^2d\varsigma\right)^{1/2}\sup \limits _{\varsigma
 \geq 0}\left (E\left |\hat f(\varsigma)\right |^{2p}\right )^{1/(2p)},
\end{equation}
где $f(\varsigma )$ -- скалярный прогрессивно измеримый случайный
процесс, интегрируемый по винеровскому процессу  $\mathcal
B(\varsigma )$ на $[0, t]$, $c_p$ --- некоторое число, зависящее от
$p\ge 1$, но не зависящее от $f(\varsigma )$ и $t$, $g(\varsigma)$
-- скалярная функция на $[0,
 \infty)$, квадрат которой локально суммируем, а $\hat f(\varsigma)$ --
скалярный случайный процесс такой, что $\sup \limits _{\varsigma
\geq 0}\left (E\left |\hat f(\varsigma)\right |^{2p}\right
)^{1/(2p)} < \infty$.

 Условия устойчивости, приведённые ниже в основном результате этого
пункта (Теорема 3), формулированы в терминах $l\times l$-матрицы
$C$, элементы которой вычисляются следующим образом:
$$
\begin{array}{crl}
c_{ss} = \frac{1}{\lambda _s } \left [\sum \limits_{j \in I_s} \bar
a^{j}_{ss}\left (\tau _{1j}
 \left (\sum \limits_{i=1}^{m_1}\bar a^{i}_{ss} \gamma (\tau
 _{1i})
 + b^s_{s1}(0)\right ) +
c_p\sqrt{\tau _{1j}} \sum\limits_{i
=2}^{m}(d^s_{si}(0) + b^s_{si}(0))\right )\right .\\
\left . +\sum\limits_{j \in {1,...,m_1}/ I_s} \bar a^{j}_{ss} +
b^s_{s1}(0)\right ] + \frac{c_p}{\sqrt{2\lambda_s }}
\sum\limits_{i=2}^{m} (b^s_{si}(0) + d^s_{si} (0)), \ \  s =
1,...,l,
\end{array}
$$
$$
\begin{array}{crl}
c_{s\nu} = \frac{1}{\lambda _s } \left [\sum \limits_{j \in I_s}
\bar a^{j}_{ss}\left (\tau _{1j}
 \left (\sum \limits_{i=1}^{m_1}\bar a^{i}_{s\nu} + b^\nu_{s1}(0)\right ) +
c_p\sqrt{\tau _{1j}} \sum\limits_{i
=2}^{m}(d^\nu_{si}(0) + b^\nu_{si}(0))\right )\right .\\
\left . +\sum\limits_{j =1}^{m_1} \bar a^{j}_{s\nu} +
b^\nu_{s1}(0)\right ] + \frac{c_p}{\sqrt{2\lambda_s }}
\sum\limits_{i=2}^{m} (b^\nu_{si}(0) +d^\nu_{si} (0)), \ \ s,\nu =
1,...,l, \ s \neq \nu.
\end{array}
$$

Справедливо следующее утверждение.

\textbf {Теорема 3.} {\it Пусть для системы \eqref{eq:kri-1} выполнены все
предыдущие предположения. Тогда, если при этом матрица $\bar E - C$
является положительно обратимой, то система \eqref{eq:kri-1} будет $M_{2p}^\gamma
y$--устойчивой при $\gamma (t) = \exp \{\beta t\}$ для некоторого
$0<\beta < \min \{\lambda _s, s = 1, ..., l \}$.}

{\it Доказательство.}  Воспользуемся Теоремой 2, где $q = 2p$, а
$\gamma (t) = \exp \{\beta t\} \,\, (t \geq 0)$, $0< \beta < \min
\{\lambda _s, s = 1, ..., l \}$.

В качестве вспомогательного уравнения \eqref{eq:kri-8} возьмем систему
\begin{equation}
    \label{eq:kri-14}
    \begin {array}{crl}
     d \bar y(t) =  (- B(t)\bar  y(t) + g_1(t))dt+\sum
    \limits_{i=2}^{m}g_i(t)d\mathcal
    B_i(t) \, \, \, (t \ge 0),\\
    \end {array}
\end{equation}
где $B(t)$ -- диагональная матрица размерности $l\times l$ с
диагональными элементами $ \sum \limits_{j\in I_s}a^{j}_{ss}(t)$, $s
= 1,...,l$, a $g_1(t)$, $g_i(t)$ ($i=2,..,m$) -- $l$--мерные
прогрессивно измеримые случайные процессы на $[0, \infty )$ с п.н.
локально суммируемыми и локально суммируемыми с квадратом
траекториями соответственно. В этом случае для решения $\bar y(t)$
уравнения \eqref{eq:kri-14}  имеет место представление
$$
 \bar y(t) = U(t,0) \bar y(0) + \sum \limits_{i=1}^{m}\int \limits _0^t
 U(t,\varsigma)g(\varsigma)dZ(\varsigma) \,\, (t \geq 0),
$$
где  матрица $U(t, \varsigma), \, (t \ge 0, 0 \leq \varsigma \leq
t)$ --- диагональная матрица с диагональными элементами $ \hat y_s(t
,\varsigma) =\exp\left \{-\int \limits _\varsigma^t\sum
\limits_{j\in I_s}a^{j}_{ss}(\varsigma)d\varsigma \right \}, \, t
\ge 0, 0 \leq \varsigma \leq t, \, s = 1, ..., l$, $g$ --- $l\times
l$--матрица, столбцами которой является случайные процессы $g_i$, $i
= 1, ..., m$ соответственно. Тогда систему \eqref{eq:kri-6} можно записать в
следующем эквивалентном виде
\begin{equation}
    \label{eq:kri-15}
    \begin{array}{crl}
    \bar y^s(t) = \hat y_s(t,0 )\bar y^s(0) +\\
     \sum \limits_{j \in
    I_s}\int \limits _0^t \hat y_s(t,\varsigma) a^{1j}_{ss}(\varsigma
    )\int \limits _{h_{1j}(\varsigma )}^\varsigma d \bar y^s(\zeta) -
    \sum \limits_{j \in\{1,...,m_1\} / I_s} \int \limits _0^t \hat
    y_s(t,\varsigma) a^{1j}_{ss}(\varsigma ) \bar y^s(h_{1j}(\varsigma
    ))d\varsigma  - \\
    \sum \limits_{j=1}^{m_1}\sum \limits_{(\nu=1,\, \nu \neq s)}^{l}\int
    \limits _0^t \hat y_s(t,\varsigma) a^{1j}_{s\nu}(\varsigma ) \bar
    y^\nu(h_{1j}(\varsigma ))d\varsigma  +
     \sum \limits_{i=2}^{m} \int \limits _0^t
    \hat y_s(t,\varsigma)((V_1\bar y)(\varsigma))^{si}d\mathcal
    B_i(\varsigma) + \\
    \int \limits _0^t \hat y_s(t,\varsigma)(G(\varsigma))^{s1}
    d\varsigma + \sum \limits_{i=2}^{m} \int \limits _0^t \hat
    y_s(t,\varsigma) (G(\varsigma))^{si} d\mathcal B_i(\varsigma), s =
    1, ... , l,
    \end{array}
\end{equation}
где $(G(\varsigma))^{si}  = ((V_2\bar C_1V_3\bar y)(\varsigma))^{si}
+ ((V_2(\bar H \bar \xi(0)))(\varsigma))^{si}+ ((V_2 \bar C_1
f^\xi)(\varsigma))^{si} + (f^y(\varsigma))^{si}$, $s = 1, ... , l, i
= 1, ... , m$.

Учитывая обозначение $\bar y^{s\gamma} (2p)$, использованное в
Теореме 2, неравенства \eqref{eq:kri-11}--\eqref{eq:kri-13}, предположения относительно
системы \eqref{eq:kri-1}, равенства \eqref{eq:kri-15}, следующие равенства
$$
 \begin{array}{crl}
 d\bar y^s(\zeta) = -
 \sum \limits_{i=1}^{m_1}\sum
 \limits_{\nu=1}^l
a^{1i}_{s\nu}(\zeta)\bar y^k(h_{1i}(\zeta )d\zeta  + \sum
\limits_{i=2}^{m} ((V_1\bar y)(\zeta ))^{si}d\mathcal B_i(\zeta) +
(G(\varsigma))^{s1}d\zeta + \sum \limits_{i=2}^{m}
(G(\varsigma))^{si}d \mathcal B_i(\zeta), \\
s = 1, ... , l,
\end{array}
 $$
которые имеют место в силу системы \eqref{eq:kri-6}, а также для $\beta < \min
\{\lambda _s, s = 1, ...,n \}$ очевидные оценки
$(E|\upsilon_s|^{2p})^{1/(2p)} \leq \|\upsilon\|_{k^n_{2p}}$ при $s
= 1, ..., n$, $\mathrel {\mathop {\sup} \limits _{t\geq 0}}\int
\limits _0^t \hat y_s(t,\varsigma)\gamma (t)\gamma (\varsigma
)^{-1}d\varsigma \leq \frac{1}{\lambda _s -\beta }$, $\mathrel
{\mathop {\sup} \limits _{t\geq 0}} \left (\int \limits _0^t(\hat
y_s(t,\varsigma)\gamma (t)\gamma (\varsigma )^{-1})^2d\varsigma
\right )^{1/2} \leq \frac{1}{\sqrt{2(\lambda _s -\beta )}}$ при $s =
1,...,l$, $\gamma (t)\gamma (\bar h_{ij}(t))^{-1} \leq \gamma (\tau
_{ij}) \ (t \geq 0)$ при $i = 1, ..., m, \ j = 1, ..., m_i$,
$\mathrel {\mathop {\sup} \limits _{\varsigma \geq 0}}\left (\gamma
(\varsigma)\int \limits _{\bar h_{1j}(\varsigma )}^\varsigma
\gamma(\zeta )^{-1}d\zeta \right )\leq \gamma (\tau _{1j}) \tau
_{1j}$, $ \mathrel {\mathop {\sup} \limits _{\varsigma \geq 0}}
\left (\gamma (\varsigma)\left (\int \limits _{\bar h_{1j}(\varsigma
)}^\varsigma \gamma (\zeta)^{-2} d\zeta \right )^{1/2}\right )$
$\leq \gamma (\tau _{1j})\sqrt{\tau _{1j}}$ при $j = 1, ..., m_1$,
для решения системы \eqref{eq:kri-6} $\bar y(t,\upsilon,\varphi) = \bar y(t)$,
получаем
\begin{equation}
    \label{eq:kri-16}
    \begin{array}{crl}
    \bar y^{s\gamma}(2p) \leq  \|\upsilon\|_{k^n_{2p}} +
    \frac{1}{\lambda _s - \beta} \sum \limits_{j \in I_s} \bar
    a^{j}_{ss}
     \mathrel {\mathop {\sup}
    \limits _{\varsigma \geq 0}} \left (E\left |\gamma (\varsigma )\int
    \limits_{h_{1j}(\varsigma)}^\varsigma
    d\bar y^s(\zeta )\right |^{2p}\right )^{1/(2p)}   +  \\
    \frac{1}{\lambda _s - \beta}\sum \limits_{j \in\{1,...,m_1\} / I_s}
    \bar a^{j}_{ss} \gamma (\tau _{1j})\hat y^{s\gamma}(2p)+
    \frac{1}{\lambda _s -
    \beta}\sum\limits_{j=1}^{m_1}\sum\limits_{(\nu=1,\, \nu \neq s )}^l
    \bar a^{j}_{s\nu}\gamma (\tau _{1j})\hat y^{\nu\gamma}(2p)
    +\\
    c_p\sum \limits_{i=2}^{m}\mathrel {\mathop {\sup}\limits _{t\geq 0}}
    \left (E \left |\int \limits _0^t (\hat y_s(t,\varsigma)\gamma
    (t)\gamma (\varsigma )^{-1}\gamma (\varsigma)((V_1\bar
    y)(\varsigma))^{si})^2d \varsigma \right
    |^{p}\right )^{1/(2p)}  + \\
    \mathrel {\mathop {\sup} \limits _{t \geq 0}} \left (E\left |\int
    \limits _0^t \hat y_s(t,\varsigma)\gamma (t)\gamma (\varsigma
    )^{-1}\gamma
    (\varsigma)(G(\varsigma))^{s1}d\varsigma \right |^{2p}\right )^{1/(2p)} + \\
    \ c_p\sum \limits_{i=2}^{m}\mathrel {\mathop {\sup}\limits _{t \geq
    0}} \left (E \left |\int \limits _0^t (\hat y_s(t,\varsigma)\gamma
    (t)\gamma (\varsigma )^{-1}\gamma (\varsigma)(G(\varsigma))^{si})^2d
    \varsigma
    \right |^{p}\right )^{1/(2p)}   \leq  \\
    \|\upsilon \|_{k^n_{2p}} + \frac{1}{\lambda _s - \beta} \sum
    \limits_{j \in I_s} \bar a^{j}_{ss}
     \mathrel {\mathop {\sup}
    \limits _{\varsigma \geq 0}} \left (E\left |\gamma (\varsigma )\int
    \limits_{h_{1j}(\varsigma)}^\varsigma
    d\bar y^s(\zeta )\right |^{2p}\right )^{1/(2p)}   +  \\
    \frac{1}{\lambda _s - \beta}\sum \limits_{j \in\{1,...,m_1\} / I_s}
    \bar a^{j}_{ss} \gamma (\tau _{1j})\hat y^{s\gamma}(2p)+
    \frac{1}{\lambda _s - \beta}\sum\limits_{j=1}^{m_1}\sum\limits_{(\nu
    =1,\, \nu \neq s )}^l \bar a^{j}_{s\nu}\gamma (\tau _{1j})\hat
    y^{\nu\gamma}(2p)
    +\\
    \frac{1}{\lambda _s -\beta } \left [\sum\limits_{\nu =1}^{l} b^\nu
    _{s1}(\beta)\bar y^{s\gamma}(2p) +
     \hat b_{s1}(\beta)
    \|\upsilon\|_{k_{2p}} +( b_{s1}(\beta) +\bar b_{s1}(\beta)
    )\|\varphi \|_{L^n_{2p}}\right ]
    +\\
    \frac{c_p}{\sqrt{2(\lambda _s -\beta )}} \sum \limits_{i=2}^{m}
    \left [\sum \limits_{\nu=1}^{l}(b^\nu_{si}(\beta) +
    d^\nu_{si}(\beta))\bar y^{s\gamma}(2p) + \hat b_{si}(\beta)
    |\upsilon\|_{k_{2p}} +( b_{si}(\beta) +\bar b_{si}(\beta ))\|\varphi
    \|_{L^n_{2p}}\right ],
    \\
    s = 1, ... , l,
    \end{array}
\end{equation}
\begin{equation}
\label{eq:kri-17}
\begin{array}{crl}
\mathrel {\mathop {\sup} \limits _{\varsigma \geq 0}} \left (E\left
|\gamma (\varsigma )\int \limits_{h_{1j}(\varsigma)}^\varsigma d\bar
y^s(\zeta )\right |^{2p}\right )^{1/(2p)} \leq \sum
\limits_{i=1}^{m_1}\sum \limits_{\nu=1}^l \mathrel {\mathop {\sup}
\limits _{\varsigma \geq 0}}\left (E\left |\gamma (\varsigma)\int
\limits _{\bar h_{1j}(\varsigma )}^\varsigma
a^{1i}_{s\nu}(\zeta)\bar y^\nu(h_{1j}(\zeta )d\zeta\right |^{2p}
\right)^{1/(2p)} +  \\
c_p\sum \limits_{i=2}^{m} \mathrel {\mathop {\sup} \limits
_{\varsigma \geq 0}} \left (E\left (\int
\limits_{h_{1j}(\varsigma)}^\varsigma (\gamma (\varsigma )((V_1)\bar
y)(\zeta )^{si})^2d\zeta \right )^{p}\right )^{1/(2p)}+ \mathrel
{\mathop {\sup} \limits _{\varsigma \geq 0}} \left (E\left |\gamma
(\varsigma )\int \limits_{h_{1j}(\varsigma)}^\varsigma
(G(\zeta))^{s1}d\zeta\right |^{2p}\right )^{1/(2p)}+
\\
 c_p\sum \limits_{i=2}^{m} \mathrel {\mathop {\sup} \limits
_{\varsigma \geq 0}} \left (E\left (\int
\limits_{h_{1j}(\varsigma)}^\varsigma (\gamma (\varsigma
)(G(\zeta))^{si})^2d\zeta\right )^{p}\right )^{1/(2p)}
\leq \\
\left (\sum \limits_{\nu=1}^l\left [\sum \limits_{i=1}^{m_1}\gamma
(\tau _{1i})\bar a^{i}_{s\nu} + b^\nu_{s1}(\beta)\right ]\bar
y^{\nu\gamma }(2p) + \hat b_{s1}(\beta)\|\upsilon\|_{k^n_{2p}}+
(b_{s1}(\beta) +\bar b_{s1}(\beta))\|\varphi\|_{L^n_{2p}}\right )
\gamma (\tau _{1j}) \tau _{1j} + \\
c_p \left (\sum \limits_{i=2}^{m}\left [\sum \limits_{\nu=1}^l\left
(d^{\nu}_{si}(\beta) + b^\nu_{si}(\beta)\right )\bar y^{\nu\gamma
}(2p) + \hat b_{si}(\beta)\|\upsilon\|_{k^n_{2p}}+ (b_{si}(\beta)
+\bar b_{si}(\beta))\|\varphi\|_{L^n_{2p}}\right ]\right )
\gamma (\tau _{1j}) \sqrt{\tau _{1j}},\\
s =1, ..., l.
\end{array}
\end{equation}

Из неравенств \eqref{eq:kri-16} и \eqref{eq:kri-17} следует, что
\begin{equation}
\label{eq:kri-18}
\begin{array}{crl}
\bar y^{s\gamma}(2p) \leq  N_s(\beta)\|\upsilon\|_{k^n_{2p}} + \sum
\limits_{\nu=1}^l c_{s\nu}(\beta ) \bar y^{\nu\gamma}(2p) +
M_s(\beta)\|\varphi \|_{L_{2p}^n}, \ \  s = 1,...,l,
\end{array}
\end{equation}
где
$$
\begin{array}{crl}
N_s(\beta) = 1+ \frac{1}{\lambda _s - \beta} \sum \limits_{j \in
I_s} \bar a^{j}_{ss}\left (\gamma (\tau _{1j})\tau _{1j}\hat
b_{s1}(\beta) + c_p \gamma (\tau _{1j})\sqrt{\tau _{1j}}\sum
\limits_{i=1}^m \hat b_{si} (\beta)\right ) +
\frac{c_p}{\sqrt{2(\lambda_s -\beta)}} \sum\limits_{i=1}^{m} \hat
b_{si} (\beta),\\  s = 1,...,l,
\end{array}
$$
$$
\begin{array}{crl}
c_{ss}(\beta) = \frac{1}{\lambda _s - \beta} \left [\sum \limits_{j
\in I_s} \bar a^{j}_{ss}\left (\gamma (\tau _{1j})\tau _{1j}
 \left (\sum \limits_{i=1}^{m_1}\bar a^{i}_{ss} \gamma (\tau
 _{1i})
 + b^s_{s1}(\beta)\right ) +
c_p\gamma (\tau _{1j})\sqrt{\tau _{1j}} \sum\limits_{i
=1}^{m}(d^s_{si}(\beta) + b^s_{si}(\beta))\right )\right .\\
\left . +\sum\limits_{j \in {1,...,m_1}/ I_s} \bar a^{j}_{ss}\gamma
(\tau _{1j}) + b^s_{s1}(\beta)\right ] +
\frac{c_p}{\sqrt{2(\lambda_s -\beta)}} \sum\limits_{i=1}^{m}
(b^s_{si}(\beta) +d^s_{si} (\beta )), \ \  s = 1,...,l,
\end{array}
$$
$$
\begin{array}{crl}
c_{s\nu}(\lambda) = \frac{1}{\lambda _s - \beta} \left [\sum
\limits_{j \in I_s} \bar a^{j}_{ss}\left (\gamma (\tau _{1j})\tau
_{1j}
 \left (\sum \limits_{i=1}^{m_1}\bar a^{i}_{s\nu} \gamma (\tau
 _{1i})
 + b^\nu_{s1}(\beta)\right ) +
c_p\gamma (\tau _{1j})\sqrt{\tau _{1j}} \sum\limits_{i
=1}^{m}(d^\nu_{si}(\beta) + b^\nu_{si}(\beta))\right )\right .\\
\left . +\sum\limits_{j =1}^{m_1} \bar a^{j}_{s\nu}\gamma (\tau
_{1j}) + b^\nu_{s1}(\beta)\right ] + \frac{c_p}{\sqrt{2(\lambda_s
-\beta)}} \sum\limits_{i=1}^{m} (b^\nu_{si}(\beta) +d^\nu_{si}
(\beta )), \ \ s,\nu = 1,...,l, \ s \neq \nu,
\end{array}
$$
$$
\begin{array}{crl}
M_s(\beta )= \frac{1}{\lambda _s - \beta} \sum \limits_{j \in I_s}
\bar a^{j}_{ss}\left [\gamma (\tau _{1j})\tau _{1j}(b_{s1}(\beta ) +
\bar b_{s1}(\beta )) +
 c_p\gamma (\tau _{1j})\sqrt{\tau _{1j}}\sum\limits_{i =1}^{m}(b_{si}(\beta ) + \bar b_{si}(\beta
 ))\right ] + \\
 \frac{1}{\lambda _s - \beta}(b_{s1}(\beta ) + \bar b_{s1}(\beta
 ))+ \frac{c_p}{\sqrt{2(\lambda_s -\beta)}} \sum\limits_{i=1}^{m}(b_{si}(\beta ) + \bar
 b_{si}(\beta), s = 1,...,l.
\end{array}
$$
Систему \eqref{eq:kri-18} запишем в матричной форме
$$
\bar E\bar y^\gamma(2p) \leq  C(\beta )\bar y^\gamma(2p)+ \bar
c(\beta ) \|\upsilon\|_{k^n_{2p}} + \hat c(\beta )\|\varphi
\|_{L_{2p}^n},
$$
где $C(\beta) =(c_{ij}(\beta))_{i,j=1}^l$
--- $l\times l$--матрица, а $\bar c(\beta) = col (N_1(\beta ), ..., M_l(\beta )),
\hat c(\beta ) = col (M_1(\beta ), ..., $ $M_l(\beta ))$ ---
$l$--мерные вектора столбцы.

Очевидно, что $C(0) = C$, где $C$ -- матрица из условия теоремы.
Поскольку $\bar E - C$ является  положительно обратимой матрицей, а
это свойство устойчиво относительно малых возмущений, то из этого
следует, что при достаточно малых $\beta > 0$ матрица $\bar E-
C(\beta )$ также будет неотрицательно обратимой. Следовательно, в
силу Теоремы 2, система \eqref{eq:kri-1} будет $M_{2p}^\gamma$--устойчивой для
некоторого $0< \beta < \min \{\lambda _s, \ s = 1, ...,l \}$.

Теорема доказана.
\smallskip

\textbf {\bf 4. Достаточные условия устойчивости.}  В дальнейшем
будем пользоваться обозначениями предыдущих пунктов и на основе
Теоремы 3 будут получены достаточные условия устойчивости по части
переменных решений для некоторых классов систем вида \eqref{eq:kri-1} в терминах
их параметров.

В дальнейшем будем считать $\gamma (t) = \exp \{\beta t\}$, где
$\beta$ --- некоторое  нефиксированное положительное число.

Пусть в системе \eqref{eq:kri-1} элементы матриц $A_{1j}, j = 2, ..., m_1$,
$A_{ij}, i = 2, ..., m, j = 1, ..., m_i$ равны нулю
$P\times\mu$--почти всюду, а элементы матрицы $A_{11}$ являются
локально суммируемыми функциями, $h_{11}(t) = t \, (t \geq 0)$ $\mu
$--почти всюду. Тогда система \eqref{eq:kri-1} является системой линейных
обыкновенных дифференциальных уравнений. В этом случае условие \eqref{eq:kri-1a}
лишнее и в условии \eqref{eq:kri-1b} $\upsilon_i, i = 1, ..., n$ ---
действительные числа. Кроме того, $f^y \equiv 0, f^\xi \equiv 0$.


Рассмотрим случай  $l = n-1$. Тогда $y = col(x^1,...,x^{n-1}), \xi =
x^n$, $(A_{11}(t))^1$ есть $(n-1)\times (n-1)$--матрица, полученная
из матрицы $A_{11}(t)$ зачеркиванием последней строки и последнего
столбца, $(A_{11}(t))^2 = col(a^{11}_{1n}(t),...,a^{11}_{n-1
n}(t))$, $(A_{11}(t))^3 = (a^{11}_{n1}(t),...,a^{11}_{n-1n-1}(t))$,
$(A_{11}(t))^4 = a^{11}_{nn}(t))$, $H(t) = \exp \left \{-\int
\limits _{0}^t a^{11}_{nn}(\zeta)d\zeta \right \}$, $(C_1g)(t) =
\int \limits _{0}^tH(t)(H(\zeta))^{-1}g(\zeta)d\zeta$.

Пусть:

- существуют положительные числа  $ \lambda _s$, $s = 1, ..., l$ и
неотрицательные числа $\bar a_{si}^1$, $s,i = 1,
 ..., l$ такие, что $a^{11}_{ss}(t) \geq
 \lambda _s$, $s = 1,...,l$, $|a^{11}_{si}(t)|\leq \bar a^1_{si}$, $s,i = 1, ..., l$, $t \geq 0$, $\mu$--почти
всюду;

- существуют неотрицательные непрерывные функции $b_{s1}^j(\beta)$,
$s, j = 1, ..., l$ такие, что для любой непрерывной функции
$u(\varsigma ) = col (u^1 (\varsigma), ..., u^l(\varsigma))
(\varsigma\geq 0)$ и любого $\beta$, $0 < \beta < \min \{\lambda _s,
s = 1,
..., l \}$ имеют  место неравенства \\
 \begin{center}
 $\mathrel {\mathop
 {vrai \sup}\limits _{\varsigma \geq 0}} \left |\gamma (\varsigma )a^{11}_{sn}(\varsigma )
\int \limits _{0}^\varsigma H(\varsigma)(H(\zeta))^{-1}\sum \limits
_{j=1}^l a_{nj}^{11}(\zeta)u^j(\zeta )d\zeta \right | \leq \sum
\limits_{j=1}^lb_{s1}^j(\beta)\sup \limits _{\varsigma \geq 0}\left
|\gamma (\varsigma)u^j(\varsigma)\right |, s = 1,...,l;$
\end{center}

- существуют неотрицательные непрерывные функции  $\hat
b_{s1}(\beta)$, $s= 1, ..., l$ такие,  что для любых  $ x (0) = (x^1
(0), ..., x^n(0)) \in R^n$ и $0 < \beta < \min \{\lambda _s, s = 1,
..., l \}$ выполняются неравенства \\
\begin{center}
$\mathrel {\mathop
 {vrai \sup}\limits _{\varsigma \geq 0}} \left |\gamma (\varsigma )a^{11}_{sn}(\varsigma )
H(\varsigma)x^n(0)\right | \leq \hat b_{s1}(\beta)|x(0)|, s =
1,...,l;$
\end{center}

- элементы $l\times l$-матрицы $C$ вычисляются следующим образом:
$$
\begin{array}{crl}
c_{ss} = \frac{b^s_{s1}(0)}{\lambda _s }, \ \  s = 1,...,l, c_{s\nu}
= \frac{\bar a^{1}_{s\nu} + b^\nu_{s1}(0)}{\lambda _s }, \ \ s,\nu =
1,...,l, \ s \neq \nu.
\end{array}
$$

В силу Теоремы 3 справедливо

\textbf {Утверждение 1.} {\it Пусть для системы \eqref{eq:kri-1} выполнены все
предыдущие предположения. Тогда, если при этом матрица $\bar E - C$
является положительно обратимой, то система \eqref{eq:kri-1} будет
экспоненциально устойчивой относительно первых $(n-1)$ компонент.}

Пусть в системе \eqref{eq:kri-1} элементы матриц $A_{ij}, i = 2, ..., m, j = 1,
..., m_i$ равны нулю $P\times\mu$--почти всюду, а элементами матриц
$A_{1j}, j = 1, ..., m_1$ являются локально суммируемые функции.
Тогда система \eqref{eq:kri-1} является системой линейных обыкновенных
дифференциальных уравнений с запаздываниями.

Рассмотрим случай  $l = n-1$. Тогда $y = col(x^1,...,x^{n-1}), \xi =
x^n$, $(A_{1j}(t))^1$ является $(n-1)\times (n-1)$--матрица,
полученная из матрицы $A_{1j}(t)$ зачеркиванием последней строки и
последнего столбца , $(A_{1j}(t))^2 =
col(a^{1j}_{1n}(t),...,a^{1j}_{n-1 n}(t))$, $(A_{1j}(t))^3 =
(a^{1j}_{n1}(t),...,a^{1j}_{n-1n-1}(t))$, $(A_{1j}(t))^4 =
a^{1j}_{nn}(t))$ при $j = 1,...,m_1 $. Предположим, что $h_{11}(t) =
t \, (t \geq 0)$ $\mu $--почти всюду и элементы матриц
$A_{1j}(t))^2, (A_{1j}(t))^3, (A_{1j}(t))^4 (t \geq 0)$ равны нулю
$\mu $--почти всюду при $j = 2,...,m_1 $. Тогда $H(t) = \exp \left
\{-\int \limits _{0}^t a^{11}_{nn}(\zeta)d\zeta \right \}$,
$(C_1g)(t) = \int \limits _{0}^tH(t)(H(\zeta))^{-1}g(\zeta)d\zeta$.

Пусть:

- существуют неотрицательные числа $\tau_{1j}$, $j = 2,...,m_1$
такие, что $0 \leq
 t- h_{1j}(t) \leq \tau _{1j}$,  $j = 2,...,m_1$ ($t
 \geq 0$) $\mu $--почти всюду;

 - существуют подмножества индексов $I_s \subset \{1,..., m_1\}$ ($s
= 1,..., l$), положительные числа  $ \lambda _s$, $s = 1, ..., l$ и
неотрицательные числа $\bar a_{si}^j$, $j = 1, ..., m_1$, $s,i = 1,
 ..., l$ такие, что $\sum \limits_{j\in I_s}a^{1j}_{ss}(t) \geq
 \lambda _s$, $s = 1,...,l$, $|a^{1j}_{si}(t)|\leq \bar a^j_{si}$, $j =
 1,...,m_1$, $s,i = 1, ..., l$, $t \geq 0$, $P\times\mu$--почти
всюду;

- существуют неотрицательные непрерывные функции $b_{s1}^j(\beta)$,
$s, j = 1, ..., l$ такие, что для любой непрерывной функции
$u(\varsigma ) = col (u^1 (\varsigma), ..., u^l(\varsigma))
(\varsigma\geq 0)$ и любого $\beta$, $0 < \beta < \min \{\lambda _s,
s = 1,
..., l \}$ имеют  место неравенства\\
 \begin{center}
 $\mathrel {\mathop
 {vrai \sup}\limits _{\varsigma \geq 0}} \left |\gamma (\varsigma )a^{11}_{sn}(\varsigma )
\int \limits _{0}^\varsigma H(\varsigma)(H(\zeta))^{-1}\sum \limits
_{j=1}^l a_{nj}^{11}(\zeta)u^j(\zeta )d\zeta \right | \leq \sum
\limits_{j=1}^lb_{s1}^j(\beta)\sup \limits _{\varsigma \geq 0}\left
|\gamma (\varsigma)u^j(\varsigma)\right |, s = 1,...,l;$
\end{center}

- существуют неотрицательные непрерывные функции  $\hat
b_{s1}(\beta)$, $s= 1, ..., l$ такие,  что для любых  $ x (0) = (x^1
(0), ..., x^n(0)) \in R^n$  и $0 < \beta < \min \{\lambda _s, s = 1,
..., l \}$выполняются неравенства\\
\begin{center}
$\mathrel {\mathop
 {vrai \sup}\limits _{\varsigma \geq 0}} \left |\gamma (\varsigma )a^{11}_{sn}(\varsigma )
H(\varsigma)x^n(0)\right | \leq \hat b_{s1}(\beta)|x(0)|, s =
1,...,l;$
\end{center}

- элементы $l\times l$-матрицы $C$ вычисляются следующим образом:
$$
\begin{array}{crl}
c_{ss} = \frac{1}{\lambda _s } \left [\sum \limits_{j \in I_s} \bar
a^{j}_{ss}\tau _{1j}
 \left (\sum \limits_{i=1}^{m_1}\bar a^{i}_{ss} \gamma (\tau
 _{1i})
 + b^s_{s1}(0)\right )
 +\sum\limits_{j \in {1,...,m_1}/ I_s} \bar a^{j}_{ss} +
b^s_{s1}(0)\right ], \ \  s = 1,...,l,
\end{array}
$$
$$
\begin{array}{crl}
c_{s\nu} = \frac{1}{\lambda _s } \left [\sum \limits_{j \in I_s}
\bar a^{j}_{ss}\tau _{1j}
 \left (\sum \limits_{i=1}^{m_1}\bar a^{i}_{s\nu} + b^\nu_{s1}(0)\right )
+\sum\limits_{j =1}^{m_1} \bar a^{j}_{s\nu} + b^\nu_{s1}(0)\right ],
\ \ s,\nu = 1,...,l, \ s \neq \mu,
\end{array}
$$
где $\tau _{11} = 0$


\textbf {Утверждение 2.} {\it Пусть для системы \eqref{eq:kri-1} выполнены все
предыдущие предположения. Тогда, если при этом матрица $\bar E - C$
является положительно обратимой, то система \eqref{eq:kri-1} будет
экспоненциально устойчивой относительно первых $(n-1)$ компонент.}

Справедливость Утверждения 2 также следует из Теоремы 3. При
проверке выполнения всех условий Теоремы 3  надо учесть, что $f^\xi
\equiv 0$, а выполнение условий для $f^y $ можно убедиться
непосредственно, проверкой.

Пусть в системе \eqref{eq:kri-1} элементы матриц $A_{ij}, i = 1, ..., m, j = 2,
..., m_i$ равны нулю $P\times\mu$--почти всюду,   $h_{i1}(t) = t, i
= 1, ..., m \, (t \geq 0)$ $\mu $--почти всюду. Тогда система \eqref{eq:kri-1}
является системой линейных обыкновенных дифференциальных уравнений
Ито. В этом случае условие \eqref{eq:kri-1a} лишнее. Кроме того, $f^y \equiv 0,
f^\xi \equiv 0$.

Рассмотрим случай  $l = n-1$. Тогда $y = col(x^1,...,x^{n-1}), \xi =
x^n$, $(A_{i1}(t))^1$ является $(n-1)\times (n-1)$--матрица,
полученная из матрицы $A_{i1}(t)$ зачеркиванием последней строки и
последнего столбца , $(A_{i1}(t))^2 =
col(a^{i1}_{1n}(t),...,a^{i1}_{n-1 n}(t))$, $(A_{i1}(t))^3 =
(a^{i1}_{n1}(t),...,a^{i1}_{n-1n-1}(t))$, $(A_{i1}(t))^4 =
a^{i1}_{nn}(t))$ при $i = 1,...,m $. Тогда $H(t) = \exp \left
\{-\int \limits _{0}^t [a^{11}_{nn}(\zeta) + (1/2)\sum \limits
_{i=2}^m (a^{i1}_{nn}(\zeta))^2]d\zeta  + \right .$ $\left . \sum
\limits_{i=2}^m \int \limits
 _0^t a^{i1}_{nn}(\zeta)d\mathcal B_i(\zeta)\right \}$, $(C_1g)(t) = \int \limits
_{0}^tH(t)(H(\zeta))^{-1}g(\zeta)dZ(\zeta)$.

Пусть:

- существуют положительные числа  $ \lambda _s$, $s = 1, ..., l$ и
неотрицательные числа $\bar a_{si}^1$, $s,i = 1,
 ..., l$ такие, что $a^{11}_{ss}(t) \geq
 \lambda _s$, $s = 1,...,l$, $|a^{11}_{si}(t)|\leq \bar a^1_{si}$, $s,i = 1, ..., l$, $t \geq 0$, $P\times\mu$--почти
всюду;

- существуют неотрицательные непрерывные функции $b_{si}^j(\beta)$,
$d_{s\nu}^j(\beta)$, $s, j = 1, ..., l$, $i = 1, ..., m$, $\nu = 2,
..., m$ такие, что для любых $u = col (u^1, ..., u^l) \in \bar D^l$
и $0 < \beta < \min \{\lambda _s, s = 1,
..., l \}$ имеют  место неравенства\\
 \begin{center}$ \mathrel {\mathop
 {vrai \sup} \limits _{\varsigma \geq 0}} \left(E\left |\gamma
(\varsigma)\gamma (\varsigma )a^{i1}_{sn}(\varsigma ) \int \limits
_{0}^\varsigma H(\varsigma)(H(\zeta))^{-1}(\sum \limits _{j=1}^l
a_{nj}^{11}(\zeta)u^j(\zeta ),...,\sum \limits _{j=1}^l
a_{nj}^{m1}(\zeta)u^j(\zeta ))dZ(\zeta )\right |^{2p}\right
)^{1/(2p)} \leq \sum \limits_{j=1}^lb_{si}^j(\beta)\sup \limits
_{\varsigma \geq 0}\left (E\left |\gamma
(\varsigma)u^j(\varsigma)\right |^{2p}\right )^{1/(2p)},$\\
$\mathrel {\mathop {vrai \sup}\limits _{\varsigma \geq 0}}
\left(E\left |\gamma (\varsigma)\sum \limits _{j=1}^l
a_{sj}^{\nu1}(\zeta)u^j(\zeta )\right |^{2p}\right )^{1/(2p)} \leq
\sum \limits_{j=1}^ld_{s\nu}^j(\beta)\sup \limits _{\varsigma \geq
0}\left (E\left |\gamma (\varsigma)u^j(\varsigma)\right |^{2p}\right
)^{1/(2p)}$ \end{center} при $s = 1,...,l, i = 1, ..., m , \nu =2,
..., m$;

- существуют неотрицательные непрерывные функции $\hat
b_{si}(\beta)$, $s= 1, ..., l$, $i= 1, ..., m$ такие,  что для любых
и $0 < \beta < \min \{\lambda _s, s = 1, ..., l \}$,  $\bar x(0) \in
k_{2p}^n$ выполняются неравенства\\
\begin{center} $\sup \limits _{\varsigma \geq 0}\left (E\left |\gamma (\varsigma)
a^{i1}_{sn}(\varsigma)H(\varsigma) \bar h(0)\right |^{2p}\right
)^{1/(2p)} \leq \hat b_{si}(\beta)\|\bar x(0)\|_{k^n_{2p}}$
\end{center} при $s = 1,...,l, i = 1, ..., m$;

- элементы $l\times l$-матрицы $C$ вычисляются следующим образом:
$$
\begin{array}{crl}
c_{ss} = \frac{b^s_{s1}(0)}{\lambda _s }  +
\frac{c_p}{\sqrt{2\lambda_s }} \sum\limits_{i=1}^{m} (b^s_{si}(0) +
d^s_{si} (0)), \ \  s = 1,...,l,
\end{array}
$$
$$
\begin{array}{crl}
c_{s\nu} = \frac{1}{\lambda _s } \left [\sum\limits_{j =1}^{m_1}
\bar a^{j}_{s\nu} + b^\nu_{s1}(0)\right ] +
\frac{c_p}{\sqrt{2\lambda_s }} \sum\limits_{i=1}^{m} (b^\nu_{si}(0)
+d^\nu_{si} (0)), \ \ s,\nu = 1,...,l, \ s \neq \nu.
\end{array}
$$

\textbf {Утверждение 3.} {\it Пусть для системы \eqref{eq:kri-1} выполнены все
предыдущие предположения. Тогда, если при этом матрица $\bar E - C$
является неотрицательно обратимой, то система \eqref{eq:kri-1} будет
экспоненциально $2p$--устойчивой относительно первых $(n-1)$
компонент.}

Справедливость Утверждения 3 также следует из Теоремы 3.


\begin{center}{СПИСОК ЛИТЕРАТУРЫ}\end{center}{\small

1. Р.И. Кадиев, З.И. Шахбанова К вопросу об устойчивости по части
компонент решений линейных непрерывно-дискретных систем Ито с
последействием // Вестник ДГУ, Серия 1. Естественные науки. 2023.
Том 38. Вып. 3. С. 40-18.

2.  Кадиев Р.И. Устойчивость по части переменных систем линейных
дифференциальных уравнений Ито с последействием //Дифференц.
уравнения. Минск. Т.59, № 10, 2023. С.1318-1334.

3. A. Ponosov, L. Idels, R. Kadiev. A novel algorithm for asymptotic
stability analysis of some classes of stochastic time-fractional
Volterra equations. Commun. Nonlinear Sci. and Numer. Simul., 126
(2023), Art ID 107491.

4. Кадиев Р.И., Поносов А.В. Асимптотическая моментная устойчивость
решений систем нелинейных дифференциальных уравнений Ито с
последействием //Известие вузов. Математика (решением заседания
редколлегии от 26.09.2023 г. принята к печати и будет опубликована в
порядке очереди (см. на сайте журнала раздел "Приняты к печати").

Тезисы в трудах конференций

1. Кадиев Р.И., Поносов А.В. К вопросу существования и
единственности решений задачи Коши для
функционально-дифференциальных уравнений Ито дробного порядка
//Материалы IV Всероссийской конференции "Актуальные проблемы
математики и информационных технологий "с международным участием (г.
Махачкала, 7-9 февраля 2023 г. С. 69-72).

2. Кадиев Р.И. К вопросу моментной устойчивости систем линейных
уравнений Ито с  дробным временем //В сборнике: АКТУАЛЬНЫЕ ПРОБЛЕМЫ
МАТЕМАТИКИ И ИНФОРМАЦИОННЫХ ТЕХНОЛОГИЙ. Материалы IV Всероссийской
конференции с международным участием. 2023. С. 78-81.

3. L. Idels, R. Kadiev., A. Ponosov Asymptotic stability of
time-fractional stochastic Volterra equations // (Тезысы в Грузии,
выдут в декабря)



% \end{document}

% %\end{document}
