\chapter{Интегральные преобразования векторных и тензорных полей, заданных на семействах прямых и ломаных}

\section{Введение}
Решаются задачи восстановления векторных и тензорных полей в трехмерном пространстве и на плоскости по заданным интегральным преобразованиям двух типов. Первое преобразование можно охарактеризовать как множество интегралов от тензорных полей, взятых по лучам, касающимся данной поверхности. Это одно из трех преобразований на трехмерных комплексах лучей, перечисленных в работе \cite{Medzhidov}. Приведенная здесь теорема обобщает теорему, доказанную в этой работе на случай тензорных полей.

Второе преобразование, называемое еще V-лучевым преобразованием, задано на семействах ломаных на плоскости. Идею V-образных преобразований Радона с фиксированным направлением оси симметрии выдвинули Трунг Т. и Нгуен М. К. \cite{Truong}. Такие преобразования могут представлять теоретический интерес в интегральной геометрии. Они возникают в результате связанного томографического процесса пропускания-отражения. Интегральные преобразования по преломленным лучам изучаются и во многих других работах. Из недавних работ можно отметить статьи \cite{Sharafutdinov}, \cite{Ambartsoumian}  и др.
Мы рассматриваем обобщения указанных преобразований на векторные и тензорные поля на плоскости.

\section{Обращение тензорных полей по интегральным данным на семействе лучей, касающихся поверхности}

Пусть $f=\left(f_1, f_2, f_3\right)$ -- гладкое (класса $C^2$) векторное поле с компактным носителем в трехмерном евклидовом пространстве $\mathbb R^3$, $\theta$-единичный вектор, $x\in \mathbb R^3$,  $l=\{x+t\theta, t\geq 0\}$ - ориентированная прямая Лучевым преобразованием поля $f$ называется функция
$$If(l)=If(x,\theta)=\int_0^{+\infty}f_k(x+t\theta)\theta^kdt;$$
по повторяющемуся индексу производится суммирование от 1 до 3.

В следующей теореме, доказанной в работе \cite{Medzhidov}, решается задача реконструкции поля  $f$ по данным интегралам   на трехмерном многообразии (комплексе) прямых в  $\mathbb R^3$, касательных данной поверхности $R$. Однозначно можно восстановить только соленоидальную часть поля, что  равносильно нахождению оператора Сен-Венана $Wf$.

\begin{theorem}\label{th:mzg-1} 
Пусть $R$ -- гладкая поверхность в ориентированном евклидовом пространстве $\mathbb R^3$, $H$ -- плоскость, трансверсальная $\Sigma$. Предположим, что кривая $C=R\cap H$ имеет гладкую натуральную параметризацию $y=y(s), s\in [0,S],  |y'(s)|=1$, и кривизну, отличную от нуля в каждой точке $y(s)$. Тогда для любого поля $f$, такого, что  
$supp f\cap H$ содержится в области значений отображения $[0,S]\times [0;+\infty)\to H$, поле $Wf$ может быть восстановлена по заданному полю $If(y(s), y'(s))$ и его первым производным для лучей $l=l((y(s), y'(s))$, где $s\in C$.
\end{theorem}

Для решения аналогичной задачи для тензорных полей второго ранга мы используем дополнительные интегральные данные. 

Пусть $f$ -- гладкое (класса $C^2$) симметричное тензорное поле с компактным носителем в трехмерном евклидовом пространстве $\mathbb R^3$. Его лучевое преобразование и момент первого порядка определяются соответственно формулами
\begin{align*}
    If(x,\theta)&=\int_0^{+\infty}f_{kl}(x+t\theta)\theta^k\theta^ldt;
    \\
    I^1f(x,\theta)&=\int_0^{+\infty}f_{kl}(x+t\theta)\theta^k\theta^l tdt.
\end{align*}

\begin{theorem} Пусть $R, H, C$ те же, что в теореме~\ref{th:mzg-1}. Тогда для любого поля $f$, такого, что  
$supp f\cap H$ содержится в области значений отображения $[0,S]\times [0;+\infty)\to H$, поле $Wf$ может быть восстановлена по заданным полям $If(y(s), y'(s)),  I^1f(y(s), y'(s))$ и их первым производным для лучей $l=l((y(s), y'(s))$, где $s\in C$.
\end{theorem}

\section{Обращение векторных полей по интегральным данным на семействах ломаных на плоскости} 

Пусть $ \omega=(\omega_1,\omega_2 )$, $\theta=(\theta_1,\theta_2) $  -- линейно независимые единичные векторы на плоскости, $x=(x_1,x_2 )\in \mathbb R^2$ -- произвольная точка. Символом $\Gamma_{\omega,\theta}(x) $ обозначим объединение двух лучей с вершиной в точке x и направляющими векторами ω и θ соответственно:
$\Gamma_{\omega,\theta}(x)= \Gamma_{\omega}(x)\cup \Gamma_\theta^-(x)$, $\Gamma_{\omega}(x)={\xi=(\xi_1, \xi_2)\in\mathbb R^2: \xi=x+\omega t}$, $t\geq 0$; знак <<-->> в обозначении $\Gamma_\theta^-(x)$ означает, что луч $\Gamma_\theta(x)$ проходится в направлении, противоположном направлению вектора θ.

Лучевым преобразованием функции $h(x)$ называется функция

$$Xh(x;\omega)=\int_{\Gamma_\omega (x)} h(y)ds=\int_0^\infty h(x+t\omega)dt.$$
V-преобразованием Радона функции $h$ называется функция

$$Vh(x;\omega,\theta)=Xh(x;\omega)-Xh(x;\theta)=\int_{\Gamma_{\omega,\theta}(x)} h(y)ds.$$

Пусть $f=(f_1,f_2 )$ -- векторное поле на плоскости. $V$-лучевым преобразованием поля $f$ называется функция

$$If(x;\omega,\theta)=X(f\cdot\omega)(x;\omega)-X(f\cdot\theta)(x;\theta),$$
где $f\cdot\omega=f_1 \omega_1+f_2 \omega_2$ -- скалярное произведение. Здесь и далее предполагается, что компоненты $f_1$ и $f_2$ векторного поля принадлежат классу $C_0^2 (\mathbb R^2 )$ дважды непрерывно дифференцируемых финитных функций.


Задача состоит в том, чтобы по заданному преобразованию $If$ определить неизвестное поле $f$. Однозначно определить поле $f$ по заданной функции $If$ невозможно, так как $I(\nabla h)=0$, где 

$$\nabla h=\left(\frac{\partial h}{\partial x_1},\frac{\partial h}{\partial x_2}\right),$$
т.е. $f$ можно найти с точностью до потенциального слагаемого. Требуются дополнительные интегральные данные.

Определим также преобразование
$$Jf(\omega,\theta)=X(f\cdot \omega)(x;\omega^\bot )-X(f\cdot\theta)(x;\theta^\bot),$$
называемое поперечным V-лучевым преобразованием поля $f$.

Пусть $\partial_\theta f=(\theta,\nabla f)$ -- производная скалярного поля $f$ по направлению вектора $\theta$, 
$$\delta f=\frac{\partial f_1}{\partial x_1}+\frac{\partial f_2}{\partial x_2},\,\, \delta^\bot f=\frac{\partial f_2}{\partial x_1}-\frac{\partial f_1}{\partial x_2}$$
--дивергенция и ортогональная дивергенция поля $f$, соответственно.


\begin{theorem} Ядро оператора $I$ состоит из потенциальных векторных полей, т.е.
$$If=0 \Leftrightarrow f=\nabla h$$ 
для некоторой функции $h$.
Ядро оператора $J$ состоит из соленоидальных векторных полей, т.е.
$$Jf=0  \Leftrightarrow  \delta f=0.$$
\end{theorem}

\begin{theorem} Векторное поле $f$ однозначно восстанавливается по заданным ее V-лучевому и поперечному V-лучевому преобразованиям.
\end{theorem}

Для полного восстановления поля $f$ введем интегральные данные другого вида.
Интегральное преобразование 
$$I^1 f(x;\omega,\theta)=X^1 (f\cdot\omega)(x;\omega)-X^1 (f\cdot\theta)(x;\theta)$$
назовем моментом первого порядка V-лучевого преобразования функции поля $f$, где
$$X^1 h(x;\omega)=\int_0^\infty h(x+t\omega )tdt$$
-- момент первого порядка лучевого преобразования функции $h$.

\begin{theorem} Векторное поле $f$ с компонентами из класса $C_0^2 (\mathbb R^2 )$ дважды непрерывно дифференцируемых финитных функций однозначно восстанавливается по заданным ее V\-/лучевому преобразованию $If$ и моменту первого порядка V-лучевого преобразования $I^1 f$.
\end{theorem}

\begin{theorem} Векторное поле $f$ однозначно восстанавливается по заданным ее поперечному V-лучевому преобразованию $Jf$ и моменту первого порядка поперечного V-лучевого преобразования $J^1 f$.
\end{theorem}




%\section{Заключение}
%Решены задачи обращения лучевого и V-лучевого преобразований векторных и тензорных полей на плоскости и в пространстве. Отличие данных результатов от результатов других авторов состоит в том, что восстанавливается векторное поле целиком, при этом используются моменты первого порядка, а в случае тензорного поля определяется оператор Сен-Венана.
%Методы решения приведенных задач могут быть применены при обращении лучевых преобразований и V-лучевых преобразований тензорных полей высокого ранга.
%
%Результаты п.1 планируется опубликовать в ближайшем номере ДЭМИ.
%Результаты п.2 сданы в Вестник ДГУ, их обобщения на тензорные поля планируется отправить в один из центральных журналов.
%
%

