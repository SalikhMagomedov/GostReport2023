
\chapter{Об аппроксимативных свойствах рядов Фурье по полиномам Лагерра -- Соболева}

\section*{Введение}

Пусть $\alpha>-1$, $r\in \mathbb{N}$, $\rho(x)=e^{-x}x^\alpha$ -- весовая функция, $1\le p<\infty$, $L^p_\rho$ -- пространство измеримых функций $f$, определенных на полуоси $[0, \infty)$ и таких, что
$$
\|f\|_{L^p_\rho}=\left(\int\limits_0^{\infty}|f(x)|^p\rho(x) dx\right)^\frac{1}{p}<\infty,
$$
$W^r_{L^p_\rho}$ -- пространство функций $f$, непрерывно дифференцируемых $r-1$ раз, для которых $f^{(r-1)}$ абсолютно непрерывна на произвольном сегменте $[a, b]\subset[0, \infty)$, а $f^{(r)}\in L^p_\rho$.
Далее, через $W^r$ обозначим функции $f$ из $W^r_{L^2_\rho}$, для которых $|f^{(r)}(x)|e^{-\frac x2}\le 1$. В пространстве $W^r_{L^2_\rho}$ определим скалярное произведение типа Соболева
\begin{equation}\label{Gadzhimirzaev:SobInnerProd}
\langle f,g \rangle _S =\sum_{\nu=0}^{r-1}f^{(\nu)}(0)g^{(\nu)}(0)+\int_{0}^{\infty} f^{(r)}(x)g^{(r)}(x)\rho(x)dx.
\end{equation}

В работе~\cite{Gadzhimirzaev:DEMR2016} была введена система полиномов
\begin{equation*}
l_{r,r+n}^{\alpha}(x) =\frac{1}{(r-1)!\sqrt{h_n^\alpha}}\int\limits_{0}^x(x-t)^{r-1}L_{n}^{\alpha}(t)dt, \quad n=0,1,\ldots.
\end{equation*}
\begin{equation*}
l_{r,n}^{\alpha}(x) =\frac{x^n}{n!}, \quad n=0,1,\ldots, r-1,
\end{equation*}
ортонормированная при $\alpha>-1$ относительно скалярного произведения \eqref{Gadzhimirzaev:SobInnerProd} и порожденная системой полиномов Лагерра $\{L_{n}^{\alpha}(x)\}_{n=0}^\infty$.
В работе~\cite{Gadzhimirzaev:ShII-MMG} было показано, что система $\{l^\alpha_{r,k}(x)\}_{k=0}^\infty$ полна в $W^r_{L^2_\rho}$. Ряд Фурье функции $f\in W^r_{L^2_\rho}$ по этой системе имеет следующий вид
\begin{equation}\label{Gadzhimirzaev:Fourier_Series}
f(x)\sim \sum_{k=0}^{r-1}f^{(k)}(0)\frac{x^k}{k!}+\sum_{k=r}^{\infty} \hat{f}_{r,k}^\alpha l_{r,k}^\alpha(x),
\end{equation}
где
\begin{equation*}
\hat{f}_{r,k}^\alpha=\frac{1}{\sqrt{h_{k-r}^\alpha}}\int\limits_0^\infty f^{(r)}(t)L_{k-r}^\alpha(t)\rho(t)dt, \quad k\ge r.
\end{equation*}
В той же работе была доказана следующая

\begin{theoremA}\label{th:grm-a}
	Пусть $-1<\alpha<1$, $f\in W^r_{L^2_\rho}$, $0\le A<\infty$. Тогда для произвольного $x\in [0,\infty)$ имеет место равенство
	\begin{equation*}
	f(x) = \sum_{k=0}^{r-1}f^{(k)}(0)\frac{x^k}{k!}+\sum_{k=r}^{\infty} \hat{f}_{r,k}^\alpha l_{r,k}^\alpha(x),
	\end{equation*}
	в котором ряд Фурье функции $f$ по полиномам $l_{r,k}^{\alpha}(x)$ сходится равномерно относительно $x\in[0,A]$.
\end{theoremA}

Относительно параметра $p$ теорема~\ref{th:grm-a} была обобщена в работе~\cite{Gadzhimirzaev:RamIzv2020}.

\begin{theoremA}
	Пусть $-1<\alpha<1$. Тогда если $f\in W^r_{L^p_\rho}$, то при $p\ge2$ ряд \eqref{Gadzhimirzaev:Fourier_Series} сходится равномерно к $f$ на любом отрезке $[0,A]$. Если же $1\le p<2$, то существует функция $f\in W^r_{L^p_\rho}$, ряд Фурье которой расходится в точке $x=\pi^2$.
\end{theoremA}

Рассмотрим случай, когда $\alpha=0$. В этом случае для полиномов $l_{r,r+n}^{0}(x)$ имеет место равенство~\cite[следствие 3.1]{Gadzhimirzaev:ShII-MMG}:
\begin{equation*}
l_{r,r+n}^{0}(x)=\frac{x^r L_n^r(x)}{(n+r)^{[r]}},
\end{equation*}
где $(n+r)^{[r]}=(n+r)(n+r-1)\ldots(n+1)$.
Тогда ряд Фурье \eqref{Gadzhimirzaev:Fourier_Series} примет следующий вид
\begin{equation*}
f(x)\sim \sum_{\nu=0}^{r-1}f^{(\nu)}(0)\frac{x^k}{k!}+x^r\sum_{k=0}^{\infty} \frac{\hat{f}_{r,k+r}^0}{(k+r)^{[r]}}L_k^r(x).
\end{equation*}
Частичную сумму этого ряда обозначим через $S_{r,n+r}(f,x)$:
\begin{equation}\label{Gadzhimirzaev:part-sum}
S_{r,n+r}(f,x)=\sum_{\nu=0}^{r-1}f^{(\nu)}(0)\frac{x^k}{k!}+x^r\sum_{k=0}^{n} \frac{\hat{f}_{r,k+r}^0}{(k+r)^{[r]}}L_k^r(x).
\end{equation}
Из \eqref{Gadzhimirzaev:part-sum} следует, что для $S_{n+r}(f,x)$ имеют место равенства
\begin{equation*}
S_{r,n+r}^{(\nu)}(f,0)=f^{(\nu)}(0), \quad 0\le\nu\le r-1.
\end{equation*}
Кроме того, если $f(x)=p_{n+r}(x)$ -- алгебраический полином степени $n+r$, то
\begin{equation}\label{Gadzhimirzaev:part-sum-second-prop}
S_{r,n+r}(p_{n+r},x)\equiv p_{n+r}(x).
\end{equation}

Используя свойство \eqref{Gadzhimirzaev:part-sum-second-prop}, авторы работы \cite{Gadzhimirzaev:ShII-MMG} исследовали аппроксимативные свойства частичных сумм $S_{r,n+r}(f,x)$. В частности, для $f\in W^r_{L^2_\omega}$, $\omega(x)=e^{-x}$ было показано,
что имеет место неравенство типа Лебега~\cite[теорема 5.2]{Gadzhimirzaev:ShII-MMG}
$$
e^{-x/2}x^{-r/2+1/4}|f(x)-S_{r,n+r}(f,x)|\le (1+\lambda_{r,n}(x))E_{n+r}^r(f),
$$
в котором для функции $\lambda_{r,n}(x)$ справедливы оценки
$$
\lambda_{r,n}(x)\le c(r)
\begin{cases}
	\ln(n+1), & x\in[0,\kappa/2]; \\
	\ln(n+1)+(x/(\kappa^{1/3}+|x-\kappa|))^{1/4}, & x\in[\kappa/2,3\kappa/2]; \\
	n^{-r/2+5/4}e^{-x/4}, & x\in[3\kappa/2,\infty).
\end{cases}
$$
Здесь $c(r)$ положительная константа, зависящая только от $r$, $\kappa=4n+2r+2$.
Величина $E_{n+r}^r(f)$ определяется равенством
$$
E_{n+r}^r(f)=\inf_{q_{n+r}}\sup_{x>0}|q_{n+r}(x)-f(x)|e^{-x/2}x^{-r/2+1/4},
$$
где нижняя грань берется по всем алгебраическим полиномам $q_{n+r}$ степени $n+r$, для которых $f^{(\nu)}(0)=q_{n+r}^{(\nu)}(0)$, $\nu=\overline{0,r-1}$.

Вышеприведенные оценки для разности $|f(x)-S_{r,n+r}(f,x)|$ содержат величину наилучшего приближения $E_{n+r}^r(f)$, поведение которой все еще не исследовано. В этом разделе для случая $r=1$ получена оценка скорости сходимости частичных сумм $S_{1,n+1}(f,x)$ к функции $f(x)$, не содержащая величины наилучшего приближения $E_{n+r}^r(f)$.

Аналогичные задачи о приближении функций из пространства Соболева алгебраическими полиномами были исследованы в работах различных авторов
(см. \cite{Approx-Xu, Approx-XuWang, Approx-Juan, Approx-Leonardo} и цитированную в них литературу). В частности, в \cite{Approx-Xu} при $1\le p<\infty$ было рассмотрено пространство $W^r_{L^p_w}=W^r_{L^p_w}[-1,1]=\{f\in C^{r-1}[-1,1]: f^{(r)}\in L^p_w\}$, $w(x)=w_{\alpha,\beta}(x)=(1-x)^\alpha(1+x)^\beta$ с нормой
$$
\|f\|_{W^r_{L^p_w}}=\left(\sum_{k=0}^{r}\|f^{(k)}\|^p_{L^p_w}\right)^{1/p}.
$$
Для функций из этого пространства были доказаны следующие теоремы.

\begin{theoremA}
	Пусть $\alpha,\beta>-1$. Предположим, что $f\in W^r_{L^p_w}$ при $1\le p<\infty$ или $f\in C^r[-1,1]$ при $p=\infty$. Тогда существует полином $p_n$ такой, что
	$$
	\|f-p_n\|_{W^r_{L^p_w}}\le c E_n(f^{(r)})_{L^p_w},
	$$
	где $E_n(f^{(r)})_{L^p_w}=\inf\limits_{p\in\Pi_n}\|f^{(r)}-p\|_{L^p_w}$ -- величина наилучшего приближения в метрике $L^p_w$.
\end{theoremA}

\begin{theoremA}
	Пусть $\alpha,\beta>-1$. Предположим, что $f\in W^r_{L^p_w}$ при $1\le p<\infty$ или $f\in C^r[-1,1]$ при $p=\infty$. Тогда существует полином $p_n$ такой, что
	$$
	\|f^{(k)}-p_n^{(k)}\|_{L^p_w}\le c n^{-r+k}E_n(f^{(r)})_{L^p_w}
	$$
	при условии, что либо $\alpha=0$, либо $\beta=0$.
\end{theoremA}

Далее, в работе \cite{Approx-Juan} было рассмотрено пространство $H^r_\omega=H^r_\omega(a,b)=\{f\in L^2_\omega(a,b): f^{(m)}\in L^2_\omega(a,b), 1\le m\le r\}$ с нормой
$$
\|f\|_{H^r_\omega}=\left(\sum_{m=0}^{r}\|f^{(m)}\|^2_{L^2_w}\right)^{1/2}.
$$
Здесь $(a, b)=\mathbb{R}$ при $\omega(x)=e^{-x^2}$, $(a,b)=(0, \infty)$ при $\omega(x)=e^{-x}x^\alpha$, $\alpha>-1$ и $(a,b)=(-1, 1)$ при $\omega(x)=(1-x)^\alpha(1+x)^\beta$, $\alpha, \beta>-1$.
Был исследован вопрос о приближении функций из этого пространства в метрике $L^2_\omega$ посредством частичных сумм $\mathcal{S}_n^Nf(x)$ ряда Фурье по системе полиномов $\{q_j(x)\}$, ортогональной относительно скалярного произведения Соболева. А именно, была доказана следующая

\begin{theoremA}
	Пусть $r\ge N+1$ и функция $f\in H^r_\omega$ такая, что $f\in L^2_{v_{r-N-1}}$. Тогда оценки
	$$
	\|f^{(m)}-(\mathcal{S}_n^Nf)^{(m)}\|_{L^2_\omega}\le c
	\begin{cases}
	\frac{(-\lambda_{n-N,0})^{(m-N)/2}}{(-\lambda_{n-r,r-N-1})^{(r-N-1)/2}}E_{n-r}(f^{(r)})_{L^2_{v_{r-N-1}}}, N\le m\le r, \\
	\frac{(-\lambda_{n-N,0})^{N/2}}{(-\lambda_{n-r,r-N-1})^{(r-N-1)/2}}E_{n-r}(f^{(r)})_{L^2_{v_{r-N-1}}}, 0\le m\le N-1,
	\end{cases}
	$$
	выполняются всегда в случае весовой функции Эрмита, для $\alpha>-1$ в случае весовой функции Лагерра и при $\alpha, \beta\ge0$ в случае весовой функции Якоби.
\end{theoremA}

\section{Некоторые сведения о полиномах Лагерра}

Пусть $\alpha$ произвольное действительное число. Тогда для полиномов Лагерра $L_n^\alpha(x)$ справедливы следующие соотношения~\cite{Gadzhimirzaev:Szego}:
\begin{itemize}
\item
формула Родрига
$$
L_n^\alpha(x)=\frac{1}{n!}x^{-\alpha}e^x\left(x^{n+\alpha}e^{-x}\right)^{(n)};
$$

\item
соотношение ортогональности
$$
\int_{0}^{\infty}L_n^\alpha(x)L_m^\alpha(x)\rho(x)dx=h_n^\alpha\delta_{n,m},\ \alpha>-1,
$$
где $\delta_{n,m}$ -- символ Кронекера, $h_n^\alpha=\frac{\Gamma(n+\alpha+1)}{\Gamma(n+1)}$;
\item
формула Кристоффеля -- Дарбу
\begin{equation*}\label{Gadzhimirzaev:Darbu}	
K_n^\alpha(x,t)=\sum_{k=0}^{n}\frac{L_k^\alpha(x)L_k^\alpha(t)}{h_k^\alpha}=
\frac{n+1}{h_n^\alpha}\frac{L_n^\alpha(x)L_{n+1}^\alpha(t)-L_{n+1}^\alpha(x)L_n^\alpha(t)}{x-t};
\end{equation*}

\item
рекуррентная формула
$$
L_0^\alpha(x)=1, \ L_1^\alpha(x)=-x+\alpha+1,
$$
\begin{equation*}\label{Gadzhimirzaev:recur}
nL_n^\alpha(x)=(-x+2n+\alpha-1)L_{n-1}^\alpha(x)-(n+\alpha-1)L_{n-2}^{\alpha}(x), \ n\ge2;
\end{equation*}

\item
равенства
\begin{equation*}\label{Gadzhimirzaev:prop1}
nL_n^\alpha(x)=(n+\alpha)L_{n-1}^\alpha(x)-xL_{n-1}^{\alpha+1}(x),
\end{equation*}
\begin{equation*}\label{Gadzhimirzaev:prop2}
L_n^{\alpha-1}(x)=L_{n}^\alpha(x)-L_{n-1}^{\alpha}(x);
\end{equation*}

\item
весовая оценка~\cite{Gadzhimirzaev:AskeyWain, Gadzhimirzaev:Mocken}
\begin{equation*}\label{Gadzhimirzaev:weight-est}
e^{-\frac{x}{2}}|L_n^\alpha(x)|\le c(\alpha)A_n^\alpha(x),\ \alpha>-1.
\end{equation*}
\end{itemize}
Здесь и далее $c$, $c(\alpha)$ -- положительные числа, зависящие от указанных параметров,
$$
A_n^\alpha(x)=
\begin{cases}
	\theta^\alpha, & 0\le x\le \frac{1}{\theta}; \\
	\theta^{\frac{\alpha}{2}-\frac14}x^{-\frac{\alpha}{2}-\frac14}, & \frac{1}{\theta}< x\le \frac{\theta}{2}; \\
	\left[\theta\left(\theta^{\frac13}+|x-\theta|\right)\right]^{-\frac14}, & \frac{\theta}{2}< x\le \frac{3\theta}{2}; \\
	e^{-\frac x4}, & \frac{3\theta}{2}<x,
\end{cases}
$$
где $\theta=\theta_n(\alpha)=4n+2\alpha+2$.

Для ортонормированных полиномов Лагерра $l_n^\alpha(x)=\frac{1}{\sqrt{h_n^\alpha}}L_n^\alpha(x)$ имеют место оценки
\begin{equation*}\label{Gadzhimirzaev:orth-est}
e^{-\frac{x}{2}}|l_{n+1}^\alpha(x)-l_{n-1}^\alpha(x)|\le c(\alpha)
\begin{cases}
		\theta^{\frac{\alpha}{2}-1}, & 0\le x\le \frac{1}{\theta}; \\
		\theta^{-\frac34}x^{-\frac{\alpha}{2}+\frac14}, & \frac{1}{\theta}< x\le \frac{\theta}{2}; \\
		x^{-\frac{\alpha}{2}}\theta^{-\frac34}\left(\theta^{\frac13}+|x-\theta|\right)^{\frac14}, & \frac{\theta}{2}< x\le \frac{3\theta}{2}; \\
		e^{-\frac x4}, & \frac{3\theta}{2}<x.
\end{cases}
\end{equation*}

\section{Вспомогательные утверждения и основной результат}

Пусть $\mathcal{K}_{n}(x,t)=\sum\limits_{k=0}^{n}L_k^1(x)L_k(t)$. Справедливы следующие утверждения.

\begin{lemma}
Имеет место равенство
\begin{equation*}\label{Gadzhimirzaev:ker}
(x-t)\mathcal{K}_{n}(x,t)=(n+1)\left(L^1_{n}(x)L_{n+1}(t)-L^1_{n+1}(x)L_{n}(t)\right)+\sum_{k=0}^{n}L_{k}(x)L_{k}(t).
\end{equation*}
\end{lemma}

\begin{lemma}
Для величины $\mathcal{K}_{n}(x,t)$ справедливо равенство
\begin{equation*}
(x-t)\mathcal{K}_{n}(x,t)=xL^2_{n}(x)L_{n}(t)-tL^1_{n}(x)L^1_{n}(t)-L^1_{n}(x)L_{n}(t)+\sum_{k=0}^{n}L_{k}(x)L_{k}(t).
\end{equation*}
\end{lemma}

\begin{lemma}
Для $K_{n}^0(x,t)=\sum\limits_{k=0}^{n}L_{k}(x)L_{k}(t)$ имеет место представление
$$
K_n^0(x,t)=\frac{n+1}{2n+1}L_{n}(x)L_{n}(t)+
$$
\begin{equation*}	
\frac{n(n+1)}{2n+1}\frac{1}{t-x}\left[L_{n}(t)\left(L_{n+1}(x)-L_{n-1}(x)\right)-L_{n}(x)\left(L_{n+1}(t)-L_{n-1}(t)\right)\right].
\end{equation*}
\end{lemma}

\begin{lemma}
Имеет место равенство
\begin{equation*}
\sum_{k=0}^{\infty}\left(\frac{x^{\frac34}L_k^1(x)}{k+1}\right)^2 = \frac{e^{x}-1}{\sqrt{x}}, \ x\in[0,\infty).
\end{equation*}
\end{lemma}

\begin{lemma}
\cite{Gadzhimirzaev:mathnot2021}	
Пусть $\alpha>-1$, $n\in\mathbb{N}$, $x\in[0,\infty)$. Тогда имеют место следующие оценки
\begin{equation*}
e^{-x}K_{n}^\alpha(x,x)\le c(\alpha)
\begin{cases}
n^{-\alpha}, & x\in[\theta_n/2, 3\theta_n/2], \\
n^{1-\alpha}(A_n^\alpha(x))^2, & x\in[0,\theta_n/2]\cup[3\theta_n/2,\infty).
\end{cases}
\end{equation*}
\end{lemma}

Основным результатом этого раздела является следующая
\begin{theorem}\label{Gadzhimirzaev:theorem1}
Пусть $f\in W^1$, $x\in[0,\infty)$. Тогда имеет место оценка
$$
\frac{x^{-\frac14}e^{-\frac x2}}{\sqrt{x+1}}|f(x)-S_{1,n+1}(f,x)|\le c
\frac{\ln(n+1)}{n^{\frac14}},
$$
где $c$ -- положительная константа.
\end{theorem}

