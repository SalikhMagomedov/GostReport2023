% TODO: change chapter name
\chapter{Отчет ведущего научного сотрудника ОМИ ДНЦ РАН 0.5 ст. д.ф.-м.н., проф. Магомедова А.М. за 2023 г.}

\underline{Объектом исследования}
явилась модификация алгоритма динамического программирования по профилю для усовершенствования решения задачи подсчета  $\tau(m, n)$ -- числа совершенных паросочетаний в решёточном графе $m\times n$ (задача о димерных числах),  равносильной задаче  перечисления разбиений прямоугольника $M(m\times n)$ на плитки $1\times 2$ ($m$ -- четное). 

Задача возникает в исследованиях свойств химических соединений, а также при исследовании адсорбции двухатомных молекул на поверхности. 
Известны классы химических соединений, которые синтезируются только тогда, когда графы соединений в топологической модели молекулы имеют совершенное паросочетание; более того, стабильность компонентов этих семейств зависит от количества совершенных паросочетаний в их графах.

Еще более актуальной задача вычисления димерных чисел оказалась в физике; вычисление числа способов объединения атомов в двухатомные молекулы (димеры) с соблюдением некоторых условий и привела известных ученых-физиков Kateleyn P. W., Temperley H. N. V. и Fisher M. E. к знаменитой формуле двойного произведения. Но все известные формулы решения задачи использовали операции с плавающей запятой, что требует значительных компьютерных ресурсов.
Исключение составляет алгоритм, получивший название <<динамическое программирование

\section{Основной результат}

Запланированные цели и направления усовершенствования известных из литературы реализаций алгоритма динамического программирования по профилю: 1) уменьшение вычислительной сложности, 2) оптимизация требуемого объема памяти.

Цели оптимизации алгоритма в основном достигнуты, но в процессе исследования оказались вытеснены на второстепенное по значимости место тесными связями с числами Фибоначчи, обнаруженными на различных стадиях решения задачи перечисления разбиений прямоугольника на плитки $1\times 2$ методом динамического программирования по профилю.

%\begin{definition}\label{Def1}
Пусть рассматривается покрытие заданного прямоугольника $M$; пронумеруем плитки покрытия по правилу: <<из двух плиток меньший номер получает та плитка, которая ближе к основанию прямоугольника $M$>> и будем считать, что каждая плитка уложена в момент времени, равный ее номеру. Таким образом, если  хотя бы одна из клеток плитки покрытия ниже обеих клеток другой плитки, то первая плитка уложена раньше  (\textit{правило хронологического покрытия}).
%\hhline  width 2mm

В каждый момент времени состояние покрытости клеток любой строки будем записывать в виде $m$-разрядной двоичной строки, где бит в разряде $j$ равен $1$, если $j$-я клетка строки покрыта плиткой, $0$ – если $j$-я клетка не покрыта; $j=0,1,...,m-1.$  
Обозначим через $t_i$ наименьший момент времени, когда в результате укладки очередной плитки $G$ оказалась покрытой последняя из непокрытых клеток $g$ строки $i-1$.
Двоичный образ строки $i$ в момент времени $t_i$ будем называть $i$-\textit{профилем};
 $i$-профиль $q$ и $(i+1)$-профиль $p$ хронологического покрытия будем называть {\it близкими} профилями.
Для заданного $i$-профиля $q$ количество всевозможных покрытий части исходного прямоугольника, образованной начальными $i+q[j]-1$ клетками каждого $j$-го столбца ($j=0,..., m-1$), обозначим $a[i,q]$.
%\end{definition}

Тогда, во-первых, 
\begin{equation}\label{eq2}
a[i+1,p]=\sum_{i-\text{профили}\; q,\, \text{близкие с}\; p}\ a[i,q] ,\,\,                        
i=1,...,n;\,\, p=0,…,2^m-1;     
  %\eqno (2)
\end{equation}
во-вторых,
%\begin{color}{red}
\[a[1,0]=1; a[1,1]=...=a[1,2^m-1]=0.\]
%\end{color}

Формула \eqref{eq2} приведена в ряде источников
%, например, в \cite{Vol13}--\cite{Vas14}, 
 с нетривиальным уточнением, что искомым результатом будет число $a[n+1,0]$, 
а не $a[n,2^m-1]$.

%Однако вычисления по формуле \eqref{eq2} сопряжены с проблемами малой скорости вычислений и нехватки памяти. 
%
%В последующих разделах обсуждаются пути их преодоления.\par
%\end{fulltext}
\begin{enumerate}
	\item 
Найден эффективно проверяемый критерий совместимости профилей:
%[Критерий совместимости профилей]

Для совместимости $m$-разрядных профилей $q$ и $p$ необходимо и достаточно выполнение следующих условий:\\
$q\boxtimes p=0$ (побитовое произведение $p$ и $q$ равно нулю); \\
побитовая сумма $q\boxplus p$ не содержит нечетных 0-серий.

Для $m=4$ результаты вычислений  см. в таблице \ref{tab02}.

%\\cline{2-9} \cline{11-18
\begin {table}[ht]
\caption 
{ $F_{i,j}=1$, если профили $i$ и $j$ совместимы  ($m=4$)}
%\centering { 
\label {tab02}
%}
\begin{center}
\begin{tabular}{|p{0.19in}|p{0.15in}|p{0.19in}|p{0.19in}|p{0.19in}|p{0.19in}|p{0.19in}|p{0.19in}|p{0.19in}|p{0.19in}|p{0.19in}|p{0.19in}|p{0.19in}|p{0.19in}|p{0.19in}|p{0.19in}|p{0.19in}|p{0.19in}| }
\hline
\centering
  & p & \scriptsize{0000} & \scriptsize{0001} & \scriptsize{0010} & \scriptsize{0011} & \scriptsize{0100} & \scriptsize{0101} & \scriptsize{0110} & \scriptsize{0111} & \scriptsize{1000} & \scriptsize{1001} & \scriptsize{1010} & \scriptsize{1011} & \scriptsize{1100} & \scriptsize{1101} & \scriptsize{1110} & \scriptsize{1111} \\
\hline
q &  & 0 & 1 & 2 & 3 & 4 & 5 & 6 & 7 & 8 & 9 & 10 & 11 & 12 & 13 & 14 & 15 \\
\hline
\scriptsize{0000} & 0 & 1 & 0 & 0 & 1 & 0 & 0 & 0 & 0 & 0 & 1 & 0 & 0 & 1 & 0 & 0 & 1 \\
\hline
\scriptsize{0001} & 1 & 0 & 0 & 1 & 0 & 0 & 0 & 0 & 0 & 1 & 0 & 0 & 0 & 0 & 0 & 1 & 0 \\
\hline
\scriptsize{0010} & 2 & 0 & 1 & 0 & 0 & 0 & 0 & 0 & 0 & 0 & 0 & 0 & 0 & 0 & 1 & 0 & 0 \\
\hline
\scriptsize{0011} & 3 & 1 & 0 & 0 & 0 & 0 & 0 & 0 & 0 & 0 & 0 & 0 & 0 & 1 & 0 & 0 & 0 \\
\hline
\scriptsize{0100} & 4 & 0 & 0 & 0 & 0 & 0 & 0 & 0 & 0 & 1 & 0 & 0 & 1 & 0 & 0 & 0 & 0 \\
\hline
\scriptsize{0101} & 5 & 0 & 0 & 0 & 0 & 0 & 0 & 0 & 0 & 0 & 0 & 1 & 0 & 0 & 0 & 0 & 0 \\
\hline
\scriptsize{0110} & 6 & 0 & 0 & 0 & 0 & 0 & 0 & 0 & 0 & 0 & 1 & 0 & 0 & 0 & 0 & 0 & 0 \\
\hline
\scriptsize{0111} & 7 & 0 & 0 & 0 & 0 & 0 & 0 & 0 & 0 & 1 & 0 & 0 & 0 & 0 & 0 & 0 & 0 \\
\hline
\scriptsize{1000} & 8 & 0 & 1 & 0 & 0 & 1 & 0 & 0 & 1 & 0 & 0 & 0 & 0 & 0 & 0 & 0 & 0 \\
\hline
\scriptsize{1001} & 9 & 1 & 0 & 0 & 0 & 0 & 0 & 1 & 0 & 0 & 0 & 0 & 0 & 0 & 0 & 0 & 0 \\
\hline
\scriptsize{1010} & 10 & 0 & 0 & 0 & 0 & 0 & 1 & 0 & 0 & 0 & 0 & 0 & 0 & 0 & 0 & 0 & 0 \\
\hline
\scriptsize{1011} & 11 & 0 & 0 & 0 & 0 & 1 & 0 & 0 & 0 & 0 & 0 & 0 & 0 & 0 & 0 & 0 & 0 \\
\hline
\scriptsize{1100} & 12 & 1 & 0 & 0 & 1 & 0 & 0 & 0 & 0 & 0 & 0 & 0 & 0 & 0 & 0 & 0 & 0 \\
\hline
\scriptsize{1101} & 13 & 0 & 0 & 1 & 0 & 0 & 0 & 0 & 0 & 0 & 0 & 0 & 0 & 0 & 0 & 0 & 0 \\
\hline
\scriptsize{1110} & 14 & 0 & 1 & 0 & 0 & 0 & 0 & 0 & 0 & 0 & 0 & 0 & 0 & 0 & 0 & 0 & 0 \\
\hline
\scriptsize{1111} & 15 & 1 & 0 & 0 & 0 & 0 & 0 & 0 & 0 & 0 & 0 & 0 & 0 & 0 & 0 & 0 & 0 \\
\hline 
\end{tabular}
\end{center}
\end{table}

\item
Поскольку в практических вопросах сокращение размеров необходимой памяти вдвое --- весьма существенное достижение, отметим еще один шаг в данном направлении.
%\underline{Теорема 2.} 

Профиль $p$: 
1) совместим с каждым профилем вида $2^k-p-1$, $k=2,4,6,\dots,m$,
2) несовместим ни с одним профилем из интервала
$$(2^k-1-p ; 2^{k+1}-1-p).$$

\item
Построен алгоритм, который для любого заданного $m$-разрядного $i$-профиля $q$ генерирует полный список профилей, близких к $q$;\\
мощность списка равна произведению
$$f(d_0+1) \cdot  f(d_1+1)\cdot ... \cdot  f(d_{r-1}+1),$$
где $f(k)$ обозначает число Фибоначчи с номером $k$, $r$ -- количество 0-серий в $q$, $d_i$ -- длина $i$-й 0-серии.
\item 
Построен модифицированный алгоритм вычисления $\tau(m,n)$, оптимизированный по быстродействию и требуемой оперативной памяти.
Об эффективности построенного алгоритма свидетельствует следующий факт.

Попытка вычислить $\tau(8,18000)$ на основе формулы \eqref{eq2} потерпела неудачу, в то же время модифицированный алгоритм <<без труда>> справился -- при том же программном и аппаратном обеспечении -- с вычислением значения $\tau(m,n)$ при $m=8$, $n=150000$ (приводим лишь начальный и конечный фрагменты десятичной записи числа, полная запись содержит 143189 цифр):
$$\tau(8,150000)=13873887020492488 \dots 05663089457. $$

\end{enumerate}
