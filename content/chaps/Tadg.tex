

\begin{abstract}
  Исследована базисность полиномов Якоби $P_n^{\alpha,\beta}(x)$ в пространствах Лебега \linebreak$L^{p(\cdot)}_{w_{a,b}}([-1,1])$ с переменным показателем $p(x)$ и весом Якоби $w_{a,b}(x)=(1-x)^a(1+x)^b$, и было показано, что при $-1<\alpha,\beta<1/2$ полиномы $P_n^{\alpha,\beta}(x)$ образуют базис в $L^{p(\cdot)}_{w_{a,b}}([-1,1])$, когда вес $w_{a,b}(x)$ удовлетворяет условию Мукенхаупта. Также был усилен результат, касающийся базисности полиномов Лежандра ($\alpha=\beta=0$) в $L^{p(\cdot)}_{w_{00}}([-1,1])$, а именно убрано условие постоянства переменного показателя у концов отрезка $[-1,1]$.
\end{abstract}


%\section*{Введение}

%Интерес к пространствам Лебега с переменным показателем $L^{p(\cdot)}$ возрос с 1990-х годов ввиду
%их использования в различных приложениях. Прежде всего, это математические
%моделирование электрореологических жидкостей. Эти пространства также использовались для моделирования поведения других физических явлений, а также для изучения процессов обработки изображений и т. д. Эти задачи, в свою очередь, приводят к поиску систем, образующих базисы в этих пространствах. В работах И.И. Шарапудинова и его учеников \cite{tad-SHII-Haar,tad-SHII-AnalisysMath,tad-SHII-Leg,tad-MMG-Haar,tad-SHII-Jacob,tad-SHII-Ult,tad-RAM-Jacob} была показана базисность в $L^{p(\cdot)}$  тригонометрической системы, системы полиномов Якоби и системы функций Хаара при определенных условиях на показатель $p(x)$.
%
%В отчетном году была предпринята попытка ослабить эти условия на переменный показатель $p(x)$ (постоянство показателя на концах отрезка $[-1,1]$) в случае системы полиномов Лежандра. При исследовании этой задачи возникла необходимость изучения свойств ядра $K(x,y)$, связанного с ядром Кристоффеля -- Дарбу для системы полиномов Лежандра. А именно для $K(x,y)$ получены  представления и оценки, в зависимости от расположения $x,y$ на квадрате $[-1,1]\times[-1,1]$.

%Ссылка на формулу -- \eqref{PifagorTh}.
%
%\begin{lemma}\label{lemma1}
%  Пример оформления леммы.
%\end{lemma}
%
%Примеры ссылок на литературу: \cite{metka1}, \cite{metka1,metka2,metka3,metka4}.
%
%
%\begin{definition}
%  Пример оформления определения.
%\end{definition}
%
%\begin{theorem}
%  Пример оформления теоремы.
%\end{theorem}
%\begin{proof}
%  Текст доказательства. Ссылка на Лемму \ref{lemma1}.
%\end{proof}
%
%
%\begin{remark}
%  Пример оформления замечания.
%\end{remark}

%Примеры вставки картинок (должны быть помещены в папку pictures):
%\begin{figure}[h]
%  %\centering
%  % Requires \usepackage{graphicx}
%  \includegraphics[width=100pt]{Figure1}
%  \caption{Название картинки}\label{FMetka1}
%\end{figure}
%\begin{figure}[h]
%  %\centering
%  % Requires \usepackage{graphicx}
%  \begin{center}
%  \includegraphics[width=100pt]{Figure1}\includegraphics[width=100pt]{Figure1}\\
%  \caption{Название картинки}\label{FMetka2}
%  \end{center}
%\end{figure}


%Ссылка на картинку -- рис. \ref{FMetka1}

\section{О базисности системы полиномов Якоби в пространствах Лебега с переменным показателем}
%\begin{abstract}
%Рассмотрен вопрос о базисности полиномов Якоби $P_n^{\alpha,\beta}(x)$ в весовом пространстве Лебега $L^{p(\cdot)}_{w_{a,b}}([-1,1])$ с переменным показателем при $-1<\alpha,\beta<-1/2$.
%\end{abstract}
\subsection{Введение}
Пусть $p(x)$ -- неотрицательная на $[-1,1]$ измеримая функция, $a,b>-1$, $w_{a,b}(x)=(1-x)^a(1+x)^b$ -- весовая функция. Через $L^{p(\cdot)}_{w_{a,b}}([-1,1])$ обозначим множество функций $f$ таких, что
\begin{equation}\label{s2-lpx-def-1}
  \int_{-1}^1|f(x)|^{p(x)}w_{a,b}(x)dx<\infty,
\end{equation}
-- пространство Лебега с переменным показателем. Введем следующие обозначения
$$p_-(A)=ess\inf_{x\in A}p(x),\ p_+(A)=ess\sup_{x\in A}p(x),$$
где $A$ -- измеримое множество.
 При условии $1\le p_-([-1,1])\le p(x)\le p_+([-1,1])<\infty$ пространство $L^{p(\cdot)}_{w_{a,b}}([-1,1])$ нормируемо \cite{tad-lpxtopology} и одну из эквивалентных норм можно задать следующим образом
\begin{equation}\label{s2-lpx-norm}
  \|f\|_{p(\cdot),w_{a,b}}([-1,1])=\inf\left\{\lambda>0:\int_{-1}^1\left|\frac{f(x)}\lambda\right|^{p(x)}w_{a,b}(x)dx\le1\right\}.
\end{equation}
Отметим отдельно случай $a=b=0$. Тогда $w_{a,b}(x)\equiv1$ на $[-1,1]$. Для этого случая положим $$L^{p(\cdot)}([-1,1])=L^{p(\cdot)}_{w_{0,0}}([-1,1]),$$
$$\|f\|_{p(\cdot)}([-1,1])=\|f\|_{p(\cdot),w_{0,0}}([-1,1]).$$
Приведем некоторые сведения о полиномах Якоби. Для произвольных $\alpha,\beta$ полиномы Якоби можно определить с помощью формулы Родрига
\begin{equation}\label{RodrigueFormula}
  P_n^{\alpha,\beta}(x) = {(-1)^n\over2^nn!}{1\over w_{\alpha,\beta}(x)}{d^n\over
dx^n} \left\{w_{\alpha,\beta}(x)\sigma^n(x)\right\},
\end{equation}
где $\sigma(x)=1-x^2$. Если $\alpha,\beta>-1$, то полиномы Якоби образуют ортогональную систему с весом  $w_{a,b}(x)$, то есть
\begin{equation}\label{JacobiOrthognality}
	\int_{-1}^1P_n^{\alpha,\beta}(x)P_m^{\alpha,\beta}(x)w_{a,b}(x)dx=\delta_{nm},
\end{equation}
где $\delta_{nm}$ -- символ Кронеккера.
При условии $\alpha,\beta>-1$ для функции $f\in L^{p(\cdot)}_{w_{\alpha,\beta}}([-1,1])$ можно определить коэффициенты Фурье -- Якоби
$$
f_k^{\alpha,\beta}=\frac{1}{h_k^{\alpha,\beta}}\int_{-1}^1f(x)P_k^{\alpha,\beta}(x)
\mu(\alpha,\beta,x)dx
$$
и сопоставить ей ряд Фурье -- Якоби
$$
f\sim \sum_{k=0}^\infty f_k^{\alpha,\beta}P_k^{\alpha,\beta}(x).
$$
Определим также частичные суммы Фурье -- Якоби
$$
S_n^{\alpha,\beta}(f)=S_n^{\alpha,\beta}(f,x)=\sum_{k=0}^nf_k^{\alpha,\beta}P_k^{\alpha,\beta}(x).
$$

 Вопрос о базисности полиномов Якоби в пространствах Лебега тесно связан со сходимости рядов Фурье--Якоби в среднем. В случае постоянного показателя он был исследован в работах \cite{tad-Pollard-1,tad-Pollard-2,tad-Pollard-3,tad-Newman-Rudin,tad-Muckenhoupt}. В частности, Г. Поллард в  \cite{tad-Pollard-1} показал, что система полиномов Лежандра является базисом пространства Лебега $L^p([-1,1])$ при $\frac43<p<4$ и не является им при $p\in[1,\frac43)\cup(4,\infty)$. Д. Ньюман и В. Рудин в \cite{tad-Newman-Rudin} дополнили этот результат, показав, что полиномы Лежандра не образуют базиса и при $p \in \{\frac43, 4\}$. В случае пространства Лебега с переменным показателем вопрос о базисности полиномов Лежандра рассмотрен И.И. Шарапудиновым в работе \cite{tad-SHII-Leg}. Им были получены условия на переменный показатель при которых имеет место базисность полиномов Лежандра в $L^{p(\cdot)}([-1,1])$. Остановимся на них подробнее. Через $\mathcal{P}(-1,1)$ обозначим класс переменных показателей $p(x)$, заданных на $[-1,1]$ и удовлетворяющих условиям:

$A)$  условие Дини -- Липшица
\begin{equation}\label{tad-DiniLipCond}
  |p(x)-p(y)|\le \frac{d}{-\ln|x-y|}, \quad |x-y|\le\frac12;
\end{equation}

$B)$  $\underline{p}([-1,1])=\min_{x\in [-1,1]}p(x)>1$;

$C)$ для $p(x)$ существуют (произвольно малые) числа $\delta_i=\delta_i(p)$ $(i=1,2)$ такие что $p(x)=p(-1)$ для $x\in[-1,-1+\delta_1]$ и $p(x)=p(1)$ для $x\in [1-\delta_2,1]$.
В \cite{tad-SHII-Leg} была доказана следующая теорема.
\begin{theoremA}\label{ShII-Th}
	Пусть $p=p(x)\in\mathcal{P}(-1,1)$  и $p(\pm1)\in(\frac43,4)$. Тогда ортонормированная система полиномов Лежандра является базисом пространства Лебега с переменным показателем $L^{p(\cdot)}([-1,1])$.
\end{theoremA}
Дальнейшее развитие вопросы  базисности полиномов Якоби получили в \cite{tad-SHII-Jacob,tad-SHII-Ult,tad-RAM-Jacob}, где была показана базисность полиномов Якоби в $L^{p(\cdot)}_{w_{\alpha,\beta}}([-1,1])$ при условии, что $p(x)\in\mathcal{P}(-1,1)$.  Как отмечается в \cite{tad-SHII-Leg}, величины $\delta_1$ и $\delta_2$ могут быть сколь угодно малыми и в отчетном году был рассмотрен вопрос устранения условия постоянства переменного показателя в окрестностях точек $\pm1$ в случае полиномов Лежандра. Кроме того, рассмотрена задача базисности полиномов $P_n^{\alpha,\beta}(x)$ ($-1<\alpha<-1/2$) в том случае, когда параметры у $w_{a,b}(x)$ могут отличаться от $\alpha$, $\beta$.

\subsection{Базисность полиномов Лежандра}
В работах \cite{tad-SHII-Leg,tad-SHII-Jacob,tad-SHII-Ult,tad-RAM-Jacob} базисность полиномов Якоби в пространствах $L^{p(\cdot)}_{w_{\alpha,\beta}}([-1,1])$ была показана при условии постоянства переменного показателя у концов отрезка $[-1,1]$. В случае полиномов Лежандра нам удалось избавиться от этого условия. Для формулировки основного результата введем некоторые обозначения. Через $P_n(x)=P^{0,0}_n(x)$ обозначим полиномы Лежандра.
Они образуют ортогональную с единичным весом систему на отрезке $[-1,1]$:
\begin{equation*}
  \int_{-1}^1P_n(x)P_m(x)dx=\frac{2}{2n+1}\delta_{nm},
\end{equation*}
Ортонормированная система полиномов Лежандра имеет вид
\begin{equation}\label{tad-LegOrthonormed}
  \hat{P}_n(x)=\sqrt{\frac{2n+1}{2}}P_n(x).
\end{equation}
Функции $f(x)$, интегрируемой на $[-1,1]$, мы можем поставить в соответствие ряд Фурье -- Лежандра
\begin{equation}\label{tad-FourLegSeries}
  f\sim\sum_{k=0}^\infty f_kP_k(x),
\end{equation}
где
\begin{equation*}
  f_k=\frac{2k+1}{2}\int_{-1}^1f(t)P_k(t)dt
\end{equation*}
-- коэффициенты Фурье -- Лежандра функции $f$. Частичная сумма ряда \eqref{tad-FourLegSeries} имеет вид
\begin{equation*}
  S_n(f,x)=\sum_{k=0}^n f_kP_k(x).
\end{equation*}
Основным результатом настоящего пункта является следующая
\begin{theorem}\label{tad-Th1}
	Пусть $p=p(x)>1$ удовлетворяет условию Дини -- Липшица \eqref{tad-DiniLipCond} и $p(\pm1)\in(\frac43,4)$. Тогда ортонормированная система полиномов Лежандра \eqref{tad-LegOrthonormed} является базисом пространства $L^{p(\cdot)}([-1,1])$.
\end{theorem}

\subsection{Базисность полиномов Якоби в $L^{p(\cdot)}_{w_{a,b}}([-1,1])$}
Другим вопросом, рассмотренным в отчетном году, является базисность полиномов Якоби с параметрами
$$-1<\alpha,\beta<-1/2$$
в весовых пространствах Лебега переменным показателем в том случае, когда параметры $a,b$ у веса могут отличаться от соответствующих параметров полиномов Якоби. Удалось показать, что условия, связывающие $p(x),a,b,\alpha,\beta$, аналогичны условиям, полученным в \cite{tad-Muckenhoupt}.
Основным результатом является следующая
\begin{theorem}\label{tad-Th1}
  Положим $-1<\alpha,\beta<-1/2$, $w_{a,b}(x)=(1-x)^a(1+x)^b$, $p(x)\in\mathcal{P}(-1,1)$ и $1<p(\pm1)<\infty$. Тогда
$$
\|S_n^{\alpha,\beta}(f)\|_{p(\cdot),w_{a,b}}([-1,1])\le c(\alpha,\beta,p)\|f\|_{p(\cdot),w_{a,b}}([-1,1]),
$$
если
$$
\left|\frac{a+1}{p(1)}-\frac{\alpha+1}2\right|<\frac{\alpha+1}2,
$$
$$
\left|\frac{b+1}{p(-1)}-\frac{\beta+1}2\right|<\frac{\beta+1}2.
$$
\end{theorem}
\subsection{Заключение}
В отчетном году удалось усилить результат, полученный в \cite{tad-SHII-Leg}, путем устранения условия постоянства переменного показателя у концов отрезка ортогональности полиномов Лежандра.

Кроме того, были получены условия базиcности в $L^{p(\cdot)}_{w_{a,b}}([-1,1])$ для полиномов Якоби ($-1<\alpha,\beta<-1/2$), которые аналогичны условиям, полученным Б. Мукенхауптом в \cite{tad-Muckenhoupt}.


%\begin{thebibliography}{50}
%\bibitem{tad-Pollard-1} Pollard H. The mean convergence of orthogonal series // Trans. Amer. Math. Soc.
%--- 1947. --- Vol. 62. --- Issue 2. --- Pp. 387---403.
%\bibitem{tad-Pollard-2} Pollard H. The mean convergence of orthogonal series. II // Trans. Amer. Math. Soc.
%--- 1948. --- Vol. 63. --- Issue 2. --- Pp. 355---367.
%\bibitem{tad-Pollard-3} Pollard H. The mean convergence of orthogonal series. III // Duke Math. J. --- 1949.
%--- Vol. 16. --- Issue 1. --- Pp. 189---191.
%\bibitem{tad-Newman-Rudin} Newman J., Rudin W. Mean convergence of orthogonal series // Proc. Amer. Math.
%Soc. --- 1952. --- Vol. 3. --- Issue 2. --- Pp. 219---222.
%\bibitem{tad-Muckenhoupt}Muckenhoupt B. Mean convergence of Jacobi series // Proc. Amer. Math. Soc. --- 1969.
%--- Vol. 23. --- Issue 2. --- Pp. 306---310.
%
%\bibitem{tad-SHII-Leg} О базисности системы полиномов Лежандра в пространстве Лебега $L^{p(x)}(-1,1)$ с переменным показателем $p(x)$ // Мат. сборник. --- 2009. --- Том 200, № 1. --- С. 137---160.
%\bibitem{tad-SHII-Jacob}Шарапудинов И.И., Шах-Эмиров Т.Н. Сходимость рядов Фурье по полиномам Якоби в весовом пространстве Лебега с переменным показателем // Дагестанские электронные математические известия. --- 2017. --- Вып. 8. --- С. 27---47.
%%\bibitem{tad-MMG-Haar} Магомед-Касумов М.Г. Базисность системы Хаара в весовых пространствах Лебега с переменным показателем // Владикавк. матем. журн. --- 2014. Том 16, № 3. --- С. 38---46.
%\bibitem{tad-SHII-Ult}Шарапудинов И.И. О базисности ультрасферических полиномов Якоби в весовом пространстве Лебега с переменным показателем // Мат. заметки. --- 2019. --- Том 106, № 4. --- С. 616---638.
%\bibitem{tad-RAM-Jacob} Shakh-Emirov T.N., Gadzhimirzaev R.M.  The Convergence of the Fourier–Jacobi Series in Weighted Variable Exponent Lebesgue Spaces // Operator Theory and Differential Equations. Trends in Mathematics. Birkhäuser, Cham.  In: Kusraev, A.G., Totieva, Z.D. (eds). --- 2021. --- Pp. 205---227.
%\bibitem{tad-lpxtopology} Шарапудинов И.И. О топологии пространства $L^{p(t)}([0,1])$ // Мат. заметки. --- 1979. --- Том 26, № 4. --- С. 613---632.
%
%%\bibitem{tad-ShII-MonogLpx} Шарапудинов И.И. Некоторые вопросы теории приближения в пространствах Лебега с переменным показателем.  --- Владикавказ: ЮМИ ВНЦ РАН и РСО-А, 2012. --- 267 с.
%
%\end{thebibliography}
%\end{document}
